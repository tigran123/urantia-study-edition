\upaper{190}{Morontia Appearances of Jesus}
\uminitoc{Heralds of the Resurrection}
\uminitoc{Jesus’ Appearance at Bethany}
\uminitoc{At the Home of Joseph}
\uminitoc{Appearance to the Greeks}
\uminitoc{The Walk with Two Brothers}
\author{Midwayer Commission}
\vs p190 0:1 The resurrected Jesus now prepares to spend a short period on Urantia for the purpose of experiencing the ascending morontia career of a mortal of the realms. Although this time of the morontia life is to be spent on the world of his mortal incarnation, it will, however, be in all respects the counterpart of the experience of Satania mortals who pass through the progressive morontia life of the seven mansion worlds of Jerusem.
\vs p190 0:2 All this power which is inherent in Jesus --- the endowment of life --- and which enabled him to rise from the dead, is the very gift of eternal life which he bestows upon kingdom believers, and which even now makes certain their resurrection from the bonds of natural death.
\vs p190 0:3 The mortals of the realms will arise in the morning of the resurrection with the same type of transition or morontia body that Jesus had when he arose from the tomb on this Sunday morning. These bodies do not have circulating blood, and such beings do not partake of ordinary material food; nevertheless, these morontia forms are \bibemph{real.} When the various believers saw Jesus after his resurrection, they really saw him; they were not the self\hyp{}deceived victims of visions or hallucinations.
\vs p190 0:4 Abiding faith in the resurrection of Jesus was the cardinal feature of the faith of all branches of the early gospel teaching. In Jerusalem, Alexandria, Antioch, and Philadelphia all the gospel teachers united in this implicit faith in the Master’s resurrection.
\vs p190 0:5 \pc In viewing the prominent part which Mary Magdalene took in proclaiming the Master’s resurrection, it should be recorded that Mary was the chief spokesman for the women’s corps, as was Peter for the apostles. Mary was not chief of the women workers, but she was their chief teacher and public spokesman. Mary had become a woman of great circumspection, so that her boldness in speaking to a man whom she considered to be the caretaker of Joseph’s garden only indicates how horrified she was to find the tomb empty. It was the depth and agony of her love, the fullness of her devotion, that caused her to forget, for a moment, the conventional restraints of a Jewish woman’s approach to a strange man.
\usection{Heralds of the Resurrection}
\vs p190 1:1 The apostles did not want Jesus to leave them; therefore had they slighted all his statements about dying, along with his promises to rise again. They were not expecting the resurrection as it came, and they refused to believe until they were confronted with the compulsion of unimpeachable evidence and the absolute proof of their own experiences.
\vs p190 1:2 When the apostles refused to believe the report of the five women who represented that they had seen Jesus and talked with him, Mary Magdalene returned to the tomb, and the others went back to Joseph’s house, where they related their experiences to his daughter and the other women. And the women believed their report. Shortly after 6:00 the daughter of Joseph of Arimathea and the 4 women who had seen Jesus went over to the home of Nicodemus, where they related all these happenings to Joseph, Nicodemus, David Zebedee, and the other men there assembled. Nicodemus and the others doubted their story, doubted that Jesus had risen from the dead; they conjectured that the Jews had removed the body. Joseph and David were disposed to believe the report, so much so that they hurried out to inspect the tomb, and they found everything just as the women had described. And they were the last to so view the sepulchre, for the high priest sent the captain of the temple guards to the tomb at 7:30 to remove the grave cloths. The captain wrapped them all up in the linen sheet and threw them over a near-by cliff.
\vs p190 1:3 From the tomb David and Joseph went immediately to the home of Elijah Mark, where they held a conference with the ten apostles in the upper chamber. Only John Zebedee was disposed to believe, even faintly, that Jesus had risen from the dead. Peter had believed at first but, when he failed to find the Master, fell into grave doubting. They were all disposed to believe that the Jews had removed the body. David would not argue with them, but when he left, he said: “You are the apostles, and you ought to understand these things. I will not contend with you; nevertheless, I now go back to the home of Nicodemus, where I have appointed with the messengers to assemble this morning, and when they have gathered together, I will send them forth on their last mission, as heralds of the Master’s resurrection. I heard the Master say that, after he should die, he would rise on the third day, and I believe him.” And thus speaking to the dejected and forlorn ambassadors of the kingdom, this self\hyp{}appointed chief of communication and intelligence took leave of the apostles. On his way from the upper chamber he dropped the bag of Judas, containing all the apostolic funds, in the lap of Matthew Levi.
\vs p190 1:4 It was about 9:30 when the last of David’s 26 messengers arrived at the home of Nicodemus. David promptly assembled them in the spacious courtyard and addressed them:
\vs p190 1:5 \pc “Men and brethren, all this time you have served me in accordance with your oath to me and to one another, and I call you to witness that I have never yet sent out false information at your hands. I am about to send you on your last mission as volunteer messengers of the kingdom, and in so doing I release you from your oaths and thereby disband the messenger corps. Men, I declare to you that we have finished our work. No more does the Master have need of mortal messengers; he has risen from the dead. He told us before they arrested him that he would die and rise again on the third day. I have seen the tomb --- it is empty. I have talked with Mary Magdalene and four other women, who have talked with Jesus. I now disband you, bid you farewell, and send you on your respective assignments, and the message which you shall bear to the believers is: ‘Jesus has risen from the dead; the tomb is empty.’”
\vs p190 1:6 \pc The majority of those present endeavoured to persuade David not to do this. But they could not influence him. They then sought to dissuade the messengers, but they would not heed the words of doubt. And so, shortly before 10:00 this Sunday morning, these 26 runners went forth as the first heralds of the mighty truth\hyp{}fact of the resurrected Jesus. And they started out on this mission as they had on so many others, in fulfilment of their oath to David Zebedee and to one another. These men had great confidence in David. They departed on this assignment without even tarrying to talk with those who had seen Jesus; they took David at his word. The majority of them believed what David had told them, and even those who somewhat doubted, carried the message just as certainly and just as swiftly.
\vs p190 1:7 \pc The apostles, the spiritual corps of the kingdom, are this day assembled in the upper chamber, where they manifest fear and express doubts, while these laymen, representing the first attempt at the socialization of the Master’s gospel of the brotherhood of man, under the orders of their fearless and efficient leader, go forth to proclaim the risen Saviour of a world and a universe. And they engage in this eventful service ere his chosen representatives are willing to believe his word or to accept the evidence of eyewitnesses.
\vs p190 1:8 \pc These 26 were dispatched to the home of Lazarus in Bethany and to all of the believer centres, from Beersheba in the south to Damascus and Sidon in the north; and from Philadelphia in the east to Alexandria in the west.
\vs p190 1:9 When David had taken leave of his brethren, he went over to the home of Joseph for his mother, and they then went out to Bethany to join the waiting family of Jesus. David abode there in Bethany with Martha and Mary until after they had disposed of their earthly possessions, and he accompanied them on their journey to join their brother, Lazarus, at Philadelphia.
\vs p190 1:10 In about one week from this time John Zebedee took Mary the mother of Jesus to his home in Bethsaida. James, Jesus’ eldest brother, remained with his family in Jerusalem. Ruth remained at Bethany with Lazarus’s sisters. The rest of Jesus’ family returned to Galilee. David Zebedee left Bethany with Martha and Mary, for Philadelphia, early in June, the day after his marriage to Ruth, Jesus’ youngest sister.
\usection{Jesus’ Appearance at Bethany}
\vs p190 2:1 From the time of the morontia resurrection until the hour of his spirit ascension on high, Jesus made 19 separate appearances in visible form to his believers on earth. He did not appear to his enemies nor to those who could not make spiritual use of his manifestation in visible form. His 1\ts{st} appearance was to the 5 women at the tomb; his 2\ts{nd}, to Mary Magdalene, also at the tomb.
\vs p190 2:2 The 3\ts{rd} appearance occurred about noon of this Sunday at Bethany. Shortly after noontide, Jesus’ oldest brother, James, was standing in the garden of Lazarus before the empty tomb of the resurrected brother of Martha and Mary, turning over in his mind the news brought to them about one hour previously by the messenger of David. James had always inclined to believe in his eldest brother’s mission on earth, but he had long since lost contact with Jesus’ work and had drifted into grave doubting regarding the later claims of the apostles that Jesus was the Messiah. The whole family was startled and well\hyp{}nigh confounded by the news brought by the messenger. Even as James stood before Lazarus’s empty tomb, Mary Magdalene arrived on the scene and was excitedly relating to the family her experiences of the early morning hours at the tomb of Joseph. Before she had finished, David Zebedee and his mother arrived. Ruth, of course, believed the report, and so did Jude after he had talked with David and Salome.
\vs p190 2:3 In the meantime, as they looked for James and before they found him, while he stood there in the garden near the tomb, he became aware of a near-by presence, as if someone had touched him on the shoulder; and when he turned to look, he beheld the gradual appearance of a strange form by his side. He was too much amazed to speak and too frightened to flee. And then the strange form spoke, saying: \textcolour{ubdarkred}{“James, I come to call you to the service of the kingdom. Join earnest hands with your brethren and follow after me.”} When James heard his name spoken, he knew that it was his eldest brother, Jesus, who had addressed him. They all had more or less difficulty in recognizing the morontia form of the Master, but few of them had any trouble recognizing his voice or otherwise identifying his charming personality when he once began to communicate with them.
\vs p190 2:4 When James perceived that Jesus was addressing him, he started to fall to his knees, exclaiming, “My father and my brother,” but Jesus bade him stand while he spoke with him. And they walked through the garden and talked for almost three minutes; talked over experiences of former days and forecast the events of the near future. As they neared the house, Jesus said, \textcolour{ubdarkred}{“Farewell, James, until I greet you all together.”}
\vs p190 2:5 James rushed into the house, even while they looked for him at Bethpage, exclaiming: “I have just seen Jesus and talked with him, visited with him. He is not dead; he has risen! He vanished before me, saying, ‘Farewell until I greet you all together.’” He had scarcely finished speaking when Jude returned, and he retold the experience of meeting Jesus in the garden for the benefit of Jude. And they all began to believe in the resurrection of Jesus. James now announced that he would not return to Galilee, and David exclaimed: “He is seen not only by excited women; even stronghearted men have begun to see him. I expect to see him myself.”
\vs p190 2:6 \pc And David did not long wait, for the 4\ts{th} appearance of Jesus to mortal recognition occurred shortly before 14:00 in this very home of Martha and Mary, when he appeared visibly before his earthly family and their friends, 20 in all. The Master appeared in the open back door, saying: \textcolour{ubdarkred}{“Peace be upon you. Greetings to those once near me in the flesh and fellowship for my brothers and sisters in the kingdom of heaven. How could you doubt? Why have you lingered so long before choosing to follow the light of truth with a whole heart? Come, therefore, all of you into the fellowship of the Spirit of Truth in the Father’s kingdom.”} As they began to recover from the first shock of their amazement and to move toward him as if to embrace him, he vanished from their sight.
\vs p190 2:7 \pc They all wanted to rush off to the city to tell the doubting apostles about what had happened, but James restrained them. Mary Magdalene, only, was permitted to return to Joseph’s house. James forbade their publishing abroad the fact of this morontia visit because of certain things which Jesus had said to him as they conversed in the garden. But James never revealed more of his visit with the risen Master on this day at the Lazarus home in Bethany.
\usection{At the Home of Joseph}
\vs p190 3:1 The 5\ts{th} morontia manifestation of Jesus to the recognition of mortal eyes occurred in the presence of some 25 women believers assembled at the home of Joseph of Arimathea, at about 16:15 on this same Sunday afternoon. Mary Magdalene had returned to Joseph’s house just a few minutes before this appearance. James, Jesus’ brother, had requested that nothing be said to the apostles concerning the Master’s appearance at Bethany. He had not asked Mary to refrain from reporting the occurrence to her sister believers. Accordingly, after Mary had pledged all the women to secrecy, she proceeded to relate what had so recently happened while she was with Jesus’ family at Bethany. And she was in the very midst of this thrilling recital when a sudden and solemn hush fell over them; they beheld in their very midst the fully visible form of the risen Jesus. He greeted them, saying: \textcolour{ubdarkred}{“Peace be upon you. In the fellowship of the kingdom there shall be neither Jew nor gentile, rich nor poor, free nor bond, man nor woman. You also are called to publish the good news of the liberty of mankind through the gospel of sonship with God in the kingdom of heaven. Go to all the world proclaiming this gospel and confirming believers in the faith thereof. And while you do this, forget not to minister to the sick and strengthen those who are fainthearted and fear\hyp{}ridden. And I will be with you always, even to the ends of the earth.”} And when he had thus spoken, he vanished from their sight, while the women fell on their faces and worshipped in silence.
\vs p190 3:2 \pc Of the 5 morontia appearances of Jesus occurring up to this time, Mary Magdalene had witnessed 4.
\vs p190 3:3 \pc As a result of sending out the messengers during the midforenoon and from the unconscious leakage of intimations concerning this appearance of Jesus at Joseph’s house, word began to come to the rulers of the Jews during the early evening that it was being reported about the city that Jesus had risen, and that many persons were claiming to have seen him. The Sanhedrists were thoroughly aroused by these rumours. After a hasty consultation with Annas, Caiaphas called a meeting of the Sanhedrin to convene at 20:00 that evening. It was at this meeting that action was taken to throw out of the synagogues any person who made mention of Jesus’ resurrection. It was even suggested that anyone claiming to have seen him should be put to death; this proposal, however, did not come to a vote since the meeting broke up in confusion bordering on actual panic. They had dared to think they were through with Jesus. They were about to discover that their real trouble with the man of Nazareth had just begun.
\usection{Appearance to the Greeks}
\vs p190 4:1 About 16:30, at the home of one Flavius, the Master made his 6\ts{th} morontia appearance to some 40 Greek believers there assembled. While they were engaged in discussing the reports of the Master’s resurrection, he manifested himself in their midst, notwithstanding that the doors were securely fastened, and speaking to them, said: \textcolour{ubdarkred}{“Peace be upon you. While the Son of Man appeared on earth among the Jews, he came to minister to all men. In the kingdom of my Father there shall be neither Jew nor gentile; you will all be brethren --- the sons of God. Go you, therefore, to all the world, proclaiming this gospel of salvation as you have received it from the ambassadors of the kingdom, and I will fellowship you in the brotherhood of the Father’s sons of faith and truth.”} And when he had thus charged them, he took leave, and they saw him no more. They remained within the house all evening; they were too much overcome with awe and fear to venture forth. Neither did any of these Greeks sleep that night; they stayed awake discussing these things and hoping that the Master might again visit them. Among this group were many of the Greeks who were at Gethsemane when the soldiers arrested Jesus and Judas betrayed him with a kiss.
\vs p190 4:2 \pc Rumours of Jesus’ resurrection and reports concerning the many appearances to his followers are spreading rapidly, and the whole city is being wrought up to a high pitch of excitement. Already the Master has appeared to his family, to the women, and to the Greeks, and presently he manifests himself in the midst of the apostles. The Sanhedrin is soon to begin the consideration of these new problems which have been so suddenly thrust upon the Jewish rulers. Jesus thinks much about his apostles but desires that they be left alone for a few more hours of solemn reflection and thoughtful consideration before he visits them.
\usection{The Walk with Two Brothers}
\vs p190 5:1 At Emmaus, about 11\,km west of Jerusalem, there lived two brothers, shepherds, who had spent the Passover week in Jerusalem attending upon the sacrifices, ceremonials, and feasts. Cleopas, the elder, was a partial believer in Jesus; at least he had been cast out of the synagogue. His brother, Jacob, was not a believer, although he was much intrigued by what he had heard about the Master’s teachings and works.
\vs p190 5:2 On this Sunday afternoon, about 4.8\,km out of Jerusalem and a few minutes before 17:00, as these two brothers trudged along the road to Emmaus, they talked in great earnestness about Jesus, his teachings, work, and more especially concerning the rumours that his tomb was empty, and that certain of the women had talked with him. Cleopas was half a mind to believe these reports, but Jacob was insistent that the whole affair was probably a fraud. While they thus argued and debated as they made their way toward home, the morontia manifestation of Jesus, his 7\ts{th} appearance, came alongside them as they journeyed on. Cleopas had often heard Jesus teach and had eaten with him at the homes of Jerusalem believers on several occasions. But he did not recognize the Master even when he spoke freely with them.
\vs p190 5:3 After walking a short way with them, Jesus said: \textcolour{ubdarkred}{“What were the words you exchanged so earnestly as I came upon you?”} And when Jesus had spoken, they stood still and viewed him with sad surprise. Said Cleopas: “Can it be that you sojourn in Jerusalem and know not the things which have recently happened?” Then asked the Master, “What things?” Cleopas replied: “If you do not know about these matters, you are the only one in Jerusalem who has not heard these rumours concerning Jesus of Nazareth, who was a prophet mighty in word and in deed before God and all the people. The chief priests and our rulers delivered him up to the Romans and demanded that they crucify him. Now many of us had hoped that it was he who would deliver Israel from the yoke of the gentiles. But that is not all. It is now the third day since he was crucified, and certain women have this day amazed us by declaring that very early this morning they went to his tomb and found it empty. And these same women insist that they talked with this man; they maintain that he has risen from the dead. And when the women reported this to the men, two of his apostles ran to the tomb and likewise found it empty” --- and here Jacob interrupted his brother to say, “but they did not see Jesus.”\tunemarkup{private}{\begin{figure}[H]\centering\includegraphics[width=\tunemarkup{pgkoboaurahd}{0.7}\tunemarkup{pgnexus10}{0.9}\columnwidth]{images/Two-Brothers.jpg}\caption{Two Brothers from Emmaus by Michael Malm}\end{figure}}
\vs p190 5:4 As they walked along, Jesus said to them: \textcolour{ubdarkred}{“How slow you are to comprehend the truth! When you tell me that it is about the teachings and work of this man that you have your discussions, then may I enlighten you since I am more than familiar with these teachings. Do you not remember that this Jesus always taught that his kingdom was not of this world, and that all men, being the sons of God, should find liberty and freedom in the spiritual joy of the fellowship of the brotherhood of loving service in this new kingdom of the truth of the heavenly Father’s love? Do you not recall how this Son of Man proclaimed the salvation of God for all men, ministering to the sick and afflicted and setting free those who were bound by fear and enslaved by evil? Do you not know that this man of Nazareth told his disciples that he must go to Jerusalem, be delivered up to his enemies, who would put him to death, and that he would arise on the third day? Have you not been told all this? And have you never read in the Scriptures concerning this day of salvation for Jew and gentile, where it says that in him shall all the families of the earth be blessed; that he will hear the cry of the needy and save the souls of the poor who seek him; that all nations shall call him blessed? That such a Deliverer shall be as the shadow of a great rock in a weary land. That he will feed the flock like a true shepherd, gathering the lambs in his arms and tenderly carrying them in his bosom. That he will open the eyes of the spiritually blind and bring the prisoners of despair out into full liberty and light; that all who sit in darkness shall see the great light of eternal salvation. That he will bind up the brokenhearted, proclaim liberty to the captives of sin, and open up the prison to those who are enslaved by fear and bound by evil. That he will comfort those who mourn and bestow upon them the joy of salvation in the place of sorrow and heaviness. That he shall be the desire of all nations and the everlasting joy of those who seek righteousness. That this Son of truth and righteousness shall rise upon the world with healing light and saving power; even that he will save his people from their sins; that he will really seek and save those who are lost. That he will not destroy the weak but minister salvation to all who hunger and thirst for righteousness. That those who believe in him shall have eternal life. That he will pour out his spirit upon all flesh, and that this Spirit of Truth shall be in each believer a well of water, springing up into everlasting life. Did you not understand how great was the gospel of the kingdom which this man delivered to you? Do you not perceive how great a salvation has come upon you?”}
\vs p190 5:5 By this time they had come near to the village where these brothers dwelt. Not a word had these two men spoken since Jesus began to teach them as they walked along the way. Soon they drew up in front of their humble dwelling place, and Jesus was about to take leave of them, going on down the road, but they constrained him to come in and abide with them. They insisted that it was near nightfall, and that he tarry with them. Finally Jesus consented, and very soon after they went into the house, they sat down to eat. They gave him the bread to bless, and as he began to break and hand to them, their eyes were opened, and Cleopas recognized that their guest was the Master himself. And when he said, “It is the Master ---,” the morontia Jesus vanished from their sight.
\vs p190 5:6 And then they said, the one to the other, “No wonder our hearts burned within us as he spoke to us while we walked along the road! and while he opened up to our understanding the teachings of the Scriptures!”
\vs p190 5:7 They would not stop to eat. They had seen the morontia Master, and they rushed from the house, hastening back to Jerusalem to spread the good news of the risen Saviour.
\vs p190 5:8 About 21:00 that evening and just before the Master appeared to the 10, these 2 excited brothers broke in upon the apostles in the upper chamber, declaring that they had seen Jesus and talked with him. And they told all that Jesus had said to them and how they had not discerned who he was until the time of the breaking of the bread.
\quizlink
