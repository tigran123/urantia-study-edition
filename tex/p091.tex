\upaper{91}{The Evolution of Prayer}
\uminitoc{Primitive Prayer}
\uminitoc{Evolving Prayer}
\uminitoc{Prayer and the Alter Ego}
\uminitoc{Ethical Praying}
\uminitoc{Social Repercussions of Prayer}
\uminitoc{The Province of Prayer}
\uminitoc{Mysticism, Ecstasy, and Inspiration}
\uminitoc{Praying as a Personal Experience}
\uminitoc{Conditions of Effective Prayer}
\author{Chief of Midwayers}
\vs p091 0:1 Prayer, as an agency of religion, evolved from previous nonreligious monologue and dialogue expressions.\tunemarkup{pgnexus10}{\linebreak} With the attainment of self\hyp{}consciousness by primitive man there occurred the inevitable corollary of other\hyp{}consciousness, the dual potential of social response and God recognition.
\vs p091 0:2 The earliest prayer forms were not addressed to Deity. These expressions were much like what you would say to a friend as you entered upon some important undertaking, “Wish me luck.” Primitive man was enslaved to magic; luck, good and bad, entered into all the affairs of life. At first, these luck petitions were monologues --- just a kind of thinking out loud by the magic server. Next, these believers in luck would enlist the support of their friends and families, and presently some form of ceremony would be performed which included the whole clan or tribe.
\vs p091 0:3 When the concepts of ghosts and spirits evolved, these petitions became superhuman in address, and with the consciousness of gods, such expressions attained to the levels of genuine prayer. As an illustration of this, among certain Australian tribes primitive religious prayers antedated their belief in spirits and superhuman personalities.
\vs p091 0:4 The Toda tribe of India now observes this practice of praying to no one in particular, just as did the early peoples before the times of religious consciousness. Only, among the Todas, this represents a regression of their degenerating religion to this primitive level. The present\hyp{}day rituals of the dairymen priests of the Todas do not represent a religious ceremony since these impersonal prayers do not contribute anything to the conservation or enhancement of any social, moral, or spiritual values.
\vs p091 0:5 Prereligious praying was part of the mana practices of the Melanesians, the oudah beliefs of the African Pygmies, and the manitou superstitions of the North American Indians. The Baganda tribes of Africa have only recently emerged from the mana level of prayer. In this early evolutionary confusion men pray to gods --- local and national --- to fetishes, amulets, ghosts, rulers, and to ordinary people.
\usection{Primitive Prayer}
\vs p091 1:1 The function of early evolutionary religion is to conserve and augment the essential social, moral, and spiritual values which are slowly taking form. This mission of religion is not consciously observed by mankind, but it is chiefly effected by the function of prayer. The practice of prayer represents the unintended, but nonetheless personal and collective, effort of any group to secure (to actualize) this conservation of higher values. But for the safeguarding of prayer, all holy days would speedily revert to the status of mere holidays.
\vs p091 1:2 \pc Religion and its agencies, the chief of which is prayer, are allied only with those values which have general social recognition, group approval. Therefore, when primitive man attempted to gratify his baser emotions or to achieve unmitigated selfish ambitions, he was deprived of the consolation of religion and the assistance of prayer. If the individual sought to accomplish anything antisocial, he was obliged to seek the aid of nonreligious magic, resort to sorcerers, and thus be deprived of the assistance of prayer. Prayer, therefore, very early became a mighty promoter of social evolution, moral progress, and spiritual attainment.
\vs p091 1:3 But the primitive mind was neither logical nor consistent. Early men did not perceive that material things were not the province of prayer. These simple\hyp{}minded souls reasoned that food, shelter, rain, game, and other material goods enhanced the social welfare, and therefore they began to pray for these physical blessings. While this constituted a perversion of prayer, it encouraged the effort to realize these material objectives by social and ethical actions. Such a prostitution of prayer, while debasing the spiritual values of a people, nevertheless directly elevated their economic, social, and ethical mores.
\vs p091 1:4 Prayer is only monologuous in the most primitive type of mind. It early becomes a dialogue and rapidly expands to the level of group worship. Prayer signifies that the premagical incantations of primitive religion have evolved to that level where the human mind recognizes the reality of beneficent powers or beings who are able to enhance social values and to augment moral ideals, and further, that these influences are superhuman and distinct from the ego of the self\hyp{}conscious human and his fellow mortals. True prayer does not, therefore, appear until the agency of religious ministry is visualized as \bibemph{personal.}
\vs p091 1:5 \pc Prayer is little associated with animism, but such beliefs may exist alongside emerging religious sentiments. Many times, religion and animism have had entirely separate origins.
\vs p091 1:6 \pc With those mortals who have not been delivered from the primitive bondage of fear, there is a real danger that all prayer may lead to a morbid sense of sin, unjustified convictions of guilt, real or fancied. But in modern times it is not likely that many will spend sufficient time at prayer to lead to this harmful brooding over their unworthiness or sinfulness. The dangers attendant upon the distortion and perversion of prayer consist in ignorance, superstition, crystallization, devitalization, materialism, and fanaticism.
\usection{Evolving Prayer}
\vs p091 2:1 The first prayers were merely verbalized wishes, the expression of sincere desires. Prayer next became a technique of achieving spirit co\hyp{}operation. And then it attained to the higher function of assisting religion in the conservation of all worth\hyp{}while values.
\vs p091 2:2 Both prayer and magic arose as a result of man’s adjustive reactions to Urantian environment. But aside from this generalized relationship, they have little in common. Prayer has always indicated positive action by the praying ego; it has been always psychic and sometimes spiritual. Magic has usually signified an attempt to manipulate reality without affecting the ego of the manipulator, the practitioner of magic. Despite their independent origins, magic and prayer often have been interrelated in their later stages of development. Magic has sometimes ascended by goal elevation from formulas through rituals and incantations to the threshold of true prayer. Prayer has sometimes become so materialistic that it has degenerated into a pseudomagical technique of avoiding the expenditure of that effort which is requisite for the solution of Urantian problems.
\vs p091 2:3 \pc When man learned that prayer could not coerce the gods, then it became more of a petition, favour seeking. But the truest prayer is in reality a communion between man and his Maker.
\vs p091 2:4 \pc The appearance of the sacrifice idea in any religion unfailingly detracts from the higher efficacy of true prayer in that men seek to substitute the offerings of material possessions for the offering of their own consecrated wills to the doing of the will of God.
\vs p091 2:5 When religion is divested of a personal God, its prayers translate to the levels of theology and philosophy. When the highest God concept of a religion is that of an impersonal Deity, such as in pantheistic idealism, although affording the basis for certain forms of mystic communion, it proves fatal to the potency of true prayer, which always stands for man’s communion with a personal and superior being.
\vs p091 2:6 During the earlier times of racial evolution and even at the present time, in the day\hyp{}by\hyp{}day experience of the average mortal, prayer is very much a phenomenon of man’s intercourse with his own subconscious. But there is also a domain of prayer wherein the intellectually alert and spiritually progressing individual attains more or less contact with the superconscious levels of the human mind, the domain of the indwelling Thought Adjuster. In addition, there is a definite spiritual phase of true prayer which concerns its reception and recognition by the spiritual forces of the universe, and which is entirely distinct from all human and intellectual association.
\vs p091 2:7 Prayer contributes greatly to the development of the religious sentiment of an evolving human mind. It is a mighty influence working to prevent isolation of personality.
\vs p091 2:8 Prayer represents one technique associated with the natural religions of racial evolution which also forms a part of the experiential values of the higher religions of ethical excellence, the religions of revelation.
\usection{Prayer and the Alter Ego}
\vs p091 3:1 Children, when first learning to make use of language, are prone to think out loud, to express their thoughts in words, even if no one is present to hear them. With the dawn of creative imagination they evince a tendency to converse with imaginary companions. In this way a budding ego seeks to hold communion with a fictitious \bibemph{alter ego.} By this technique the child early learns to convert his monologue conversations into pseudo dialogues in which this alter ego makes replies to his verbal thinking and wish expression. Very much of an adult’s thinking is mentally carried on in conversational form.
\vs p091 3:2 The early and primitive form of prayer was much like the semimagical recitations of the present\hyp{}day Toda tribe, prayers that were not addressed to anyone in particular. But such techniques of praying tend to evolve into the dialogue type of communication by the emergence of the idea of an alter ego. In time the alter\hyp{}ego concept is exalted to a superior status of divine dignity, and prayer as an agency of religion has appeared. Through many phases and during long ages this primitive type of praying is destined to evolve before attaining the level of intelligent and truly ethical prayer.
\vs p091 3:3 As it is conceived by successive generations of praying mortals, the alter ego evolves up through ghosts, fetishes, and spirits to polytheistic gods, and eventually to the One God, a divine being embodying the highest ideals and the loftiest aspirations of the praying ego. And thus does prayer function as the most potent agency of religion in the conservation of the highest values and ideals of those who pray. From the moment of the conceiving of an alter ego to the appearance of the concept of a divine and heavenly Father, prayer is always a socializing, moralizing, and spiritualizing practice.
\vs p091 3:4 The simple prayer of faith evidences a mighty evolution in human experience where\-by the ancient conversations with the fictitious symbol of the alter ego of primitive religion have become exalted to the level of communion with the spirit of the Infinite and to that of a bona fide consciousness of the reality of the eternal God and Paradise Father of all intelligent creation.
\vs p091 3:5 Aside from all that is superself in the experience of praying, it should be remembered that ethical prayer is a splendid way to elevate one’s ego and reinforce the self for better living and higher attainment. Prayer induces the human ego to look both ways for help: for material aid to the subconscious reservoir of mortal experience, for inspiration and guidance to the superconscious borders of the contact of the material with the spiritual, with the Mystery Monitor.
\vs p091 3:6 Prayer ever has been and ever will be a twofold human experience: a psychologic procedure interassociated with a spiritual technique. And these two functions of prayer can never be fully separated.
\vs p091 3:7 Enlightened prayer must recognize not only an external and personal God but also an internal and impersonal Divinity, the indwelling Adjuster. It is altogether fitting that man, when he prays, should strive to grasp the concept of the Universal Father on Paradise; but the more effective technique for most practical purposes will be to revert to the concept of a near-by alter ego, just as the primitive mind was wont to do, and then to recognize that the idea of this alter ego has evolved from a mere fiction to the truth of God’s indwelling mortal man in the factual presence of the Adjuster so that man can talk face to face, as it were, with a real and genuine and divine alter ego that indwells him and is the very presence and essence of the living God, the Universal Father.
\usection{Ethical Praying}
\vs p091 4:1 No prayer can be ethical when the petitioner seeks for selfish advantage over his fellows. Selfish and materialistic praying is incompatible with the ethical religions which are predicated on unselfish and divine love. All such unethical praying reverts to the primitive levels of pseudo magic and is unworthy of advancing civilizations and enlightened religions. Selfish praying transgresses the spirit of all ethics founded on loving justice.
\vs p091 4:2 Prayer must never be so prostituted as to become a substitute for action. All ethical prayer is a stimulus to action and a guide to the progressive striving for idealistic goals of superself\hyp{}attainment.
\vs p091 4:3 In all your praying be \bibemph{fair;} do not expect God to show partiality, to love you more than his other children, your friends, neighbours, even enemies. But the prayer of the natural or evolved religions is not at first ethical, as it is in the later revealed religions. All praying, whether individual or communal, may be either egoistic or altruistic. That is, the prayer may be centred upon the self or upon others. When the prayer seeks nothing for the one who prays nor anything for his fellows, then such attitudes of the soul tend to the levels of true worship. Egoistic prayers involve confessions and petitions and often consist in requests for material favours. Prayer is somewhat more ethical when it deals with forgiveness and seeks wisdom for enhanced self\hyp{}control.
\vs p091 4:4 While the nonselfish type of prayer is strengthening and comforting, materialistic praying is destined to bring disappointment and disillusionment as advancing scientific discoveries demonstrate that man lives in a physical universe of law and order. The childhood of an individual or a race is characterized by primitive, selfish, and materialistic praying. And, to a certain extent, all such petitions are efficacious in that they unvaryingly lead to those efforts and exertions which are contributory to achieving the answers to such prayers. The real prayer of faith always contributes to the augmentation of the technique of living, even if such petitions are not worthy of spiritual recognition. But the spiritually advanced person should exercise great caution in attempting to discourage the primitive or immature mind regarding such prayers.
\vs p091 4:5 \pc Remember, even if prayer does not change God, it very often effects great and lasting changes in the one who prays in faith and confident expectation. Prayer has been the ancestor of much peace of mind, cheerfulness, calmness, courage, self\hyp{}mastery, and fair\hyp{}mindedness in the men and women of the evolving races.
\usection{Social Repercussions of Prayer}
\vs p091 5:1 In ancestor worship, prayer leads to the cultivation of ancestral ideals. But prayer, as a feature of Deity worship, transcends all other such practices since it leads to the cultivation of divine ideals. As the concept of the alter ego of prayer becomes supreme and divine, so are man’s ideals accordingly elevated from mere human toward supernal and divine levels, and the result of all such praying is the enhancement of human character and the profound unification of human personality.
\vs p091 5:2 But prayer need not always be individual. Group or congregational praying is very effective in that it is highly socializing in its repercussions. When a group engages in community prayer for moral enhancement and spiritual uplift, such devotions are reactive upon the individuals composing the group; they are all made better because of participation. Even a whole city or an entire nation can be helped by such prayer devotions. Confession, repentance, and prayer have led individuals, cities, nations, and whole races to mighty efforts of reform and courageous deeds of valorous achievement.
\vs p091 5:3 \pc If you truly desire to overcome the habit of criticizing some friend, the quickest and surest way of achieving such a change of attitude is to establish the habit of praying for that person every day of your life. But the social repercussions of such prayers are dependent largely on two conditions:
\vs p091 5:4 \ublistelem{1.}\bibnobreakspace The person who is prayed for should know that he is being prayed for.
\vs p091 5:5 \ublistelem{2.}\bibnobreakspace The person who prays should come into intimate social contact with the person for whom he is praying.
\vs p091 5:6 \pc Prayer is the technique whereby, sooner or later, every religion becomes institutionalized. And in time prayer becomes associated with numerous secondary agencies, some helpful, others decidedly deleterious, such as priests, holy books, worship rituals, and ceremonials.
\vs p091 5:7 But the minds of greater spiritual illumination should be patient with, and tolerant of, those less endowed intellects that crave symbolism for the mobilization of their feeble spiritual insight. The strong must not look with disdain upon the weak. Those who are God\hyp{}conscious without symbolism must not deny the grace\hyp{}ministry of the symbol to those who find it difficult to worship Deity and to revere truth, beauty, and goodness without form and ritual. In prayerful worship, most mortals envision some symbol of the object\hyp{}goal of their devotions.
\usection{The Province of Prayer}
\vs p091 6:1 Prayer, unless in liaison with the will and actions of the personal spiritual forces and material supervisors of a realm, can have no direct effect upon one’s physical environment. While there is a very definite limit to the province of the petitions of prayer, such limits do not equally apply to the \bibemph{faith} of those who pray.
\vs p091 6:2 Prayer is not a technique for curing real and organic diseases, but it has contributed enormously to the enjoyment of abundant health and to the cure of numerous mental, emotional, and nervous ailments. And even in actual bacterial disease, prayer has many times added to the efficacy of other remedial procedures. Prayer has turned many an irritable and complaining invalid into a paragon of patience and made him an inspiration to all other human sufferers.
\vs p091 6:3 No matter how difficult it may be to reconcile the scientific doubtings regarding the efficacy of prayer with the ever\hyp{}present urge to seek help and guidance from divine sources, never forget that the sincere prayer of faith is a mighty force for the promotion of personal happiness, individual self\hyp{}control, social harmony, moral progress, and spiritual attainment.
\vs p091 6:4 Prayer, even as a purely human practice, a dialogue with one’s alter ego, constitutes a technique of the most efficient approach to the realization of those reserve powers of human nature which are stored and conserved in the unconscious realms of the human mind. Prayer is a sound psychologic practice, aside from its religious implications and its spiritual significance. It is a fact of human experience that most persons, if sufficiently hard pressed, will pray in some way to some source of help.
\vs p091 6:5 \pc Do not be so slothful as to ask God to solve your difficulties, but never hesitate to ask him for wisdom and spiritual strength to guide and sustain you while you yourself resolutely and courageously attack the problems at hand.
\vs p091 6:6 \pc Prayer has been an indispensable factor in the progress and preservation of religious civilization, and it still has mighty contributions to make to the further enhancement and spiritualization of society if those who pray will only do so in the light of scientific facts, philosophic wisdom, intellectual sincerity, and spiritual faith. Pray as Jesus taught his disciples --- honestly, unselfishly, with fairness, and without doubting.
\vs p091 6:7 But the efficacy of prayer in the personal spiritual experience of the one who prays is in no way dependent on such a worshipper’s intellectual understanding, philosophic acumen, social level, cultural status, or other mortal acquirements. The psychic and spiritual concomitants of the prayer of faith are immediate, personal, and experiential. There is no other technique whereby every man, regardless of all other mortal accomplishments, can so effectively and immediately approach the threshold of that realm wherein he can communicate with his Maker, where the creature contacts with the reality of the Creator, with the indwelling Thought Adjuster.
\usection{Mysticism, Ecstasy, and Inspiration}
\vs p091 7:1 Mysticism, as the technique of the cultivation of the consciousness of the presence of God, is altogether praiseworthy, but when such practices lead to social isolation and culminate in religious fanaticism, they are all but reprehensible. Altogether too frequently that which the overwrought mystic evaluates as divine inspiration is the uprisings of his own deep mind. The contact of the mortal mind with its indwelling Adjuster, while often favoured by devoted meditation, is more frequently facilitated by wholehearted and loving service in unselfish ministry to one’s fellow creatures.
\vs p091 7:2 The great religious teachers and the prophets of past ages were not extreme mystics. They were God\hyp{}knowing men and women who best served their God by unselfish ministry to their fellow mortals. Jesus often took his apostles away by themselves for short periods to engage in meditation and prayer, but for the most part he kept them in service\hyp{}contact with the multitudes. The soul of man requires spiritual exercise as well as spiritual nourishment.
\vs p091 7:3 Religious ecstasy is permissible when resulting from sane antecedents, but such experiences are more often the outgrowth of purely emotional influences than a manifestation of deep spiritual character. Religious persons must not regard every vivid psychologic presentiment and every intense emotional experience as a divine revelation or a spiritual communication. Genuine spiritual ecstasy is usually associated with great outward calmness and almost perfect emotional control. But true prophetic vision is a superpsychologic presentiment. Such visitations are not pseudo hallucinations, neither are they trancelike ecstasies.
\vs p091 7:4 The human mind may perform in response to so\hyp{}called inspiration when it is sensitive either to the uprisings of the subconscious or to the stimulus of the superconscious. In either case it appears to the individual that such augmentations of the content of consciousness are more or less foreign. Unrestrained mystical enthusiasm and rampant religious ecstasy are not the credentials of inspiration, supposedly divine credentials.
\vs p091 7:5 The practical test of all these strange religious experiences of mysticism, ecstasy, and inspiration is to observe whether these phenomena cause an individual:
\vs p091 7:6 \ublistelem{1.}\bibnobreakspace To enjoy better and more complete physical health.
\vs p091 7:7 \ublistelem{2.}\bibnobreakspace To function more efficiently and practically in his mental life.
\vs p091 7:8 \ublistelem{3.}\bibnobreakspace More fully and joyfully to socialize his religious experience.
\vs p091 7:9 \ublistelem{4.}\bibnobreakspace More completely to spiritualize his day\hyp{}by\hyp{}day living while faithfully discharging the commonplace duties of routine mortal existence.
\vs p091 7:10 \ublistelem{5.}\bibnobreakspace To enhance his love for, and appreciation of, truth, beauty, and goodness.
\vs p091 7:11 \ublistelem{6.}\bibnobreakspace To conserve currently recognized social, moral, ethical, and spiritual values.
\vs p091 7:12 \ublistelem{7.}\bibnobreakspace To increase his spiritual insight --- God\hyp{}consciousness.
\vs p091 7:13 \pc But prayer has no real association with these exceptional religious experiences. When prayer becomes overmuch aesthetic, when it consists almost exclusively in beautiful and blissful contemplation of paradisiacal divinity, it loses much of its socializing influence and tends toward mysticism and the isolation of its devotees. There is a certain danger associated with overmuch private praying which is corrected and prevented by group praying, community devotions.
\usection{Praying as a Personal Experience}
\vs p091 8:1 There is a truly spontaneous aspect to prayer, for primitive man found himself praying long before he had any clear concept of a God. Early man was wont to pray in two diverse situations: When in dire need, he experienced the impulse to reach out for help; and when jubilant, he indulged the impulsive expression of joy.
\vs p091 8:2 \pc Prayer is not an evolution of magic; they each arose independently. Magic was an attempt to adjust Deity to conditions; prayer is the effort to adjust the personality to the will of Deity. True prayer is both moral and religious; magic is neither.
\vs p091 8:3 \pc Prayer may become an established custom; many pray because others do. Still others pray because they fear something direful may happen if they do not offer their regular supplications.
\vs p091 8:4 To some individuals prayer is the calm expression of gratitude; to others, a group expression of praise, social devotions; sometimes it is the imitation of another’s religion, while in true praying it is the sincere and trusting communication of the spiritual nature of the creature with the anywhere presence of the spirit of the Creator.
\vs p091 8:5 Prayer may be a spontaneous expression of God\hyp{}consciousness or a meaningless recitation of theologic formulas. It may be the ecstatic praise of a God\hyp{}knowing soul or the slavish obeisance of a fear\hyp{}ridden mortal. It is sometimes the pathetic expression of spiritual craving and sometimes the blatant shouting of pious phrases. Prayer may be joyous praise or a humble plea for forgiveness.
\vs p091 8:6 Prayer may be the childlike plea for the impossible or the mature entreaty for moral growth and spiritual power. A petition may be for daily bread or may embody a wholehearted yearning to find God and to do his will. It may be a wholly selfish request or a true and magnificent gesture toward the realization of unselfish brotherhood.
\vs p091 8:7 Prayer may be an angry cry for vengeance or a merciful intercession for one’s enemies. It may be the expression of a hope of changing God or the powerful technique of changing one’s self. It may be the cringing plea of a lost sinner before a supposedly stern Judge or the joyful expression of a liberated son of the living and merciful heavenly Father.
\vs p091 8:8 \pc Modern man is perplexed by the thought of talking things over with God in a purely personal way. Many have abandoned regular praying; they only pray when under unusual pressure --- in emergencies. Man should be unafraid to talk to God, but only a spiritual child would undertake to persuade, or presume to change, God.
\vs p091 8:9 \pc But real praying does attain reality. Even when the air currents are ascending, no bird can soar except by outstretched wings. Prayer elevates man because it is a technique of progressing by the utilization of the ascending spiritual currents of the universe.
\vs p091 8:10 Genuine prayer adds to spiritual growth, modifies attitudes, and yields that satisfaction which comes from communion with divinity. It is a spontaneous outburst of God\hyp{}consciousness.
\vs p091 8:11 God answers man’s prayer by giving him an increased revelation of truth, an enhanced appreciation of beauty, and an augmented concept of goodness. Prayer is a subjective gesture, but it contacts with mighty objective realities on the spiritual levels of human experience; it is a meaningful reach by the human for superhuman values. It is the most potent spiritual\hyp{}growth stimulus.
\vs p091 8:12 Words are irrelevant to prayer; they are merely the intellectual channel in which the river of spiritual supplication may chance to flow. The word value of a prayer is purely autosuggestive in private devotions and sociosuggestive in group devotions. God answers the soul’s attitude, not the words.
\vs p091 8:13 Prayer is not a technique of escape from conflict but rather a stimulus to growth in the very face of conflict. Pray only for values, not things; for growth, not for gratification.
\usection{Conditions of Effective Prayer}
\vs p091 9:1 If you would engage in effective praying, you should bear in mind the laws of prevailing petitions:
\vs p091 9:2 \ublistelem{1.}\bibnobreakspace You must qualify as a potent prayer by sincerely and courageously facing the problems of universe reality. You must possess cosmic stamina.
\vs p091 9:3 \ublistelem{2.}\bibnobreakspace You must have honestly exhausted the human capacity for human adjustment. You must have been industrious.
\vs p091 9:4 \ublistelem{3.}\bibnobreakspace You must surrender every wish of mind and every craving of soul to the transforming embrace of spiritual growth. You must have experienced an enhancement of meanings and an elevation of values.
\vs p091 9:5 \ublistelem{4.}\bibnobreakspace You must make a wholehearted choice of the divine will. You must obliterate the dead centre of indecision.
\vs p091 9:6 \ublistelem{5.}\bibnobreakspace You not only recognize the Father’s will and choose to do it, but you have effected an unqualified consecration, and a dynamic dedication, to the actual doing of the Father’s will.
\vs p091 9:7 \ublistelem{6.}\bibnobreakspace Your prayer will be directed exclusively for divine wisdom to solve the specific human problems encountered in the Paradise ascension --- the attainment of divine perfection.
\vs p091 9:8 \ublistelem{7.}\bibnobreakspace And you must have faith --- living faith.
\vsetoff
\vs p091 9:9 [Presented by the Chief of the Urantia Midwayers.]
\quizlink
