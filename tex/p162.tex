\upaper{162}{At the Feast of Tabernacles}
\uminitoc{The Dangers of the Visit to Jerusalem}
\uminitoc{The First Temple Talk}
\uminitoc{The Woman Taken in Adultery}
\uminitoc{The Feast of Tabernacles}
\uminitoc{Sermon on the Light of the World}
\uminitoc{Discourse on the Water of Life}
\uminitoc{The Discourse on Spiritual Freedom}
\uminitoc{The Visit with Martha and Mary}
\uminitoc{At Bethlehem with Abner}
\author{Midwayer Commission}
\vs p162 0:1 When Jesus started up to Jerusalem with the ten apostles, he planned to go through Samaria, that being the shorter route. Accordingly, they passed down the eastern shore of the lake and, by way of Scythopolis, entered the borders of Samaria. Near nightfall Jesus sent Philip and Matthew over to a village on the eastern slopes of Mount Gilboa to secure lodging for the company. It so happened that these villagers were greatly prejudiced against the Jews, even more so than the average Samaritans, and these feelings were heightened at this particular time as so many were on their way to the feast of tabernacles. These people knew very little about Jesus, and they refused him lodging because he and his associates were Jews. When Matthew and Philip manifested indignation and informed these Samaritans that they were declining to entertain the Holy One of Israel, the infuriated villagers chased them out of the little town with sticks and stones.
\vs p162 0:2 After Philip and Matthew had returned to their fellows and reported how they had been driven out of the village, James and John stepped up to Jesus and said: “Master, we pray you to give us permission to bid fire come down from heaven to devour these insolent and impenitent Samaritans.” But when Jesus heard these words of vengeance, he turned upon the sons of Zebedee and severely rebuked them: \textcolour{ubdarkred}{“You know not what manner of attitude you manifest. Vengeance savours not of the outlook of the kingdom of heaven. Rather than dispute, let us journey over to the little village by the Jordan ford.”} Thus because of sectarian prejudice these Samaritans denied themselves the honour of showing hospitality to the Creator Son of a universe.
\vs p162 0:3 Jesus and the ten stopped for the night at the village near the Jordan ford. Early the next day they crossed the river and continued on to Jerusalem by way of the east Jordan highway, arriving at Bethany late Wednesday evening. Thomas and Nathaniel arrived on Friday, having been delayed by their conferences with Rodan.
\vs p162 0:4 \pc Jesus and the twelve remained in the vicinity of Jerusalem until the end of the following month (October), about 4½ weeks. Jesus himself went into the city only a few times, and these brief visits were made during the days of the feast of tabernacles. He spent a considerable portion of October with Abner and his associates at Bethlehem.
\usection{The Dangers of the Visit to Jerusalem}
\vs p162 1:1 Long before they fled from Galilee, the followers of Jesus had implored him to go to Jerusalem to proclaim the gospel of the kingdom in order that his message might have the prestige of having been preached at the centre of Jewish culture and learning; but now that he had actually come to Jerusalem to teach, they were afraid for his life. Knowing that the Sanhedrin had sought to bring Jesus to Jerusalem for trial and recalling the Master’s recently reiterated declarations that he must be subject to death, the apostles had been literally stunned by his sudden decision to attend the feast of tabernacles. To all their previous entreaties that he go to Jerusalem he had replied, \textcolour{ubdarkred}{“The hour has not yet come.”} Now, to their protests of fear he answered only, \textcolour{ubdarkred}{“But the hour has come.”}
\vs p162 1:2 During the feast of tabernacles Jesus went boldly into Jerusalem on several occasions and publicly taught in the temple. This he did in spite of the efforts of his apostles to dissuade him. Though they had long urged him to proclaim his message in Jerusalem, they now feared to see him enter the city at this time, knowing full well that the scribes and Pharisees were bent on bringing about his death.
\vs p162 1:3 Jesus’ bold appearance in Jerusalem more than ever confused his followers. Many of his disciples, and even Judas Iscariot, the apostle, had dared to think that Jesus had fled in haste into Phoenicia because he feared the Jewish leaders and Herod Antipas. They failed to comprehend the significance of the Master’s movements. His presence in Jerusalem at the feast of tabernacles, even in opposition to the advice of his followers, sufficed forever to put an end to all whisperings about fear and cowardice.
\vs p162 1:4 During the feast of tabernacles, thousands of believers from all parts of the Roman Empire saw Jesus, heard him teach, and many even journeyed out to Bethany to confer with him regarding the progress of the kingdom in their home districts.
\vs p162 1:5 There were many reasons why Jesus was able publicly to preach in the temple courts throughout the days of the feast, and chief of these was the fear that had come over the officers of the Sanhedrin as a result of the secret division of sentiment in their own ranks. It was a fact that many of the members of the Sanhedrin either secretly believed in Jesus or else were decidedly averse to arresting him during the feast, when such large numbers of people were present in Jerusalem, many of whom either believed in him or were at least friendly to the spiritual movement which he sponsored.
\vs p162 1:6 The efforts of Abner and his associates throughout Judea had also done much to consolidate sentiment favourable to the kingdom, so much so that the enemies of Jesus dared not be too outspoken in their opposition. This was one of the reasons why Jesus could publicly visit Jerusalem and live to go away. One or two months before this he would certainly have been put to death.
\vs p162 1:7 But the audacious boldness of Jesus in publicly appearing in Jerusalem overawed his enemies; they were not prepared for such a daring challenge. Several times during this month the Sanhedrin made feeble attempts to place the Master under arrest, but nothing came of these efforts. His enemies were so taken aback by Jesus’ unexpected public appearance in Jerusalem that they conjectured he must have been promised protection by the Roman authorities. Knowing that Philip (Herod Antipas’s brother) was almost a follower of Jesus, the members of the Sanhedrin speculated that Philip had secured for Jesus promises of protection against his enemies. Jesus had departed from their jurisdiction before they awakened to the realization that they had been mistaken in the belief that his sudden and bold appearance in Jerusalem had been due to a secret understanding with the Roman officials.
\vs p162 1:8 Only the twelve apostles had known that Jesus intended to attend the feast of tabernacles when they had departed from Magadan. The other followers of the Master were greatly astonished when he appeared in the temple courts and began publicly to teach, and the Jewish authorities were surprised beyond expression when it was reported that he was teaching in the temple.
\vs p162 1:9 Although his disciples had not expected Jesus to attend the feast, the vast majority of the pilgrims from afar who had heard of him entertained the hope that they might see him at Jerusalem. And they were not disappointed, for on several occasions he taught in Solomon’s Porch and elsewhere in the temple courts. These teachings were really the official or formal announcement of the divinity of Jesus to the Jewish people and to the whole world.
\vs p162 1:10 The multitudes who listened to the Master’s teachings were divided in their opinions. Some said he was a good man; some a prophet; some that he was truly the Messiah; others said he was a mischievous meddler, that he was leading the people astray with his strange doctrines. His enemies hesitated to denounce him openly for fear of his friendly believers, while his friends feared to acknowledge him openly for fear of the Jewish leaders, knowing that the Sanhedrin was determined to put him to death. But even his enemies marveled at his teaching, knowing that he had not been instructed in the schools of the rabbis.
\vs p162 1:11 Every time Jesus went to Jerusalem, his apostles were filled with terror. They were the more afraid as, from day to day, they listened to his increasingly bold pronouncements regarding the nature of his mission on earth. They were unaccustomed to hearing Jesus make such positive claims and such amazing assertions even when preaching among his friends.
\usection{The First Temple Talk}
\vs p162 2:1 The first afternoon that Jesus taught in the temple, a considerable company sat listening to his words depicting the liberty of the new gospel and the joy of those who believe the good news, when a curious listener interrupted him to ask: “Teacher, how is it you can quote the Scriptures and teach the people so fluently when I am told that you are untaught in the learning of the rabbis?” Jesus replied: \textcolour{ubdarkred}{“No man has taught me the truths which I declare to you. And this teaching is not mine but His who sent me. If any man really desires to do my Father’s will, he shall certainly know about my teaching, whether it be God’s or whether I speak for myself. He who speaks for himself seeks his own glory, but when I declare the words of the Father, I thereby seek the glory of him who sent me. But before you try to enter into the new light, should you not rather follow the light you already have? Moses gave you the law, yet how many of you honestly seek to fulfil its demands? Moses in this law enjoins you, saying, ‘You shall not kill’; notwithstanding this command some of you seek to kill the Son of Man.”}
\vs p162 2:2 \pc When the crowd heard these words, they fell to wrangling among themselves. Some said he was mad; some that he had a devil. Others said this was indeed the prophet of Galilee whom the scribes and Pharisees had long sought to kill. Some said the religious authorities were afraid to molest him; others thought that they laid not hands upon him because they had become believers in him. After considerable debate one of the crowd stepped forward and asked Jesus, “Why do the rulers seek to kill you?” And he replied: \textcolour{ubdarkred}{“The rulers seek to kill me because they resent my teaching about the good news of the kingdom, a gospel that sets men free from the burdensome traditions of a formal religion of ceremonies which these teachers are determined to uphold at any cost. They circumcise in accordance with the law on the Sabbath day, but they would kill me because I once on the Sabbath day set free a man held in the bondage of affliction. They follow after me on the Sabbath to spy on me but would kill me because on another occasion I chose to make a grievously stricken man completely whole on the Sabbath day. They seek to kill me because they well know that, if you honestly believe and dare to accept my teaching, their system of traditional religion will be overthrown, forever destroyed. Thus will they be deprived of authority over that to which they have devoted their lives since they steadfastly refuse to accept this new and more glorious gospel of the kingdom of God. And now do I appeal to every one of you: Judge not according to outward appearances but rather judge by the true spirit of these teachings; judge righteously.”}
\vs p162 2:3 Then said another inquirer: “Yes, Teacher, we do look for the Messiah, but when he comes, we know that his appearance will be in mystery. We know whence you are. You have been among your brethren from the beginning. The deliverer will come in power to restore the throne of David’s kingdom. Do you really claim to be the Messiah?” And Jesus replied: \textcolour{ubdarkred}{“You claim to know me and to know whence I am. I wish your claims were true, for indeed then would you find abundant life in that knowledge. But I declare that I have not come to you for myself; I have been sent by the Father, and he who sent me is true and faithful. By refusing to hear me, you are refusing to receive Him who sends me. You, if you will receive this gospel, shall come to know Him who sent me. I know the Father, for I have come from the Father to declare and reveal him to you.”}
\vs p162 2:4 The agents of the scribes wanted to lay hands upon him, but they feared the multitude, for many believed in him. Jesus’ work since his baptism had become well known to all Jewry, and as many of these people recounted these things, they said among themselves: “Even though this teacher is from Galilee, and even though he does not meet all of our expectations of the Messiah, we wonder if the deliverer, when he does come, will really do anything more wonderful than this Jesus of Nazareth has already done.”\fnst{In 1955 text ``done?'' with the question mark. This is an indirect question contained within a declarative sentence, so the period rather than the question mark is the correct closing punctuation mark.}
\vs p162 2:5 When the Pharisees and their agents heard the people talking this way, they took counsel with their leaders and decided that something should be done forthwith to put a stop to these public appearances of Jesus in the temple courts. The leaders of the Jews, in general, were disposed to avoid a clash with Jesus, believing that the Roman authorities had promised him immunity. They could not otherwise account for his boldness in coming at this time to Jerusalem; but the officers of the Sanhedrin did not wholly believe this rumour. They reasoned that the Roman rulers would not do such a thing secretly and without the knowledge of the highest governing body of the Jewish nation.
\vs p162 2:6 Accordingly, Eber, the proper officer of the Sanhedrin, with two assistants was dispatched to arrest Jesus. As Eber made his way toward Jesus, the Master said: \textcolour{ubdarkred}{“Fear not to approach me. Draw near while you listen to my teaching. I know you have been sent to apprehend me, but you should understand that nothing will befall the Son of Man until his hour comes. You are not arrayed against me; you come only to do the bidding of your masters, and even these rulers of the Jews verily think they are doing God’s service when they secretly seek my destruction.}
\vs p162 2:7 \textcolour{ubdarkred}{“I bear none of you ill will. The Father loves you, and therefore do I long for your deliverance from the bondage of prejudice and the darkness of tradition. I offer you the liberty of life and the joy of salvation. I proclaim the new and living way, the deliverance from evil and the breaking of the bondage of sin. I have come that you might have life, and have it eternally. You seek to be rid of me and my disquieting teachings. If you could only realize that I am to be with you only a little while! In just a short time I go to Him who sent me into this world. And then will many of you diligently seek me, but you shall not discover my presence, for where I am about to go you cannot come. But all who truly seek to find me shall sometime attain the life that leads to my Father’s presence.”}
\vs p162 2:8 Some of the scoffers said among themselves: “Where will this man go that we cannot find him? Will he go to live among the Greeks? Will he destroy himself? What can he mean when he declares that soon he will depart from us, and that we cannot go where he goes?”
\vs p162 2:9 Eber and his assistants refused to arrest Jesus; they returned to their meeting place without him. When, therefore, the chief priests and the Pharisees upbraided Eber and his assistants because they had not brought Jesus with them, Eber only replied: “We feared to arrest him in the midst of the multitude because many believe in him. Besides, we never heard a man speak like this man. There is something out of the ordinary about this teacher. You would all do well to go over to hear him.” And when the chief rulers heard these words, they were astonished and spoke tauntingly to Eber: “Are you also led astray? Are you about to believe in this deceiver? Have you heard that any of our learned men or any of the rulers have believed in him? Have any of the scribes or the Pharisees been deceived by his clever teachings? How does it come that you are influenced by the behaviour of this ignorant multitude who know not the law or the prophets? Do you not know that such untaught people are accursed?” And then answered Eber: “Even so, my masters, but this man speaks to the multitude words of mercy and hope. He cheers the downhearted, and his words were comforting even to our souls. What can there be wrong in these teachings even though he may not be the Messiah of the Scriptures? And even then does not our law require fairness? Do we condemn a man before we hear him?” And the chief of the Sanhedrin was wroth with Eber and, turning upon him, said: “Have you gone mad? Are you by any chance also from Galilee? Search the Scriptures, and you will discover that out of Galilee arises no prophet, much less the Messiah.”
\vs p162 2:10 The Sanhedrin disbanded in confusion, and Jesus withdrew to Bethany for the night.
\usection{The Woman Taken in Adultery}
\vs p162 3:1 It was during this visit to Jerusalem that Jesus dealt with a certain woman of evil repute who was brought into his presence by her accusers and his enemies. The distorted record you have of this episode would suggest that this woman had been brought before Jesus by the scribes and Pharisees, and that Jesus so dealt with them as to indicate that these religious leaders of the Jews might themselves have been guilty of immorality. Jesus well knew that, while these scribes and Pharisees were spiritually blind and intellectually prejudiced by their loyalty to tradition, they were to be numbered among the most thoroughly moral men of that day and generation.
\vs p162 3:2 What really happened was this: Early the third morning of the feast, as Jesus approached the temple, he was met by a group of the hired agents of the Sanhedrin who were dragging a woman along with them. As they came near, the spokesman said: “Master, this woman was taken in adultery --- in the very act. Now, the law of Moses commands that we should stone such a woman. What do you say should be done with her?”
\vs p162 3:3 It was the plan of Jesus’ enemies, if he upheld the law of Moses requiring that the self\hyp{}confessed transgressor be stoned, to involve him in difficulty with the Roman rulers, who had denied the Jews the right to inflict the death penalty without the approval of a Roman tribunal. If he forbade stoning the woman, they would accuse him before the Sanhedrin of setting himself up above Moses and the Jewish law. If he remained silent, they would accuse him of cowardice. But the Master so managed the situation that the whole plot fell to pieces of its own sordid weight.
\vs p162 3:4 This woman, once comely, was the wife of an inferior citizen of Nazareth, a man who had been a troublemaker for Jesus throughout his youthful days. The man, having married this woman, did most shamefully force her to earn their living by making commerce of her body. He had come up to the feast at Jerusalem that his wife might thus prostitute her physical charms for financial gain. He had entered into a bargain with the hirelings of the Jewish rulers thus to betray his own wife in her commercialized vice. And so they came with the woman and her companion in transgression for the purpose of ensnaring Jesus into making some statement which could be used against him in case of his arrest.
\vs p162 3:5 Jesus, looking over the crowd, saw her husband standing behind the others. He knew what sort of man he was and perceived that he was a party to the despicable transaction. Jesus first walked around to near where this degenerate husband stood and wrote upon the sand a few words which caused him to depart in haste. Then he came back before the woman and wrote again upon the ground for the benefit of her would\hyp{}be accusers; and when they read his words, they, too, went away, one by one. And when the Master had written in the sand the third time, the woman’s companion in evil took his departure, so that, when the Master raised himself up from this writing, he beheld the woman standing alone before him. Jesus said: \textcolour{ubdarkred}{“Woman, where are your accusers? did no man remain to stone you?”} And the woman, lifting up her eyes, answered, “No man, Lord.” And then said Jesus: \textcolour{ubdarkred}{“I know about you; neither do I condemn you. Go your way in peace.”} And this woman, Hildana, forsook her wicked husband and joined herself to the disciples of the kingdom.
\usection{The Feast of Tabernacles}
\vs p162 4:1 The presence of people from all of the known world, from Spain to India, made the feast of tabernacles an ideal occasion for Jesus for the first time publicly to proclaim his full gospel in Jerusalem. At this feast the people lived much in the open air, in leafy booths. It was the feast of the harvest ingathering, and coming, as it did, in the cool of the autumn months, it was more generally attended by the Jews of the world than was the Passover at the end of the winter or Pentecost at the beginning of summer. The apostles at last beheld their Master making the bold announcement of his mission on earth before all the world, as it were.
\vs p162 4:2 This was the feast of feasts, since any sacrifice not made at the other festivals could be made at this time. This was the occasion of the reception of the temple offerings; it was a combination of vacation pleasures with the solemn rites of religious worship. Here was a time of racial rejoicing, mingled with sacrifices, Levitical chants, and the solemn blasts of the silvery trumpets of the priests. At night the impressive spectacle of the temple and its pilgrim throngs was brilliantly illuminated by the great candelabras which burned brightly in the court of the women as well as by the glare of scores of torches standing about the temple courts. The entire city was gaily decorated except the Roman castle of Antonia, which looked down in grim contrast upon this festive and worshipful scene. And how the Jews did hate this ever\hyp{}present reminder of the Roman yoke!
\vs p162 4:3 Seventy bullocks were sacrificed during the feast, the symbol of the seventy nations of heathendom. The ceremony of the outpouring of the water symbolized the outpouring of the divine spirit. This ceremony of the water followed the sunrise procession of the priests and Levites. The worshippers passed down the steps leading from the court of Israel to the court of the women while successive blasts were blown upon the silvery trumpets. And then the faithful marched on toward the beautiful gate, which opened upon the court of the gentiles. Here they turned about to face westward, to repeat their chants, and to continue their march for the symbolic water.
\vs p162 4:4 \pc On the last day of the feast almost 450 priests with a corresponding number of Levites officiated. At daybreak the pilgrims assembled from all parts of the city, each carrying in the right hand a sheaf of myrtle, willow, and palm branches, while in the left hand each one carried a branch of the paradise apple --- the citron, or the “forbidden fruit.” These pilgrims divided into three groups for this early morning ceremony. One band remained at the temple to attend the morning sacrifices; another group marched down below Jerusalem to near Maza to cut the willow branches for the adornment of the sacrificial altar, while the third group formed a procession to march from the temple behind the water priest, who, to the sound of the silvery trumpets, bore the golden pitcher which was to contain the symbolic water, out through Ophel to near Siloam, where was located the fountain gate. After the golden pitcher had been filled at the pool of Siloam, the procession marched back to the temple, entering by way of the water gate and going directly to the court of the priests, where the priest bearing the water pitcher was joined by the priest bearing the wine for the drink offering. These two priests then repaired to the silver funnels leading to the base of the altar and poured the contents of the pitchers therein. The execution of this rite of pouring the wine and the water was the signal for the assembled pilgrims to begin the chanting of the Psalms from 113 to 118 inclusive, in alternation with the Levites. And as they repeated these lines, they would wave their sheaves at the altar. Then followed the sacrifices for the day, associated with the repeating of the Psalm for the day, the Psalm for the last day of the feast being the 82\ts{nd}, beginning with the 5\ts{th} verse.
\usection{Sermon on the Light of the World}
\vs p162 5:1 On the evening of the next to the last day of the feast, when the scene was brilliantly illuminated by the lights of the candelabras and the torches, Jesus stood up in the midst of the assembled throng and said:
\vs p162 5:2 \pc \textcolour{ubdarkred}{“I am the light of the world. He who follows me shall not walk in darkness but shall have the light of life. Presuming to place me on trial and assuming to sit as my judges, you declare that, if I bear witness of myself, my witness cannot be true. But never can the creature sit in judgment on the Creator. Even if I do bear witness about myself, my witness is everlastingly true, for I know whence I came, who I am, and whither I go. You who would kill the Son of Man know not whence I came, who I am, or whither I go. You only judge by the appearances of the flesh; you do not perceive the realities of the spirit. I judge no man, not even my archenemy. But if I should choose to judge, my judgment would be true and righteous, for I would judge not alone but in association with my Father, who sent me into the world, and who is the source of all true judgment. You even allow that the witness of two reliable persons may be accepted --- well, then, I bear witness of these truths; so also does my Father in heaven. And when I told you this yesterday, in your darkness you asked me, ‘Where is your Father?’ Truly, you know neither me nor my Father, for if you had known me, you would also have known the Father.}
\vs p162 5:3 \textcolour{ubdarkred}{“I have already told you that I am going away, and that you will seek me and not find me, for where I am going you cannot come. You who would reject this light are from beneath; I am from above. You who prefer to sit in darkness are of this world; I am not of this world, and I live in the eternal light of the Father of lights. You all have had abundant opportunity to learn who I am, but you shall have still other evidence confirming the identity of the Son of Man. I am the light of life, and every one who deliberately and with understanding rejects this saving light shall die in his sins. Much I have to tell you, but you are unable to receive my words. However, he who sent me is true and faithful; my Father loves even his erring children. And all that my Father has spoken I also proclaim to the world.}
\vs p162 5:4 \textcolour{ubdarkred}{“When the Son of Man is lifted up, then shall you all know that I am he, and that I have done nothing of myself but only as the Father has taught me. I speak these words to you and to your children. And he who sent me is even now with me; he has not left me alone, for I do always that which is pleasing in his sight.”}
\vs p162 5:5 \pc As Jesus thus taught the pilgrims in the temple courts, many believed. And no man dared to lay hands upon him.
\usection{Discourse on the Water of Life}
\vs p162 6:1 On the last day, the great day of the feast, as the procession from the pool of Siloam passed through the temple courts, and just after the water and the wine had been poured down upon the altar by the priests, Jesus, standing among the pilgrims, said: \textcolour{ubdarkred}{“If any man thirst, let him come to me and drink. From the Father above I bring to this world the water of life. He who believes me shall be filled with the spirit which this water represents, for even the Scriptures have said, ‘Out of him shall flow rivers of living waters.’ When the Son of Man has finished his work on earth, there shall be poured out upon all flesh the living Spirit of Truth. Those who receive this spirit shall never know spiritual thirst.”}
\vs p162 6:2 Jesus did not interrupt the service to speak these words. He addressed the worshippers immediately after the chanting of the Hallel, the responsive reading of the Psalms accompanied by waving of the branches before the altar. Just here was a pause while the sacrifices were being prepared, and it was at this time that the pilgrims heard the fascinating voice of the Master declare that he was the giver of living water to every spirit\hyp{}thirsting soul.
\vs p162 6:3 At the conclusion of this early morning service Jesus continued to teach the multitude, saying: \textcolour{ubdarkred}{“Have you not read in the Scripture: ‘Behold, as the waters are poured out upon the dry ground and spread over the parched soil, so will I give the spirit of holiness to be poured out upon your children for a blessing even to your children’s children’? Why will you thirst for the ministry of the spirit while you seek to water your souls with the traditions of men, poured from the broken pitchers of ceremonial service? That which you see going on about this temple is the way in which your fathers sought to symbolize the bestowal of the divine spirit upon the children of faith, and you have done well to perpetuate these symbols, even down to this day. But now has come to this generation the revelation of the Father of spirits through the bestowal of his Son, and all of this will certainly be followed by the bestowal of the spirit of the Father and the Son upon the children of men. To every one who has faith shall this bestowal of the spirit become the true teacher of the way which leads to life everlasting, to the true waters of life in the kingdom of heaven on earth and in the Father’s Paradise over there.”}
\vs p162 6:4 And Jesus continued to answer the questions of both the multitude and the Pharisees. Some thought he was a prophet; some believed him to be the Messiah; others said he could not be the Christ, seeing that he came from Galilee, and that the Messiah must restore David’s throne. Still they dared not arrest him.
\usection{The Discourse on Spiritual Freedom}
\vs p162 7:1 On the afternoon of the last day of the feast and after the apostles had failed in their efforts to persuade him to flee from Jerusalem, Jesus again went into the temple to teach. Finding a large company of believers assembled in Solomon’s Porch, he spoke to them, saying:
\vs p162 7:2 \pc “If my words abide in you and you are minded to do the will of my Father, then are you truly my disciples. You shall know the truth, and the truth shall make you free. I know how you will answer me: We are the children of Abraham, and we are in bondage to none; how then shall we be made free? Even so, I do not speak of outward subjection to another’s rule; I refer to the liberties of the soul. Verily, verily, I say to you, everyone who commits sin is the bond servant of sin. And you know that the bond servant is not likely to abide forever in the master’s house. You also know that the son does remain in his father’s house. If, therefore, the Son shall make you free, shall make you sons, you shall be free indeed.
\vs p162 7:3 \textcolour{ubdarkred}{“I know that you are Abraham’s seed, yet your leaders seek to kill me because my word has not been allowed to have its transforming influence in their hearts. Their souls are sealed by prejudice and blinded by the pride of revenge. I declare to you the truth which the eternal Father shows me, while these deluded teachers seek to do the things which they have learned only from their temporal fathers. And when you reply that Abraham is your father, then do I tell you that, if you were the children of Abraham, you would do the works of Abraham. Some of you believe my teaching, but others seek to destroy me because I have told you the truth which I received from God. But Abraham did not so treat the truth of God. I perceive that some among you are determined to do the works of the evil one. If God were your Father, you would know me and love the truth which I reveal. Will you not see that I come forth from the Father, that I am sent by God, that I am not doing this work of myself? Why do you not understand my words? Is it because you have chosen to become the children of evil? If you are the children of darkness, you will hardly walk in the light of the truth which I reveal. The children of evil follow only in the ways of their father, who was a deceiver and stood not for the truth because there came to be no truth in him. But now comes the Son of Man speaking and living the truth, and many of you refuse to believe.}
\vs p162 7:4 \textcolour{ubdarkred}{“Which of you convicts me of sin? If I, then, proclaim and live the truth shown me by the Father, why do you not believe? He who is of God hears gladly the words of God; for this cause many of you hear not my words, because you are not of God. Your teachers have even presumed to say that I do my works by the power of the prince of devils. One near by has just said that I have a devil, that I am a child of the devil. But all of you who deal honestly with your own souls know full well that I am not a devil. You know that I honour the Father even while you would dishonour me. I seek not my own glory, only the glory of my Paradise Father. And I do not judge you, for there is one who judges for me.}
\vs p162 7:5 \textcolour{ubdarkred}{“Verily, verily, I say to you who believe the gospel that, if a man will keep this word of truth alive in his heart, he shall never taste death. And now just at my side a scribe says this statement proves that I have a devil, seeing that Abraham is dead, also the prophets. And he asks: ‘Are you so much greater than Abraham and the prophets that you dare to stand here and say that whoso keeps your word shall not taste death? Who do you claim to be that you dare to utter such blasphemies?’ And I say to all such that, if I glorify myself, my glory is as nothing. But it is the Father who shall glorify me, even the same Father whom you call God. But you have failed to know this your God and my Father, and I have come to bring you together; to show you how to become truly the sons of God. Though you know not the Father, I truly know him. Even Abraham rejoiced to see my day, and by faith he saw it and was glad.”}
\vs p162 7:6 \pc When the unbelieving Jews and the agents of the Sanhedrin who had gathered about by this time heard these words, they raised a tumult, shouting: “You are not 50 years of age, and yet you talk about seeing Abraham; you are a child of the devil!” Jesus was unable to continue the discourse. He only said as he departed, \textcolour{ubdarkred}{“Verily, verily, I say to you, before Abraham was, I am.”} Many of the unbelievers rushed forth for stones to cast at him, and the agents of the Sanhedrin sought to place him under arrest, but the Master quickly made his way through the temple corridors and escaped to a secret meeting place near Bethany where Martha, Mary, and Lazarus awaited him.
\usection{The Visit with Martha and Mary}
\vs p162 8:1 It had been arranged that Jesus should lodge with Lazarus and his sisters at a friend’s house, while the apostles were scattered here and there in small groups, these precautions being taken because the Jewish authorities were again becoming bold with their plans to arrest him.
\vs p162 8:2 For years it had been the custom for these three to drop everything and listen to Jesus’ teaching whenever he chanced to visit them. With the loss of their parents, Martha had assumed the responsibilities of the home life, and so on this occasion, while Lazarus and Mary sat at Jesus’ feet drinking in his refreshing teaching, Martha made ready to serve the evening meal. It should be explained that Martha was unnecessarily distracted by numerous needless tasks, and that she was cumbered by many trivial cares; that was her disposition.
\vs p162 8:3 As Martha busied herself with all these supposed duties, she was perturbed because Mary did nothing to help. Therefore she went to Jesus and said: “Master, do you not care that my sister has left me alone to do all of the serving? Will you not bid her to come and help me?” Jesus answered: \textcolour{ubdarkred}{“Martha, Martha, why are you always anxious about so many things and troubled by so many trifles? Only one thing is really worth while, and since Mary has chosen this good and needful part, I shall not take it away from her. But when will both of you learn to live as I have taught you: both serving in co\hyp{}operation and both refreshing your souls in unison? Can you not learn that there is a time for everything --- that the lesser matters of life should give way before the greater things of the heavenly kingdom?”}
\usection{At Bethlehem with Abner}
\vs p162 9:1 Throughout the week that followed the feast of tabernacles, scores of believers forgathered at Bethany and received instruction from the twelve apostles. The Sanhedrin made no effort to molest these gatherings since Jesus was not present; he was throughout this time working with Abner and his associates in Bethlehem. The day following the close of the feast, Jesus had departed for Bethany, and he did not again teach in the temple during this visit to Jerusalem.
\vs p162 9:2 \pc At this time, Abner was making his headquarters at Bethlehem, and from that centre many workers had been sent to the cities of Judea and southern Samaria and even to Alexandria. Within a few days of his arrival, Jesus and Abner completed the arrangements for the consolidation of the work of the two groups of apostles.
\vs p162 9:3 Throughout his visit to the feast of tabernacles, Jesus had divided his time about equally between Bethany and Bethlehem. At Bethany he spent considerable time with his apostles; at Bethlehem he gave much instruction to Abner and the other former apostles of John. And it was this intimate contact that finally led them to believe in him. These former apostles of John the Baptist were influenced by the courage he displayed in his public teaching in Jerusalem as well as by the sympathetic understanding they experienced in his private teaching at Bethlehem. These influences finally and fully won over each of Abner’s associates to a wholehearted acceptance of the kingdom and all that such a step implied.
\vs p162 9:4 \pc Before leaving Bethlehem for the last time, the Master made arrangements for them all to join him in the united effort which was to precede the ending of his earth career in the flesh. It was agreed that Abner and his associates were to join Jesus and the twelve in the near future at Magadan Park.
\vs p162 9:5 In accordance with this understanding, early in November Abner and his eleven fellows cast their lot with Jesus and the twelve and laboured with them as one organization right on down to the crucifixion.
\vs p162 9:6 In the latter part of October Jesus and the twelve withdrew from the immediate vicinity of Jerusalem. On Sunday, October 30, Jesus and his associates left the city of Ephraim, where he had been resting in seclusion for a few days, and, going by the west Jordan highway directly to Magadan Park, arrived late on the afternoon of Wednesday, November 2.
\vs p162 9:7 The apostles were greatly relieved to have the Master back on friendly soil; no more did they urge him to go up to Jerusalem to proclaim the gospel of the kingdom.
\quizlink
