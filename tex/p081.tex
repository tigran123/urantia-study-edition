\upaper{81}{Development of Modern Civilization}
\uminitoc{The Cradle of Civilization}
\uminitoc{The Tools of Civilization}
\uminitoc{Cities, Manufacture, and Commerce}
\uminitoc{The Mixed Races}
\uminitoc{Cultural Society}
\uminitoc{The Maintenance of Civilization}
\author{Archangel}
\vs p081 0:1 Regardless of the ups and downs of the miscarriage of the plans for world betterment projected in the missions of Caligastia and Adam, the basic organic evolution of the human species continued to carry the races forward in the scale of human progress and racial development. Evolution can be delayed but it cannot be stopped.
\vs p081 0:2 The influence of the violet race, though in numbers smaller than had been planned, produced an advance in civilization which, since the days of Adam, has far exceeded the progress of mankind throughout its entire previous existence of almost a million years.
\usection{The Cradle of Civilization}
\vs p081 1:1 For about 35,000 years after the days of Adam, the cradle of civilization was in south\hyp{}western Asia, extending from the Nile valley eastward and slightly to the north across northern Arabia, through Mesopotamia, and on into Turkestan. And \bibemph{climate} was the decisive factor in the establishment of civilization in that area.
\vs p081 1:2 It was the great climatic and geologic changes in northern Africa and western Asia that terminated the early migrations of the Adamites, barring them from Europe by the expanded Mediterranean and diverting the stream of migration north and east into Turkestan. By the time of the completion of these land elevations and associated climatic changes, about 15,000\,B.C., civilization had settled down to a world\hyp{}wide stalemate except for the cultural ferments and biologic reserves of the Andites still confined by mountains to the east in Asia and by the expanding forests in Europe to the west.
\vs p081 1:3 Climatic evolution is now about to accomplish what all other efforts had failed to do, that is, to compel Eurasian man to abandon hunting for the more advanced callings of herding and farming. Evolution may be slow, but it is terribly effective.
\vs p081 1:4 Since slaves were so generally employed by the earlier agriculturists, the farmer was formerly looked down on by both the hunter and the herder. For ages it was considered menial to till the soil; wherefore the idea that soil toil is a curse, whereas it is the greatest of all blessings. Even in the days of Cain and Abel the sacrifices of the pastoral life were held in greater esteem than the offerings of agriculture.
\vs p081 1:5 Man ordinarily evolved into a farmer from a hunter by transition through the era of the herder, and this was also true among the Andites, but more often the evolutionary coercion of climatic necessity would cause whole tribes to pass directly from hunters to successful farmers. But this phenomenon of passing immediately from hunting to agriculture only occurred in those regions where there was a high degree of race mixture with the violet stock.
\vs p081 1:6 The evolutionary peoples (notably the Chinese) early\tunemarkup{pgnexus10}{\linebreak} learned to plant seeds and to cultivate crops through observation of the sprouting of seeds accidentally moistened or which had been put in graves as food for the departed. But throughout south\hyp{}west Asia, along the fertile river bottoms and adjacent plains, the Andites were carrying out the improved agricultural techniques inherited from their ancestors, who had made farming and gardening the chief pursuits within the boundaries of the second garden.
\vs p081 1:7 For thousands of years the descendants of Adam had grown wheat and barley, as improved in the Garden, throughout the highlands of the upper border of Mesopotamia. The descendants of Adam and Adamson here met, traded, and socially mingled.
\vs p081 1:8 It was these enforced changes in living conditions which caused such a large proportion of the human race to become omnivorous in dietetic practice. And the combination of the wheat, rice, and vegetable diet with the flesh of the herds marked a great forward step in the health and vigour of these ancient peoples.
\usection{The Tools of Civilization}
\vs p081 2:1 The growth of culture is predicated upon the development of the tools of civilization. And the tools which man utilized in his ascent from savagery were effective just to the extent that they released man power for the accomplishment of higher tasks.
\vs p081 2:2 You who now live amid latter\hyp{}day scenes of budding culture and beginning progress in social affairs, who actually have some little spare time in which to \bibemph{think} about society and civilization, must not overlook the fact that your early ancestors had little or no leisure which could be devoted to thoughtful reflection and social thinking.
\vs p081 2:3 \pc The first four great advances in human civilization were:
\vs p081 2:4 \ublistelem{1.}\bibnobreakspace The taming of fire.
\vs p081 2:5 \ublistelem{2.}\bibnobreakspace The domestication of animals.
\vs p081 2:6 \ublistelem{3.}\bibnobreakspace The enslavement of captives.
\vs p081 2:7 \ublistelem{4.}\bibnobreakspace Private property.
\vs p081 2:8 \pc While fire, the first great discovery, eventually unlocked the doors of the scientific world, it was of little value in this regard to primitive man. He refused to recognize natural causes as explanations for commonplace phenomena.
\vs p081 2:9 When asked where fire came from, the simple story of Andon and the flint was soon replaced by the legend of how some Prometheus stole it from heaven. The ancients sought a supernatural explanation for all natural phenomena not within the range of their personal comprehension; and many moderns continue to do this. The depersonalization of so\hyp{}called natural phenomena has required ages, and it is not yet completed. But the frank, honest, and fearless search for true causes gave birth to modern science: It turned astrology into astronomy, alchemy into chemistry, and magic into medicine.
\vs p081 2:10 \pc In the premachine age the only way in which man could accomplish work without doing it himself was to use an animal. Domestication of animals placed in his hands living tools, the intelligent use of which prepared the way for both agriculture and transportation. And without these animals man could not have risen from his primitive estate to the levels of subsequent civilization.
\vs p081 2:11 Most of the animals best suited to domestication were found in Asia, especially in the central to south\hyp{}west regions. This was one reason why civilization progressed faster in that locality than in other parts of the world. Many of these animals had been twice before domesticated, and in the Andite age they were retamed once again. But the dog had remained with the hunters ever since being adopted by the blue man long, long before.
\vs p081 2:12 The Andites of Turkestan were the first peoples to extensively domesticate the horse\fnst{In the human source \cite{Peake1} we read: ``All the available evidence indicates that the horse was first tamed in some part of the Asiatic steppe. The more lowland portion of this steppe runs from Galicia intermittently to South Russia, and thence continuously across Russian Turkestan with a narrowed eastward extension of varying relief reaching right away to the Sea of Okhotsk.''}, and this is another reason why their culture was for so long predominant. By 5000\,B.C. the Mesopotamian, Turkestan, and Chinese farmers had begun the raising of sheep, goats, cows, camels, horses, fowls, and elephants. They employed as beasts of burden the ox, camel, horse, and yak. Man was himself at one time the beast of burden. One ruler of the blue race once had 100,000 men in his colony of burden bearers.
\vs p081 2:13 \pc The institutions of slavery and private ownership of land came with agriculture. Slavery raised the master’s standard of living and provided more leisure for social culture.
\vs p081 2:14 The savage is a slave to nature, but scientific civilization is slowly conferring increasing liberty on mankind. Through animals, fire, wind, water, electricity, and other undiscovered sources of energy, man has liberated, and will continue to liberate, himself from the necessity for unremitting toil. Regardless of the transient trouble produced by the prolific invention of machinery, the ultimate benefits to be derived from such mechanical inventions are inestimable. Civilization can never flourish, much less be established, until man has \bibemph{leisure} to think, to plan, to imagine new and better ways of doing things.
\vs p081 2:15 \pc Man first simply appropriated his shelter, lived under ledges or dwelt in caves. Next he adapted such natural materials as wood and stone to the creation of family huts. Lastly he entered the creative stage of home building, learned to manufacture brick and other building materials.
\vs p081 2:16 The peoples of the Turkestan highlands were the first of the more modern races to build their homes of wood, houses not at all unlike the early log cabins of the American pioneer settlers. Throughout the plains human dwellings were made of brick; later on, of burned bricks.
\vs p081 2:17 The older river races made their huts by setting tall poles in the ground in a circle; the tops were then brought together, making the skeleton frame for the hut, which was interlaced with transverse reeds, the whole creation resembling a huge inverted basket. This structure could then be daubed over with clay and, after drying in the sun, would make a very serviceable weatherproof habitation.
\vs p081 2:18 It was from these early huts that the subsequent idea of all sorts of basket weaving independently originated. Among one group the idea of making pottery arose from observing the effects of smearing these pole frameworks with moist clay. The practice of hardening pottery by baking was discovered when one of these clay\hyp{}covered primitive huts accidentally burned. The arts of olden days were many times derived from the accidental occurrences attendant upon the daily life of early peoples. At least, this was almost wholly true of the evolutionary progress of mankind up to the coming of Adam.
\vs p081 2:19 While pottery had been first introduced by the staff of the Prince about 500,000 years ago, the making of clay vessels had practically ceased for over 150,000 years. Only the gulf coast pre\hyp{}Sumerian Nodites continued to make clay vessels. The art of pottery making was revived during Adam’s time. The dissemination of this art was simultaneous with the extension of the desert areas of Africa, Arabia, and central Asia, and it spread in successive waves of improving technique from Mesopotamia out over the Eastern Hemisphere.
\vs p081 2:20 These civilizations of the Andite age cannot always be traced by the stages of their pottery or other arts. The smooth course of human evolution was tremendously complicated by the regimes of both Dalamatia and Eden. It often occurs that the later vases and implements are inferior to the earlier products of the purer Andite peoples.
\usection{Cities, Manufacture, and Commerce}
\vs p081 3:1 The climatic destruction of the rich, open grassland hunting and grazing grounds of Turkestan, beginning about 12,000\,B.C., compelled the men of those regions to resort to new forms of industry and crude manufacturing. Some turned to the cultivation of domesticated flocks, others became agriculturists or collectors of water\hyp{}borne food, but the higher type of Andite intellects chose to engage in trade and manufacture. It even became the custom for entire tribes to dedicate themselves to the development of a single industry. From the valley of the Nile to the Hindu Kush and from the Ganges to the Yellow River, the chief business of the superior tribes became the cultivation of the soil, with commerce as a side line.
\vs p081 3:2 The increase in trade and in the manufacture of raw materials into various articles of commerce was directly instrumental in producing those early and semipeaceful communities which were so influential in spreading the culture and the arts of civilization. Before the era of extensive world trade, social communities were tribal --- expanded family groups. Trade brought into fellowship different sorts of human beings, thus contributing to a more speedy cross\hyp{}fertilization of culture.
\vs p081 3:3 About 12,000 years ago the era of the independent cities was dawning. And these primitive trading and manufacturing cities were always surrounded by zones of agriculture and cattle raising. While it is true that industry was promoted by the elevation of the standards of living, you should have no misconception regarding the refinements of early urban life. The early races were not overly neat and clean, and the average primitive community rose 30--60\,cm every 25 years as the result of the mere accumulation of dirt and rubbish. Certain of these olden cities also rose above the surrounding ground very quickly because their unbaked mud huts were short\hyp{}lived, and it was the custom to build new dwellings directly on top of the ruins of the old.
\vs p081 3:4 \pc The widespread use of metals was a feature of this era of the early industrial and trading cities. You have already found a bronze culture in Turkestan dating before 9000\,B.C., and the Andites early learned to work in iron, gold, and copper, as well. But conditions were very different away from the more advanced centres of civilization. There were no distinct periods, such as the Stone, Bronze, and Iron Ages; all three existed at the same time in different localities.
\vs p081 3:5 Gold was the first metal to be sought by man; it was easy to work and, at first, was used only as an ornament. Copper was next employed but not extensively until it was admixed with tin to make the harder bronze. The discovery of mixing copper and tin to make bronze was made by one of the Adamsonites of Turkestan whose highland copper mine happened to be located alongside a tin deposit.
\vs p081 3:6 \pc With the appearance of crude manufacture and beginning industry, commerce quickly became the most potent influence in the spread of cultural civilization. The opening up of the trade channels by land and by sea greatly facilitated travel and the mixing of cultures as well as the blending of civilizations. By 5000\,B.C. the horse was in general use throughout civilized and semicivilized lands. These later races not only had the domesticated horse but also various sorts of wagons and chariots. Ages before, the wheel had been used, but now vehicles so equipped became universally employed both in commerce and war.
\vs p081 3:7 The travelling trader and the roving explorer did more to advance historic civilization than all other influences combined. Military conquests, colonization, and missionary enterprises fostered by the later religions were also factors in the spread of culture; but these were all secondary to the trading relations, which were ever accelerated by the rapidly developing arts and sciences of industry.
\vs p081 3:8 Infusion of the Adamic stock into the human races not only quickened the pace of civilization, but it also greatly stimulated their proclivities toward adventure and exploration to the end that most of Eurasia and northern Africa was presently occupied by the rapidly multiplying mixed descendants of the Andites.
\usection{The Mixed Races}
\vs p081 4:1 As contact is made with the dawn of historic times, all of Eurasia, northern Africa, and the Pacific Islands is overspread with the composite races of mankind. And these races of today have resulted from a blending and reblending of the five basic human stocks of Urantia.
\vs p081 4:2 Each of the Urantia races was identified by certain distinguishing physical characteristics. The Adamites and Nodites were long\hyp{}headed; the Andonites were broad\hyp{}headed. The Sangik races were medium\hyp{}headed, with the yellow and blue men tending to broad\hyp{}headedness. The blue races, when mixed with the Andonite stock, were decidedly broad\hyp{}headed. The secondary Sangiks were medium\hyp{} to long\hyp{}headed.
\vs p081 4:3 Although these skull dimensions are serviceable in deciphering racial origins, the skeleton as a whole is far more dependable. In the early development of the Urantia races there were originally five distinct types of skeletal structure:
\vs p081 4:4 \ublistelem{1.}\bibnobreakspace Andonic, Urantia aborigines.
\vs p081 4:5 \ublistelem{2.}\bibnobreakspace Primary Sangik, red, yellow, and blue.
\vs p081 4:6 \ublistelem{3.}\bibnobreakspace Secondary Sangik, orange, green, and indigo.
\vs p081 4:7 \ublistelem{4.}\bibnobreakspace Nodites, descendants of the Dalamatians.
\vs p081 4:8 \ublistelem{5.}\bibnobreakspace Adamites, the violet race.
\vs p081 4:9 \pc As these five great racial groups extensively intermingled, continual mixture tended to obscure the Andonite type by Sangik hereditary dominance. The Lapps and the Eskimos are blends of Andonite and Sangik\hyp{}blue races. Their skeletal structures come the nearest to preserving the aboriginal Andonic type. But the Adamites and the Nodites have become so admixed with the other races that they can be detected only as a generalized Caucasoid order.
\vs p081 4:10 In general, therefore, as the human remains of the last 20,000 years are unearthed, it will be impossible clearly to distinguish the five original types. Study of such skeletal structures will disclose that mankind is now divided into approximately three classes:
\vs p081 4:11 \ublistelem{1.}\bibnobreakspace \bibemph{The Caucasoid ---} the Andite blend of the Nodite and Adamic stocks, further modified by primary and (some) secondary Sangik admixture and by considerable Andonic crossing. The Occidental white races, together with some Indian and Turanian peoples, are included in this group. The unifying factor in this division is the greater or lesser proportion of Andite inheritance.
\vs p081 4:12 \ublistelem{2.}\bibnobreakspace \bibemph{The Mongoloid ---} the primary Sangik type, including the original red, yellow, and blue races. The Chinese and Amerinds belong to this group. In Europe the Mongoloid type has been modified by secondary Sangik and Andonic mixture; still more by Andite infusion. The Malayan and other Indonesian peoples are included in this classification, though they contain a high percentage of secondary Sangik blood.
\vs p081 4:13 \ublistelem{3.}\bibnobreakspace \bibemph{The Negroid ---} the secondary Sangik type, which originally included the orange, green, and indigo races. This is the type best illustrated by the Negro, and it will be found through Africa, India, and Indonesia wherever the secondary Sangik races located.
\vs p081 4:14 \pc In North China there is a certain blending of Caucasoid and Mongoloid types; in the Levant the Caucasoid and Negroid have intermingled; in India, as in South America, all three types are represented. And the skeletal characteristics of the three surviving types still persist and help to identify the later ancestry of present\hyp{}day human races.
\usection{Cultural Society}
\vs p081 5:1 Biologic evolution and cultural civilization are not necessarily correlated; organic evolution in any age may proceed unhindered in the very midst of cultural decadence. But when lengthy periods of human history are surveyed, it will be observed that eventually evolution and culture become related as cause and effect. Evolution may advance in the absence of culture, but cultural civilization does not flourish without an adequate background of antecedent racial progression. Adam and Eve introduced no art of civilization foreign to the progress of human society, but the Adamic blood did augment the inherent ability of the races and did accelerate the pace of economic development and industrial progression. Adam’s bestowal improved the brain power of the races, thereby greatly hastening the processes of natural evolution.
\vs p081 5:2 Through agriculture, animal domestication, and improved architecture, mankind gradually escaped the worst of the incessant struggle to live and began to cast about to find wherewith to sweeten the process of living; and this was the beginning of the striving for higher and ever higher standards of material comfort. Through manufacture and industry man is gradually augmenting the pleasure content of mortal life.
\vs p081 5:3 But cultural society is no great and beneficent club of inherited privilege into which all men are born with free membership and entire equality. Rather is it an exalted and ever\hyp{}advancing guild of earth workers, admitting to its ranks only the nobility of those toilers who strive to make the world a better place in which their children and their children’s children may live and advance in subsequent ages. And this guild of civilization exacts costly admission fees, imposes strict and rigorous disciplines, visits heavy penalties on all dissenters and nonconformists, while it confers few personal licenses or privileges except those of enhanced security against common dangers and racial perils.
\vs p081 5:4 Social association is a form of survival insurance which human beings have learned is profitable; therefore are most individuals willing to pay those premiums of self\hyp{}sacrifice and personal\hyp{}liberty curtailment which society exacts from its members in return for this enhanced group protection. In short, the present\hyp{}day social mechanism is a trial\hyp{}and\hyp{}error insurance plan designed to afford some degree of assurance and protection against a return to the terrible and antisocial conditions which characterized the early experiences of the human race.
\vs p081 5:5 Society thus becomes a co\hyp{}operative\tunemarkup{pgkoboaurahd}{\linebreak} scheme for securing civil freedom through institutions, economic freedom through capital and invention, social liberty through culture, and freedom from violence through police regulation.
\vs p081 5:6 \bibemph{Might does not make right, but it does enforce the commonly recognized rights of each succeeding generation.} The prime mission of government is the definition of the right, the just and fair regulation of class differences, and the enforcement of equality of opportunity under the rules of law. Every human right is associated with a social duty; group privilege is an insurance mechanism which unfailingly demands the full payment of the exacting premiums of group service. And group rights, as well as those of the individual, must be protected, including the regulation of the sex propensity.
\vs p081 5:7 Liberty subject to group regulation is the legitimate goal of social evolution. Liberty without restrictions is the vain and fanciful dream of unstable and flighty human minds.
\usection{The Maintenance of Civilization}
\vs p081 6:1 While biologic evolution has proceeded ever upward, much of cultural evolution went out from the Euphrates valley in waves, which successively weakened as time passed until finally the whole of the pure\hyp{}line Adamic posterity had gone forth to enrich the civilizations of Asia and Europe. The races did not fully blend, but their civilizations did to a considerable extent mix. Culture did slowly spread throughout the world. And this civilization must be maintained and fostered, for there exist today no new sources of culture, no Andites to invigorate and stimulate the slow progress of the evolution of civilization.
\vs p081 6:2 \pc The civilization which is now evolving on Urantia grew out of, and is predicated on, the following factors:
\vs p081 6:3 \ublistelem{1.}\bibnobreakspace \bibemph{Natural circumstances.} The nature and extent of a material civilization is in large measure determined by the natural resources available. Climate, weather, and numerous physical conditions are factors in the evolution of culture.
\vs p081 6:4 At the opening of the Andite era there were only two extensive and fertile open hunting areas in all the world. One was in North America and was overspread by the Amerinds; the other was to the north of Turkestan and was partly occupied by an Andonic\hyp{}yellow race. The decisive factors in the evolution of a superior culture in south\hyp{}western Asia were race and climate. The Andites were a great people, but the crucial factor in determining the course of their civilization was the increasing aridity of Iran, Turkestan, and Sinkiang, which \bibemph{forced} them to invent and adopt new and advanced methods of wresting a livelihood from their decreasingly fertile lands.
\vs p081 6:5 The configuration of continents and other land\hyp{}arrangement situations are very influential in determining peace or war. Very few Urantians have ever had such a favourable opportunity for continuous and unmolested development as has been enjoyed by the peoples of North America --- protected on practically all sides by vast oceans.
\vs p081 6:6 \ublistelem{2.}\bibnobreakspace \bibemph{Capital goods.} Culture is never developed under conditions of poverty; leisure is essential to the progress of civilization. Individual character of moral and spiritual value may be acquired in the absence of material wealth, but a cultural civilization is only derived from those conditions of material prosperity which foster leisure combined with ambition.
\vs p081 6:7 During primitive times life on Urantia was a serious and sober business. And it was to escape this incessant struggle and interminable toil that mankind constantly tended to drift toward the salubrious climate of the tropics. While these warmer zones of habitation afforded some remission from the intense struggle for existence, the races and tribes who thus sought ease seldom utilized their unearned leisure for the advancement of civilization. Social progress has invariably come from the thoughts and plans of those races that have, by their intelligent toil, learned how to wrest a living from the land with lessened effort and shortened days of labour and thus have been able to enjoy a well\hyp{}earned and profitable margin of leisure.
\vs p081 6:8 \ublistelem{3.}\bibnobreakspace \bibemph{Scientific knowledge.} The material aspects of civilization must always await the accumulation of scientific data. It was a long time after the discovery of the bow and arrow and the utilization of animals for power purposes before man learned how to harness wind and water, to be followed by the employment of steam and electricity. But slowly the tools of civilization improved. Weaving, pottery, the domestication of animals, and metalworking were followed by an age of writing and printing.
\vs p081 6:9 Knowledge is power. Invention always precedes the acceleration of cultural development on a world\hyp{}wide scale. Science and invention benefited most of all from the printing press, and the interaction of all these cultural and inventive activities has enormously accelerated the rate of cultural advancement.
\vs p081 6:10 Science teaches man to speak the new language of mathematics and trains his thoughts along lines of exacting precision. And science also stabilizes philosophy through the elimination of error, while it purifies religion by the destruction of superstition.
\vs p081 6:11 \ublistelem{4.}\bibnobreakspace \bibemph{Human resources.} Man power is indispensable to the spread of civilization. All things equal, a numerous people will dominate the civilization of a smaller race. Hence failure to increase in numbers up to a certain point prevents the full realization of national destiny, but there comes a point in population increase where further growth is suicidal. Multiplication of numbers beyond the optimum of the normal man\hyp{}land ratio means either a lowering of the standards of living or an immediate expansion of territorial boundaries by peaceful penetration or by military conquest, forcible occupation.
\vs p081 6:12 You are sometimes shocked at the ravages of war, but you should recognize the necessity for producing large numbers of mortals so as to afford ample opportunity for social and moral development; with such planetary fertility there soon occurs the serious problem of overpopulation. Most of the inhabited worlds are small. Urantia is average, perhaps a trifle undersized. The optimum stabilization of national population enhances culture and prevents war. And it is a wise nation which knows when to cease growing.
\vs p081 6:13 But the continent richest in natural deposits and the most advanced mechanical equipment will make little progress if the intelligence of its people is on the decline. Knowledge can be had by education, but wisdom, which is indispensable to true culture, can be secured only through experience and by men and women who are innately intelligent. Such a people are able to learn from experience; they may become truly wise.
\vs p081 6:14 \ublistelem{5.}\bibnobreakspace \bibemph{Effectiveness of material resources.} Much depends on the wisdom displayed in the utilization of natural resources, scientific knowledge, capital goods, and human potentials. The chief factor in early civilization was the \bibemph{force} exerted by wise social masters; primitive man had civilization literally thrust upon him by his superior contemporaries. Well\hyp{}organized and superior minorities have largely ruled this world.
\vs p081 6:15 Might does not make right, but might does make what is and what has been in history. Only recently has Urantia reached that point where society is willing to debate the ethics of might and right.
\vs p081 6:16 \ublistelem{6.}\bibnobreakspace \bibemph{Effectiveness of language.} The spread of civilization must wait upon language. Live and growing languages ensure the expansion of civilized thinking and planning. During the early ages important advances were made in language. Today, there is great need for further linguistic development to facilitate the expression of evolving thought.
\vs p081 6:17 Language evolved out of group associations, each local group developing its own system of word exchange. Language grew up through gestures, signs, cries, imitative sounds, intonation, and accent to the vocalization of subsequent alphabets. Language is man’s greatest and most serviceable thinking tool, but it never flourished until social groups acquired some leisure. The tendency to play with language develops new words --- slang. If the majority adopt the slang, then usage constitutes it language. The origin of dialects is illustrated by the indulgence in “baby talk” in a family group.
\vs p081 6:18 Language differences have ever been the great barrier to the extension of peace. The conquest of dialects must precede the spread of a culture throughout a race, over a continent, or to a whole world. A universal language promotes peace, ensures culture, and augments happiness. Even when the tongues of a world are reduced to a few, the mastery of these by the leading cultural peoples mightily influences the achievement of world\hyp{}wide peace and prosperity.
\vs p081 6:19 While very little progress has been made on Urantia toward developing an international language, much has been accomplished by the establishment of international commercial exchange. And all these international relations should be fostered, whether they involve language, trade, art, science, competitive play, or religion.
\vs p081 6:20 \ublistelem{7.}\bibnobreakspace \bibemph{Effectiveness of mechanical devices.} The progress of civilization is directly related to the development and possession of tools, machines, and channels of distribution. Improved tools, ingenious and efficient machines, determine the survival of contending groups in the arena of advancing civilization.
\vs p081 6:21 In the early days the only energy applied to land cultivation was man power. It was a long struggle to substitute oxen for men since this threw men out of employment. Latterly, machines have begun to displace men, and every such advance is directly contributory to the progress of society because it liberates man power for the accomplishment of more valuable tasks.
\vs p081 6:22 Science, guided by wisdom, may become man’s great social liberator. A mechanical age can prove disastrous only to a nation whose intellectual level is too low to discover those wise methods and sound techniques for successfully adjusting to the transition difficulties arising from the sudden loss of employment by large numbers consequent upon the too rapid invention of new types of laboursaving machinery.
\vs p081 6:23 \ublistelem{8.}\bibnobreakspace \bibemph{Character of torchbearers.} Social inheritance enables man to stand on the shoulders of all who have preceded him, and who have contributed aught to the sum of culture and knowledge. In this work of passing on the cultural torch to the next generation, the home will ever be the basic institution. The play and social life comes next, with the school last but equally indispensable in a complex and highly organized society.
\vs p081 6:24 Insects are born fully educated and equipped for life --- indeed, a very narrow and purely instinctive existence. The human baby is born without an education; therefore man possesses the power, by controlling the educational training of the younger generation, greatly to modify the evolutionary course of civilization.
\vs p081 6:25 The greatest XX century influences contributing to the furtherance of civilization and the advancement of culture are the marked increase in world travel and the unparalleled improvements in methods of communication. But the improvement in education has not kept pace with the expanding social structure; neither has the modern appreciation of ethics developed in correspondence with growth along more purely intellectual and scientific lines. And modern civilization is at a standstill in spiritual development and the safeguarding of the home institution.
\vs p081 6:26 \ublistelem{9.}\bibnobreakspace \bibemph{The racial ideals.} The ideals of one generation carve out the channels of destiny for immediate posterity. The \bibemph{quality} of the social torchbearers will determine whether civilization goes forward or backward. The homes, churches, and schools of one generation predetermine the character trend of the succeeding generation. The moral and spiritual momentum of a race or a nation largely determines the cultural velocity of that civilization.
\vs p081 6:27 Ideals elevate the source of the social stream. And no stream will rise any higher than its source no matter what technique of pressure or directional control may be employed. The driving power of even the most material aspects of a cultural civilization is resident in the least material of society’s achievements. Intelligence may control the mechanism of civilization, wisdom may direct it, but spiritual idealism is the energy which really uplifts and advances human culture from one level of attainment to another.
\vs p081 6:28 At first life was a struggle for existence; now, for a standard of living; next it will be for quality of thinking, the coming earthly goal of human existence.
\vs p081 6:29 \ublistelem{10.}\bibnobreakspace \bibemph{Co\hyp{}ordination of specialists.} Civilization has been enormously advanced by the early division of labour and by its later corollary of specialization. Civilization is now dependent on the effective co\hyp{}ordination of specialists. As society expands, some method of drawing together the various specialists must be found.
\vs p081 6:30 Social, artistic, technical, and industrial specialists will continue to multiply and increase in skill and dexterity. And this diversification of ability and dissimilarity of employment will eventually weaken and disintegrate human society if effective means of co\hyp{}ordination and co\hyp{}operation are not developed. But the intelligence which is capable of such inventiveness and such specialization should be wholly competent to devise adequate methods of control and adjustment for all problems resulting from the rapid growth of invention and the accelerated pace of cultural expansion.
\vs p081 6:31 \ublistelem{11.}\bibnobreakspace \bibemph{Place\hyp{}finding devices.} The next age of social development will be embodied in a better and more effective co\hyp{}operation and co\hyp{}ordination of ever\hyp{}increasing and expanding specialization. And as labour more and more diversifies, some technique for directing individuals to suitable employment must be devised. Machinery is not the only cause for unemployment among the civilized peoples of Urantia. Economic complexity and the steady increase of industrial and professional specialism add to the problems of labour placement.
\vs p081 6:32 It is not enough to train men for work; in a complex society there must also be provided efficient methods of place finding. Before training citizens in the highly specialized techniques of earning a living, they should be trained in one or more methods of commonplace labour, trades or callings which could be utilized when they were transiently unemployed in their specialized work. No civilization can survive the long\hyp{}time harbouring of large classes of unemployed. In time, even the best of citizens will become distorted and demoralized by accepting support from the public treasury. Even private charity becomes pernicious when long extended to able\hyp{}bodied citizens.
\vs p081 6:33 Such a highly specialized society will not take kindly to the ancient communal and feudal practices of olden peoples. True, many common services can be acceptably and profitably socialized, but highly trained and ultraspecialized human beings can best be managed by some technique of intelligent co\hyp{}operation. Modernized co\hyp{}ordination and fraternal regulation will be productive of longer\hyp{}lived co\hyp{}operation than will the older and more primitive methods of communism or dictatorial regulative institutions based on force.
\vs p081 6:34 \ublistelem{12.}\bibnobreakspace \bibemph{The willingness to co\hyp{}operate.} One of the great hindrances to the progress of human society is the conflict between the interests and welfare of the larger, more socialized human groups and of the smaller, contrary\hyp{}minded asocial associations of mankind, not to mention antisocially\hyp{}minded single individuals.
\vs p081 6:35 No national civilization long endures unless its educational methods and religious ideals inspire a high type of intelligent patriotism and national devotion. Without this sort of intelligent patriotism and cultural solidarity, all nations tend to disintegrate as a result of provincial jealousies and local self\hyp{}interests.
\vs p081 6:36 The maintenance of world\hyp{}wide civilization is dependent on human beings learning how to live together in peace and fraternity. Without effective co\hyp{}ordination, industrial civilization is jeopardized by the dangers of ultraspecialization: monotony, narrowness, and the tendency to breed distrust and jealousy.
\vs p081 6:37 \ublistelem{13.}\bibnobreakspace \bibemph{Effective and wise leadership.} In civilization much, very much, depends on an enthusiastic and effective load\hyp{}pulling spirit. 10 men are of little more value than 1 in lifting a great load unless they lift together --- all at the same moment. And such teamwork --- social co\hyp{}operation --- is dependent on leadership. The cultural civilizations of the past and the present have been based upon the intelligent co\hyp{}operation of the citizenry with wise and progressive leaders; and until man evolves to higher levels, civilization will continue to be dependent on wise and vigorous leadership.
\vs p081 6:38 High civilizations are born of the sagacious correlation of material wealth, intellectual greatness, moral worth, social cleverness, and cosmic insight.
\vs p081 6:39 \ublistelem{14.}\bibnobreakspace \bibemph{Social changes.} Society is not a divine institution; it is a phenomenon of progressive evolution; and advancing civilization is always delayed when its leaders are slow in making those changes in the social organization which are essential to keeping pace with the scientific developments of the age. For all that, things must not be despised just because they are old, neither should an idea be unconditionally embraced just because it is novel and new.
\vs p081 6:40 Man should be unafraid to experiment with the mechanisms of society. But always should these adventures in cultural adjustment be controlled by those who are fully conversant with the history of social evolution; and always should these innovators be counselled by the wisdom of those who have had practical experience in the domains of contemplated social or economic experiment. \bibemph{No great social or economic change should be attempted suddenly.} Time is essential to all types of human adjustment --- physical, social, or economic. Only moral and spiritual adjustments can be made on the spur of the moment, and even these require the passing of time for the full outworking of their material and social repercussions. The ideals of the race are the chief support and assurance during the critical times when civilization is in transit from one level to another.
\vs p081 6:41 \ublistelem{15.}\bibnobreakspace \bibemph{The prevention of transitional breakdown.} Society is the offspring of age upon age of trial and error; it is what survived the selective adjustments and readjustments in the successive stages of mankind’s agelong rise from animal to human levels of planetary status. The great danger to any civilization --- at any one moment --- is the threat of breakdown during the time of transition from the established methods of the past to those new and better, but untried, procedures of the future.
\vs p081 6:42 Leadership is vital to progress. Wisdom, insight, and foresight are indispensable to the endurance of nations. Civilization is never really jeopardized until able leadership begins to vanish. And the quantity of such wise leadership has never exceeded 1\% of the population.
\vs p081 6:43 And it was by these rungs on the evolutionary ladder that civilization climbed to that place where those mighty influences could be initiated which have culminated in the rapidly expanding culture of the XX century. And only by adherence to these essentials can man hope to maintain his present\hyp{}day civilizations while providing for their continued development and certain survival.
\vs p081 6:44 \pc This is the gist of the long, long struggle of the peoples of earth to establish civilization since the age of Adam. Present\hyp{}day culture is the net result of this strenuous evolution. Before the discovery of printing, progress was relatively slow since one generation could not so rapidly benefit from the achievements of its predecessors. But now human society is plunging forward under the force of the accumulated momentum of all the ages through which civilization has struggled.
\vsetoff
\vs p081 6:45 [Sponsored by an Archangel of Nebadon.]
\quizlink
\tunemarkuptwo{nobiblio}{}{%
\begin{thebibliography}{100}
\bibitem{Peake1}
Harold Peake \&\ Herbert John Fleure.
{``The Corridors of Time III: Peasants \&\ Potters.''}
{\em Oxford: Clarendon Press}, 1927.
\end{thebibliography}
}
