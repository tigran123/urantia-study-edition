\upaper{10}{The Paradise Trinity}
\author{Universal Censor}
\vs p010 0:1 The Paradise Trinity of eternal Deities facilitates the Father’s escape from personality absolutism. The Trinity perfectly associates the limitless expression of God’s infinite personal will with the absoluteness of Deity. The Eternal Son and the various Sons of divine origin, together with the Conjoint Actor and his universe children, effectively provide for the Father’s liberation from the limitations otherwise inherent in primacy, perfection, changelessness, eternity, universality, absoluteness, and infinity.
\vs p010 0:2 The Paradise Trinity effectively provides for the full expression and perfect revelation of the eternal nature of Deity. The Stationary Sons of the Trinity likewise afford a full and perfect revelation of divine justice. The Trinity is Deity unity, and this unity rests eternally upon the absolute foundations of the divine oneness of the three original and co\hyp{}ordinate and coexistent personalities, God the Father, God the Son, and God the Spirit.
\vs p010 0:3 \pc From the present situation on the circle of eternity, looking backward into the endless past, we can discover only one inescapable inevitability in universe affairs, and that is the Paradise Trinity. I deem the Trinity to have been inevitable. As I view the past, present, and future of time, I consider nothing else in all the universe of universes to have been inevitable. The present master universe, viewed in retrospect or in prospect, is unthinkable without the Trinity. Given the Paradise Trinity, we can postulate alternate or even multiple ways of doing all things, but without the Trinity of Father, Son, and Spirit we are unable to conceive how the Infinite could achieve threefold and co\hyp{}ordinate personalization in the face of the absolute oneness of Deity. No other concept of creation measures up to the Trinity standards of the completeness of the absoluteness inherent in Deity unity coupled with the repleteness of volitional liberation inherent in the threefold personalization of Deity.
\usection{1.\bibnobreakspace Self\hyp{}Distribution of the First Source and Centre}
\vs p010 1:1 It would seem that the Father, back in eternity, inaugurated a policy of profound self\hyp{}distribution. There is inherent in the selfless, loving, and lovable nature of the Universal Father something which causes him to reserve to himself the exercise of only those powers and that authority which he apparently finds it impossible to delegate or to bestow.
\vs p010 1:2 The Universal Father all along has divested himself of every part of himself that was bestowable on any other Creator or creature. He has delegated to his divine Sons and their associated intelligences every power and all authority that could be delegated. He has actually transferred to his Sovereign Sons, in their respective universes, every prerogative of administrative authority that was transferable. In the affairs of a local universe, he has made each Sovereign Creator Son just as perfect, competent, and authoritative as is the Eternal Son in the original and central universe. He has given away, actually bestowed, with the dignity and sanctity of personality possession, all of himself and all of his attributes, everything he possibly could divest himself of, in every way, in every age, in every place, and to every person, and in every universe except that of his central indwelling.
\vs p010 1:3 \pc Divine personality is not self\hyp{}centred; self\hyp{}distribution and sharing of personality characterize divine freewill selfhood. Creatures crave association with other personal creatures; Creators are moved to share divinity with their universe children; the personality of the Infinite is disclosed as the Universal Father, who shares reality of being and equality of self with two co\hyp{}ordinate personalities, the Eternal Son and the Conjoint Actor.
\vs p010 1:4 \pc For knowledge concerning the Father’s personality and divine attributes we will always be dependent on the revelations of the Eternal Son, for when the conjoint act of creation was effected, when the Third Person of Deity sprang into personality existence and executed the combined concepts of his divine parents, the Father ceased to exist as the unqualified personality. With the coming into being of the Conjoint Actor and the materialization of the central core of creation, certain eternal changes took place. God gave himself as an absolute personality to his Eternal Son. Thus does the Father bestow the “personality of infinity” upon his only\hyp{}begotten Son, while they both bestow the “conjoint personality” of their eternal union upon the Infinite Spirit.
\vs p010 1:5 For these and other reasons beyond the concept of the finite mind, it is exceedingly difficult for the human creature to comprehend God’s infinite father\hyp{}personality except as it is universally revealed in the Eternal Son and, with the Son, is universally active in the Infinite Spirit.
\vs p010 1:6 Since the Paradise Sons of God visit the evolutionary worlds and sometimes even there dwell in the likeness of mortal flesh, and since these bestowals make it possible for mortal man actually to know something of the nature and character of divine personality, therefore must the creatures of the planetary spheres look to the bestowals of these Paradise Sons for reliable and trustworthy information regarding the Father, the Son, and the Spirit.
\usection{2.\bibnobreakspace Deity Personalization}
\vs p010 2:1 By the technique of trinitization the Father divests himself of that unqualified spirit personality which is the Son, but in so doing he constitutes himself the Father of this very Son and thereby possesses himself of unlimited capacity to become the divine Father of all subsequently created, eventuated, or other personalized types of intelligent will creatures. As the \bibemph{absolute and unqualified personality} the Father can function only as and with the Son, but as a \bibemph{personal Father} he continues to bestow personality upon the diverse hosts of the differing levels of intelligent will creatures, and he forever maintains personal relations of loving association with this vast family of universe children.
\vs p010 2:2 After the Father has bestowed upon the personality of his Son the fullness of himself, and when this act of self\hyp{}bestowal is complete and perfect, of the infinite power and nature which are thus existent in the Father\hyp{}Son union, the eternal partners conjointly bestow those qualities and attributes which constitute still another being like themselves; and this conjoint personality, the Infinite Spirit, completes the existential personalization of Deity.
\vs p010 2:3 The Son is indispensable to the fatherhood of God. The Spirit is indispensable to the fraternity of the Second and Third Persons. Three persons are a minimum social group, but this is least of all the many reasons for believing in the inevitability of the Conjoint Actor.
\vs p010 2:4 \pc The First Source and Centre is the infinite \bibemph{father\hyp{}personality,} the unlimited source personality. The Eternal Son is the unqualified \bibemph{personality\hyp{}absolute,} that divine being who stands throughout all time and eternity as the perfect revelation of the personal nature of God. The Infinite Spirit is the \bibemph{conjoint personality,} the unique personal consequence of the everlasting Father\hyp{}Son union.
\vs p010 2:5 \pc The personality of the First Source and Centre is the personality of infinity minus the absolute personality of the Eternal Son. The personality of the Third Source and Centre is the superadditive consequence of the union of the liberated Father\hyp{}personality and the absolute Son\hyp{}personality.
\vs p010 2:6 \pc The Universal Father, the Eternal Son, and the Infinite Spirit are unique persons; none is a duplicate; each is original; all are united.
\vs p010 2:7 \pc The Eternal Son alone experiences the fullness of divine personality relationship, consciousness of both sonship with the Father and paternity to the Spirit and of divine equality with both Father\hyp{}ancestor and Spirit\hyp{}associate. The Father knows the experience of having a Son who is his equal, but the Father knows no ancestral antecedents. The Eternal Son has the experience of sonship, recognition of personality ancestry, and at the same time the Son is conscious of being joint parent to the Infinite Spirit. The Infinite Spirit is conscious of twofold personality ancestry but is not parental to a co\hyp{}ordinate Deity personality. With the Spirit the existential cycle of Deity personalization attains completion; the primary personalities of the Third Source and Centre are experiential and are seven in number.
\vs p010 2:8 I am of origin in the Paradise Trinity. I know the Trinity as unified Deity; I also know that the Father, Son, and Spirit exist and act in their definite personal capacities. I positively know that they not only act personally and collectively, but that they also co\hyp{}ordinate their performances in various groupings, so that in the end they function in seven different singular and plural capacities. And since these seven associations exhaust the possibilities for such divinity combination, it is inevitable that the realities of the universe shall appear in seven variations of values, meanings, and personality.
\usection{3.\bibnobreakspace The Three Persons of Deity}
\vs p010 3:1 Notwithstanding there is only one Deity, there are three positive and divine personalizations of Deity. Regarding the endowment of man with the divine Adjusters, the Father said: “Let us make mortal man in our own image.” Repeatedly throughout the Urantian writings there occurs this reference to the acts and doings of plural Deity, clearly showing recognition of the existence and working of the three Sources and Centres.
\vs p010 3:2 \pc We are taught that the Son and the Spirit sustain the same and equal relations to the Father in the Trinity association. In eternity and as Deities they undoubtedly do, but in time and as personalities they certainly disclose relationships of a very diverse nature. Looking from Paradise out on the universes, these relationships do seem to be very similar, but when viewed from the domains of space, they appear to be quite different.
\vs p010 3:3 The divine Sons are indeed the “Word of God,” but the children of the Spirit are truly the “Act of God.” God speaks through the Son and, with the Son, acts through the Infinite Spirit, while in all universe activities the Son and the Spirit are exquisitely fraternal, working as two equal brothers with admiration and love for an honoured and divinely respected common Father.
\vs p010 3:4 The Father, Son, and Spirit are certainly equal in nature, co\hyp{}ordinate in being, but there are unmistakable differences in their universe performances, and when acting alone, each person of Deity is apparently limited in absoluteness.
\vs p010 3:5 \pc The Universal Father, prior to his self\hyp{}willed divestment of the personality, powers, and attributes which constitute the Son and the Spirit, seems to have been (philosophically considered) an unqualified, absolute, and infinite Deity. But such a theoretical First Source and Centre without a Son could not in any sense of the word be considered the \bibemph{Universal Father;} fatherhood is not real without sonship. Furthermore, the Father, to have been absolute in a total sense, must have existed at some eternally distant moment alone. But he never had such a solitary existence; the Son and the Spirit are both coeternal with the Father. The First Source and Centre has always been, and will forever be, the eternal Father of the Original Son and, with the Son, the eternal progenitor of the Infinite Spirit.
\vs p010 3:6 We observe that the Father has divested himself of all direct manifestations of absoluteness except absolute fatherhood and absolute volition. We do not know whether volition is an inalienable attribute of the Father; we can only observe that he did \bibemph{not} divest himself of volition. Such infinity of will must have been eternally inherent in the First Source and Centre.
\vs p010 3:7 In bestowing absoluteness of personality upon the Eternal Son, the Universal Father escapes from the fetters of personality absolutism, but in so doing he takes a step which makes it forever impossible for him to act alone as the personality\hyp{}absolute. And with the final personalization of coexistent Deity --- the Conjoint Actor --- there ensues the critical trinitarian interdependence of the three divine personalities with regard to the totality of Deity function in absolute.
\vs p010 3:8 God is the Father\hyp{}Absolute of all personalities in the universe of universes. The Father is personally absolute in liberty of action, but in the universes of time and space, made, in the making, and yet to be made, the Father is not discernibly absolute as total Deity except in the Paradise Trinity.
\vs p010 3:9 \pc The First Source and Centre functions outside of Havona in the phenomenal universes as follows:
\vs p010 3:10 \ublistelem{1.}\bibnobreakspace As creator, through the Creator Sons, his grandsons.
\vs p010 3:11 \ublistelem{2.}\bibnobreakspace As controller, through the gravity centre of Paradise.
\vs p010 3:12 \ublistelem{3.}\bibnobreakspace As spirit, through the Eternal Son.
\vs p010 3:13 \ublistelem{4.}\bibnobreakspace As mind, through the Conjoint Creator.
\vs p010 3:14 \ublistelem{5.}\bibnobreakspace As a Father, he maintains parental contact with all creatures through his personality circuit.
\vs p010 3:15 \ublistelem{6.}\bibnobreakspace As a person, he acts \bibemph{directly} throughout creation by his exclusive fragments --- in mortal man by the Thought Adjusters.
\vs p010 3:16 \ublistelem{7.}\bibnobreakspace As total Deity, he functions only in the Paradise Trinity.
\vs p010 3:17 \pc All these relinquishments and delegations of jurisdiction by the Universal Father are wholly voluntary and self\hyp{}imposed. The all\hyp{}powerful Father purposefully assumes these limitations of universe authority.
\vs p010 3:18 \pc The Eternal Son seems to function as one with the Father in all spiritual respects except in the bestowals of the God fragments and in other prepersonal activities. Neither is the Son closely identified with the intellectual activities of material creatures nor with the energy activities of the material universes. As absolute the Son functions as a person and only in the domain of the spiritual universe.
\vs p010 3:19 \pc The Infinite Spirit is amazingly universal and unbelievably versatile in all his operations. He performs in the spheres of mind, matter, and spirit. The Conjoint Actor represents the Father\hyp{}Son association, but he also functions as himself. He is not directly concerned with physical gravity, with spiritual gravity, or with the personality circuit, but he more or less participates in all other universe activities. While apparently dependent on three existential and absolute gravity controls, the Infinite Spirit appears to exercise three supercontrols. This threefold endowment is employed in many ways to transcend and seemingly to neutralize even the manifestations of primary forces and energies, right up to the superultimate borders of absoluteness. In certain situations these supercontrols absolutely transcend even the primal manifestations of cosmic reality.
\usection{4.\bibnobreakspace The Trinity Union of Deity}
\vs p010 4:1 Of all absolute associations, the Paradise Trinity (the first triunity) is unique as an exclusive association of personal Deity. God functions as God only in relation to God and to those who can know God, but as absolute Deity only in the Paradise Trinity and in relation to universe totality.
\vs p010 4:2 \pc Eternal Deity is perfectly unified; nevertheless there are three perfectly individualized persons of Deity. The Paradise Trinity makes possible the simultaneous expression of all the diversity of the character traits and infinite powers of the First Source and Centre and his eternal co\hyp{}ordinates and of all the divine unity of the universe functions of undivided Deity.
\vs p010 4:3 The Trinity is an association of infinite persons functioning in a nonpersonal capacity but not in contravention of personality. The illustration is crude, but a father, son, and grandson could form a corporate entity which would be nonpersonal but nonetheless subject to their personal wills.
\vs p010 4:4 The Paradise Trinity is \bibemph{real.} It exists as the Deity union of Father, Son, and Spirit; yet the Father, the Son, or the Spirit, or any two of them, can function in relation to this selfsame Paradise Trinity. The Father, Son, and Spirit can collaborate in a non\hyp{}Trinity manner, but not as three Deities. As persons they can collaborate as they choose, but that is not the Trinity.
\vs p010 4:5 \pc Ever remember that what the Infinite Spirit does is the function of the Conjoint Actor. Both the Father and the Son are functioning in and through and as him. But it would be futile to attempt to elucidate the Trinity mystery: three as one and in one, and one as two and acting for two.
\vs p010 4:6 \pc The Trinity is so related to total universe affairs that it must be reckoned with in our attempts to explain the totality of any isolated cosmic event or personality relationship. The Trinity functions on all levels of the cosmos, and mortal man is limited to the finite level; therefore must man be content with a finite concept of the Trinity as the Trinity.
\vs p010 4:7 As a mortal in the flesh you should view the Trinity in accordance with your individual enlightenment and in harmony with the reactions of your mind and soul. You can know very little of the absoluteness of the Trinity, but as you ascend Paradiseward, you will many times experience astonishment at successive revelations and unexpected discoveries of Trinity supremacy and ultimacy, if not of absoluteness.
\usection{5.\bibnobreakspace Functions of the Trinity}
\vs p010 5:1 The personal Deities have attributes, but it is hardly consistent to speak of the Trinity as having attributes. This association of divine beings may more properly be regarded as having \bibemph{functions,} such as justice administration, totality attitudes, co\hyp{}ordinate action, and cosmic overcontrol. These functions are actively supreme, ultimate, and (within the limits of Deity) absolute as far as all living realities of personality value are concerned.
\vs p010 5:2 The functions of the Paradise Trinity are not simply the sum of the Father’s apparent endowment of divinity plus those specialized attributes that are unique in the personal existence of the Son and the Spirit. The Trinity association of the three Paradise Deities results in the evolution, eventuation, and deitization of new meanings, values, powers, and capacities for universal revelation, action, and administration. Living associations, human families, social groups, or the Paradise Trinity are not augmented by mere arithmetical summation. The group potential is always far in excess of the simple sum of the attributes of the component individuals.
\vs p010 5:3 \pc The Trinity maintains a unique attitude as the Trinity towards the entire universe of the past, present, and future. And the functions of the Trinity can best be considered in relation to the universe attitudes of the Trinity. Such attitudes are simultaneous and may be multiple concerning any isolated situation or event:
\vs p010 5:4 \ublistelem{1.}\bibnobreakspace \bibemph{Attitude toward the Finite.} The maximum self\hyp{}limitation of the Trinity is its attitude toward the finite. The Trinity is not a person, nor is the Supreme Being an exclusive personalization of the Trinity, but the Supreme is the nearest approach to a power\hyp{}personality focalization of the Trinity which can be comprehended by finite creatures. Hence the Trinity in relation to the finite is sometimes spoken of as the Trinity of Supremacy.
\vs p010 5:5 \ublistelem{2.}\bibnobreakspace \bibemph{Attitude toward the Absonite.} The Paradise Trinity has regard for those levels of existence which are more than finite but less than absolute, and this relationship is sometimes denominated the Trinity of Ultimacy. Neither the Ultimate nor the Supreme are wholly representative of the Paradise Trinity, but in a qualified sense and to their respective levels, each seems to represent the Trinity during the prepersonal eras of experiential\hyp{}power development.
\vs p010 5:6 \ublistelem{3.}\bibnobreakspace \bibemph{The Absolute Attitude} of the Paradise Trinity is in relation to absolute existences and culminates in the action of total Deity.
\vs p010 5:7 \pc The Trinity Infinite involves the co\hyp{}ordinate action of all triunity relationships of the First Source and Centre --- undeified as well as deified --- and hence is very difficult for personalities to grasp. In the contemplation of the Trinity as infinite, do not ignore the seven triunities\fnst{\textbf{seven triunities}, defined at \bibref[104:4]{p104 4:1}.}; thereby certain difficulties of understanding may be avoided, and certain paradoxes may be partially resolved.
\vs p010 5:8 \pc But I do not command language which would enable me to convey to the limited human mind the full truth and the eternal significance of the Paradise Trinity and the nature of the never\hyp{}ending interassociation of the three beings of infinite perfection.
\usection{6.\bibnobreakspace The Stationary Sons of the Trinity}
\vs p010 6:1 All law takes origin in the First Source and Centre; \bibemph{he is law.} The administration of spiritual law inheres in the Second Source and Centre. The revelation of law, the promulgation and interpretation of the divine statutes, is the function of the Third Source and Centre. The application of law, justice, falls within the province of the Paradise Trinity and is carried out by certain Sons of the Trinity.
\vs p010 6:2 \pc \bibemph{Justice} is inherent in the universal sovereignty of the Paradise Trinity, but goodness, mercy, and truth are the universe ministry of the divine personalities, whose Deity union constitutes the Trinity. Justice is not the attitude of the Father, the Son, or the Spirit. Justice is the Trinity attitude of these personalities of love, mercy, and ministry. No one of the Paradise Deities fosters the administration of justice. Justice is never a personal attitude; it is always a plural function.
\vs p010 6:3 \pc \bibemph{Evidence,} the basis of fairness (justice in harmony with mercy), is supplied by the personalities of the Third Source and Centre, the conjoint representative of the Father and the Son to all realms and to the minds of the intelligent beings of all creation.
\vs p010 6:4 \pc \bibemph{Judgment,} the final application of justice in accordance with the evidence submitted by the personalities of the Infinite Spirit, is the work of the Stationary Sons of the Trinity, beings partaking of the Trinity nature of the united Father, Son, and Spirit.
\vs p010 6:5 \pc This group of Trinity Sons embraces the following personalities:
\vs p010 6:6 \ublistelem{1.}\bibnobreakspace Trinitized Secrets of Supremacy.
\vs p010 6:7 \ublistelem{2.}\bibnobreakspace Eternals of Days.
\vs p010 6:8 \ublistelem{3.}\bibnobreakspace Ancients of Days.
\vs p010 6:9 \ublistelem{4.}\bibnobreakspace Perfections of Days.
\vs p010 6:10 \ublistelem{5.}\bibnobreakspace Recents of Days.
\vs p010 6:11 \ublistelem{6.}\bibnobreakspace Unions of Days.
\vs p010 6:12 \ublistelem{7.}\bibnobreakspace Faithfuls of Days.
\vs p010 6:13 \ublistelem{8.}\bibnobreakspace Perfectors of Wisdom.
\vs p010 6:14 \ublistelem{9.}\bibnobreakspace Divine Counsellors.
\vs p010 6:15 \ublistelem{10.}\bibnobreakspace Universal Censors.
\vs p010 6:16 \pc We are the children of the three Paradise Deities functioning as the Trinity, for I chance to belong to the tenth order of this group, the Universal Censors. These orders are not representative of the attitude of the Trinity in a universal sense; they represent this collective attitude of Deity only in the domains of executive judgment --- justice. They were specifically designed by the Trinity for the precise work to which they are assigned, and they represent the Trinity only in those functions for which they were personalized.
\vs p010 6:17 The Ancients of Days and their Trinity\hyp{}origin associates mete out the just judgment of supreme fairness to the seven superuniverses. In the central universe such functions exist in theory only; there fairness is self\hyp{}evident in perfection, and Havona perfection precludes all possibility of disharmony.
\vs p010 6:18 Justice is the collective thought of righteousness; mercy is its personal expression. Mercy is the attitude of love; precision characterizes the operation of law; divine judgment is the soul of fairness, ever conforming to the justice of the Trinity, ever fulfilling the divine love of God. When fully perceived and completely understood, the righteous justice of the Trinity and the merciful love of the Universal Father are coincident. But man has no such full understanding of divine justice. Thus in the Trinity, as man would view it, the personalities of Father, Son, and Spirit are adjusted to co\hyp{}ordinate ministry of love and law in the experiential universes of time.
\usection{7.\bibnobreakspace The Overcontrol of Supremacy}
\vs p010 7:1 The First, Second, and Third Persons of Deity are equal to each other, and they are one. “The Lord our God is one God.” There is perfection of purpose and oneness of execution in the divine Trinity of eternal Deities. The Father, the Son, and the Conjoint Actor are truly and divinely one. Of a truth it is written: “I am the first, and I am the last, and beside me there is no God.”
\vs p010 7:2 \pc As things appear to the mortal on the finite level, the Paradise Trinity, like the Supreme Being, is concerned only with the total --- total planet, total universe, total superuniverse, total grand universe. This totality attitude exists because the Trinity is the total of Deity and for many other reasons.
\vs p010 7:3 The Supreme Being is something less and something other than the Trinity functioning in the finite universes; but within certain limits and during the present era of incomplete power\hyp{}personalization, this evolutionary Deity does appear to reflect the attitude of the Trinity of Supremacy. The Father, Son, and Spirit do not personally function with the Supreme Being, but during the present universe age they collaborate with him as the Trinity. We understand that they sustain a similar relationship to the Ultimate. We often conjecture as to what will be the personal relationship between the Paradise Deities and God the Supreme when he has finally evolved, but we do not really know.
\vs p010 7:4 \pc We do not find the overcontrol of Supremacy to be wholly predictable. Furthermore, this unpredictability appears to be characterized by a certain developmental incompleteness, undoubtedly an earmark of the incompleteness of the Supreme and of the incompleteness of finite reaction to the Paradise Trinity.
\vs p010 7:5 The mortal mind can immediately think of a thousand and one things --- catastrophic physical events, appalling accidents, horrific disasters, painful illnesses, and world\hyp{}wide scourges --- and ask whether such visitations are correlated in the unknown maneuvering of this probable functioning of the Supreme Being. Frankly, we do not know; we are not really sure. But we do observe that, as time passes, all these difficult and more or less mysterious situations \bibemph{always} work out for the welfare and progress of the universes. It may be that the circumstances of existence and the inexplicable vicissitudes of living are all interwoven into a meaningful pattern of high value by the function of the Supreme and the overcontrol of the Trinity.
\vs p010 7:6 As a son of God you can discern the personal attitude of love in all the acts of God the Father. But you will not always be able to understand how many of the universe acts of the Paradise Trinity redound to the good of the individual mortal on the evolutionary worlds of space. In the progress of eternity the acts of the Trinity will be revealed as altogether meaningful and considerate, but they do not always so appear to the creatures of time.
\usection{8.\bibnobreakspace The Trinity Beyond the Finite}
\vs p010 8:1 Many truths and facts pertaining to the Paradise Trinity can only be even partially comprehended by recognizing a function that transcends the finite.
\vs p010 8:2 It would be inadvisable to discuss the functions of the Trinity of Ultimacy, but it may be disclosed that God the Ultimate is the Trinity manifestation comprehended by the Transcendentalers. We are inclined to the belief that the unification of the master universe is the eventuating act of the Ultimate and is probably reflective of certain, but not all, phases of the absonite overcontrol of the Paradise Trinity. The Ultimate is a qualified manifestation of the Trinity in relation to the absonite only in the sense that the Supreme thus partially represents the Trinity in relation to the finite.
\vs p010 8:3 \pc The Universal Father, the Eternal Son, and the Infinite Spirit are, in a certain sense, the constituent personalities of total Deity. Their union in the Paradise Trinity and the absolute function of the Trinity equivalate to the function of total Deity. And such completion of Deity transcends both the finite and the absonite.
\vs p010 8:4 While no single person of the Paradise Deities actually fills all Deity potential, collectively all three do. Three infinite persons seem to be the minimum number of beings required to activate the prepersonal and existential potential of total Deity --- the Deity Absolute.
\vs p010 8:5 We know the Universal Father, the Eternal Son, and the Infinite Spirit as \bibemph{persons,} but I do not personally know the Deity Absolute. I love and worship God the Father; I respect and honour the Deity Absolute.
\vs p010 8:6 \pc I once sojourned in a universe where a certain group of beings taught that the finaliters, in eternity, were eventually to become the children of the Deity Absolute. But I am unwilling to accept this solution of the mystery which enshrouds the future of the finaliters.
\vs p010 8:7 The Corps of the Finality embrace, among others, those mortals of time and space who have attained perfection in all that pertains to the will of God. As creatures and within the limits of creature capacity they fully and truly know God. Having thus found God as the Father of all creatures, these finaliters must sometime begin the quest for the superfinite Father. But this quest involves a grasp of the absonite nature of the ultimate attributes and character of the Paradise Father. Eternity will disclose whether such an attainment is possible, but we are convinced, even if the finaliters do grasp this ultimate of divinity, they will probably be unable to attain the superultimate levels of absolute Deity.
\vs p010 8:8 It may be possible that the finaliters will partially attain the Deity Absolute, but even if they should, still in the eternity of eternities the problem of the Universal Absolute will continue to intrigue, mystify, baffle, and challenge the ascending and progressing finaliters, for we perceive that the unfathomability of the cosmic relationships of the Universal Absolute will tend to grow in proportions as the material universes and their spiritual administration continue to expand.
\vs p010 8:9 \pc Only infinity can disclose the Father\hyp{}Infinite.
\vsetoff
\vs p010 8:10 [Sponsored by a Universal Censor acting by authority from the Ancients of Days resident on Uversa.]
