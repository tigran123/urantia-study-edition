\upaper{135}{John the Baptist}
\uminitoc{John Becomes a Nazarite}
\uminitoc{The Death of Zacharias}
\uminitoc{The Life of a Shepherd}
\uminitoc{The Death of Elizabeth}
\uminitoc{The Kingdom of God}
\uminitoc{John Begins to Preach}
\uminitoc{John Journeys North}
\uminitoc{Meeting of Jesus and John}
\uminitoc{Forty Days of Preaching}
\uminitoc{John Journeys South}
\uminitoc{John in Prison}
\uminitoc{Death of John the Baptist}
\author{Midwayer Commission}
\vs p135 0:1 John the Baptist was born March 25, 7\,B.C., in accordance with the promise that Gabriel made to Elizabeth in June of the previous year. For five months Elizabeth kept secret Gabriel’s visitation; and when she told her husband, Zacharias, he was greatly troubled and fully believed her narrative only after he had an unusual dream about six weeks before the birth of John. Excepting the visit of Gabriel to Elizabeth and the dream of Zacharias, there was nothing unusual or supernatural connected with the birth of John the Baptist.
\vs p135 0:2 On the eighth day John was circumcised according to the Jewish custom. He grew up as an ordinary child, day by day and year by year, in the small village known in those days as the City of Judah, about 6.4\,km west of Jerusalem.
\vs p135 0:3 The most eventful occurrence in John’s early childhood was the visit, in company with his parents, to Jesus and the Nazareth family. This visit occurred in the month of June, 1\,B.C., when he was a little over six years of age.
\vs p135 0:4 After their return from Nazareth John’s parents began the systematic education of the lad. There was no synagogue school in this little village; however, as he was a priest, Zacharias was fairly well educated, and Elizabeth was far better educated than the average Judean woman; she was also of the priesthood, being a descendant of the “daughters of Aaron.” Since John was an only child, they spent a great deal of time on his mental and spiritual training. Zacharias had only short periods of service at the temple in Jerusalem so that he devoted much of his time to teaching his son.
\vs p135 0:5 Zacharias and Elizabeth had a small farm on which they raised sheep. They hardly made a living on this land, but Zacharias received a regular allowance from the temple funds dedicated to the priesthood.
\usection{John Becomes a Nazarite}
\vs p135 1:1 John had no school from which to graduate at the age of 14, but his parents had selected this as the appropriate year for him to take the formal Nazarite vow. Accordingly, Zacharias and Elizabeth took their son to Engedi, down by the Dead Sea. This was the southern headquarters of the Nazarite brotherhood, and there the lad was duly and solemnly inducted into this order for life. After these ceremonies and the making of the vows to abstain from all intoxicating drinks, to let the hair grow, and to refrain from touching the dead, the family proceeded to Jerusalem, where, before the temple, John completed the making of the offerings which were required of those taking Nazarite vows.
\vs p135 1:2 John took the same life vows that had been administered to his illustrious predecessors, Samson and the prophet Samuel. A life Nazarite was looked upon as a sanctified and holy personality. The Jews regarded a Nazarite with almost the respect and veneration accorded the high priest, and this was not strange since Nazarites of lifelong consecration were the only persons, except high priests, who were ever permitted to enter the holy of holies in the temple.
\vs p135 1:3 \pc John returned home from Jerusalem to tend his father’s sheep and grew up to be a strong man with a noble character.
\vs p135 1:4 When 16 years old, John, as a result of reading about Elijah, became greatly impressed with the prophet of Mount Carmel and decided to adopt his style of dress. From that day on John always wore a hairy garment with a leather girdle. At 16 he was more than 1.8\,m tall and almost full grown. With his flowing hair and peculiar mode of dress he was indeed a picturesque youth. And his parents expected great things of this their only son, a child of promise and a Nazarite for life.
\usection{The Death of Zacharias}
\vs p135 2:1 After an illness of several months Zacharias died in July, A.D.\,12, when John was just past 18 years of age. This was a time of great embarrassment to John since the Nazarite vow forbade contact with the dead, even in one’s own family. Although John had endeavoured to comply with the restrictions of his vow regarding contamination by the dead, he doubted that he had been wholly obedient to the requirements of the Nazarite order; therefore, after his father’s burial he went to Jerusalem, where, in the Nazarite corner of the women’s court, he offered the sacrifices required for his cleansing.
\vs p135 2:2 \pc In September of this year Elizabeth and John made a journey to Nazareth to visit Mary and Jesus. John had just about made up his mind to launch out in his lifework, but he was admonished, not only by Jesus’ words but also by his example, to return home, take care of his mother, and await the “coming of the Father’s hour.” After bidding Jesus and Mary good\hyp{}bye at the end of this enjoyable visit, John did not again see Jesus until the event of his baptism in the Jordan.
\vs p135 2:3 John and Elizabeth returned to their home and began to lay plans for the future. Since John refused to accept the priest’s allowance due him from the temple funds, by the end of two years they had all but lost their home; so they decided to go south with the sheep herd. Accordingly, the summer that John was 20 years of age witnessed their removal to Hebron. In the so\hyp{}called “wilderness of Judea” John tended his sheep along a brook that was tributary to a larger stream which entered the Dead Sea at Engedi. The Engedi colony included not only Nazarites of lifelong and time\hyp{}period consecration but numerous other ascetic herdsmen who congregated in this region with their herds and fraternized with the Nazarite brotherhood. They supported themselves by sheep raising and from gifts which wealthy Jews made to the order.
\vs p135 2:4 As time passed, John returned less often to Hebron, while he made more frequent visits to Engedi. He was so entirely different from the majority of the Nazarites that he found it very difficult fully to fraternize with the brotherhood. But he was very fond of Abner, the acknowledged leader and head of the Engedi colony.
\usection{The Life of a Shepherd}
\vs p135 3:1 Along the valley of this little brook John built no less than a dozen stone shelters and night corrals, consisting of piled\hyp{}up stones, wherein he could watch over and safeguard his herds of sheep and goats. John’s life as a shepherd afforded him a great deal of time for thought. He talked much with Ezda, an orphan lad of Beth\hyp{}zur, whom he had in a way adopted, and who cared for the herds when he made trips to Hebron to see his mother and to sell sheep, as well as when he went down to Engedi for Sabbath services. John and the lad lived very simply, subsisting on mutton, goat’s milk, wild honey, and the edible locusts of that region. This, their regular diet, was supplemented by provisions brought from Hebron and Engedi from time to time.
\vs p135 3:2 \pc Elizabeth kept John posted about Palestinian and world affairs, and his conviction grew deeper and deeper that the time was fast approaching when the old order was to end; that he was to become the herald of the approach of a new age, “the kingdom of heaven.” This rugged shepherd was very partial to the writings of the Prophet Daniel. He read a thousand times Daniel’s description of the great image, which Zacharias had told him represented the history of the great kingdoms of the world, beginning with Babylon, then Persia, Greece, and finally Rome. John perceived that already was Rome composed of such polyglot peoples and races that it could never become a strongly cemented and firmly consolidated empire. He believed that Rome was even then divided, as Syria, Egypt, Palestine, and other provinces; and then he further read “in the days of these kings shall the God of heaven set up a kingdom which shall never be destroyed. And this kingdom shall not be left to other people but shall break in pieces and consume all these kingdoms, and it shall stand forever.” “And there was given him dominion and glory and a kingdom that all peoples, nations, and languages should serve him. His dominion is an everlasting dominion, which shall not pass away, and his kingdom never shall be destroyed.” “And the kingdom and dominion and the greatness of the kingdom under the whole heaven shall be given to the people of the saints of the Most High, whose kingdom is an everlasting kingdom, and all dominions shall serve and obey him.”
\vs p135 3:3 \pc John was never able completely to rise above the confusion produced by what he had heard from his parents concerning Jesus and by these passages which he read in the Scriptures. In Daniel he read: “I saw in the night visions, and, behold, one like the Son of Man came with the clouds of heaven, and there was given him dominion and glory and a kingdom.” But these words of the prophet did not harmonize with what his parents had taught him. Neither did his talk with Jesus, at the time of his visit when he was 18 years old, correspond with these statements of the Scriptures. Notwithstanding this confusion, throughout all of his perplexity his mother assured him that his distant cousin, Jesus of Nazareth, was the true Messiah, that he had come to sit on the throne of David, and that he (John) was to become his advance herald and chief support.
\vs p135 3:4 From all John heard of the vice and wickedness of Rome and the dissoluteness and moral barrenness of the empire, from what he knew of the evil doings of Herod Antipas and the governors of Judea, he was minded to believe that the end of the age was impending. It seemed to this rugged and noble child of nature that the world was ripe for the end of the age of man and the dawn of the new and divine age --- the kingdom of heaven. The feeling grew in John’s heart that he was to be the last of the old prophets and the first of the new. And he fairly vibrated with the mounting impulse to go forth and proclaim to all men: “Repent! Get right with God! Get ready for the end; prepare yourselves for the appearance of the new and eternal order of earth affairs, the kingdom of heaven.”
\usection{The Death of Elizabeth}
\vs p135 4:1 On August 17, A.D.\,22, when John was 28 years of age, his mother suddenly passed away. Elizabeth’s friends, knowing of the Nazarite restrictions regarding contact with the dead, even in one’s own family, made all arrangements for the burial of Elizabeth before sending for John. When he received word of the death of his mother, he directed Ezda to drive his herds to Engedi and started for Hebron.
\vs p135 4:2 On returning to Engedi from his mother’s funeral, he presented his flocks to the brotherhood and for a season detached himself from the outside world while he fasted and prayed. John knew only of the old methods of approach to divinity; he knew only of the records of such as Elijah, Samuel, and Daniel. Elijah was his ideal of a prophet. Elijah was the first of the teachers of Israel to be regarded as a prophet, and John truly believed that he was to be the last of this long and illustrious line of the messengers of heaven.
\vs p135 4:3 For 2½ years John lived at Engedi, and he persuaded most of the brotherhood that “the end of the age was at hand”; that “the kingdom of heaven was about to appear.” And all his early teaching was based upon the current Jewish idea and concept of the Messiah as the promised deliverer of the Jewish nation from the domination of their gentile rulers.
\vs p135 4:4 Throughout this period John read much in the sacred writings which he found at the Engedi home of the Nazarites. He was especially impressed by Isaiah and by Malachi, the last of the prophets up to that time. He read and reread the last five chapters of Isaiah, and he believed these prophecies. Then he would read in Malachi: “Behold, I will send you Elijah the prophet before the coming of the great and dreadful day of the Lord; and he shall turn the hearts of the fathers toward the children and the hearts of the children toward their fathers, lest I come and smite the earth with a curse.” And it was only this promise of Malachi that Elijah would return that deterred John from going forth to preach about the coming kingdom and to exhort his fellow Jews to flee from the wrath to come. John was ripe for the proclamation of the message of the coming kingdom, but this expectation of the coming of Elijah held him back for more than two years. He knew he was not Elijah. What did Malachi mean? Was the prophecy literal or figurative? How could he know the truth? He finally dared to think that, since the first of the prophets was called Elijah, so the last should be known, eventually, by the same name. Nevertheless, he had doubts, doubts sufficient to prevent his ever calling himself Elijah.
\vs p135 4:5 It was the influence of Elijah that caused John to adopt his methods of direct and blunt assault upon the sins and vices of his contemporaries. He sought to dress like Elijah, and he endeavoured to talk like Elijah; in every outward aspect he was like the olden prophet. He was just such a stalwart and picturesque child of nature, just such a fearless and daring preacher of righteousness. John was not illiterate, he did well know the Jewish sacred writings, but he was hardly cultured. He was a clear thinker, a powerful speaker, and a fiery denunciator. He was hardly an example to his age, but he was an eloquent rebuke.
\vs p135 4:6 At last he thought out the method of proclaiming the new age, the kingdom of God; he settled that he was to become the herald of the Messiah; he swept aside all doubts and departed from Engedi one day in March of A.D.\,25 to begin his short but brilliant career as a public preacher.
\usection{The Kingdom of God}
\vs p135 5:1 In order to understand John’s message, account should be taken of the status of the Jewish people at the time he appeared upon the stage of action. For almost one hundred years all Israel had been in a quandary; they were at a loss to explain their continuous subjugation to gentile overlords. Had not Moses taught that righteousness was always rewarded with prosperity and power? Were they not God’s chosen people? Why was the throne of David desolate and vacant? In the light of the Mosaic doctrines and the precepts of the prophets the Jews found it difficult to explain their long\hyp{}continued national desolation.
\vs p135 5:2 About one hundred years before the days of Jesus and John a new school of religious teachers arose in Palestine, the apocalyptists. These new teachers evolved a system of belief that accounted for the sufferings and humiliation of the Jews on the ground that they were paying the penalty for the nation’s sins. They fell back onto the well\hyp{}known reasons assigned to explain the Babylonian and other captivities of former times. But, so taught the apocalyptists, Israel should take heart; the days of their affliction were almost over; the discipline of God’s chosen people was about finished; God’s patience with the gentile foreigners was about exhausted. The end of Roman rule was synonymous with the end of the age and, in a certain sense, with the end of the world. These new teachers leaned heavily on the predictions of Daniel, and they consistently taught that creation was about to pass into its final stage; the kingdoms of this world were about to become the kingdom of God. To the Jewish mind of that day this was the meaning of that phrase --- the kingdom of heaven --- which runs throughout the teachings of both John and Jesus. To the Jews of Palestine the phrase “kingdom of heaven” had but one meaning: an absolutely righteous state in which God (the Messiah) would rule the nations of earth in perfection of power just as he ruled in heaven --- “Your will be done on earth as in heaven.”
\vs p135 5:3 In the days of John all Jews were expectantly asking, “How soon will the kingdom come?” There was a general feeling that the end of the rule of the gentile nations was drawing near. There was present throughout all Jewry a lively hope and a keen expectation that the consummation of the desire of the ages would occur during the lifetime of that generation.
\vs p135 5:4 While the Jews differed greatly in their estimates of the nature of the coming kingdom, they were alike in their belief that the event was impending, near at hand, even at the door. Many who read the Old Testament literally looked expectantly for a new king in Palestine, for a regenerated Jewish nation delivered from its enemies and presided over by the successor of King David, the Messiah who would quickly be acknowledged as the rightful and righteous ruler of all the world. Another, though smaller, group of devout Jews held a vastly different view of this kingdom of God. They taught that the coming kingdom was not of this world, that the world was approaching its certain end, and that “a new heaven and a new earth” were to usher in the establishment of the kingdom of God; that this kingdom was to be an everlasting dominion, that sin was to be ended, and that the citizens of the new kingdom were to become immortal in their enjoyment of this endless bliss.
\vs p135 5:5 All were agreed that some drastic purging or purifying discipline would of necessity precede the establishment of the new kingdom on earth. The literalists taught that a world\hyp{}wide war would ensue which would destroy all unbelievers, while the faithful would sweep on to universal and eternal victory. The spiritists taught that the kingdom would be ushered in by the great judgment of God which would relegate the unrighteous to their well\hyp{}deserved judgment of punishment and final destruction, at the same time elevating the believing saints of the chosen people to high seats of honour and authority with the Son of Man, who would rule over the redeemed nations in God’s name. And this latter group even believed that many devout gentiles might be admitted to the fellowship of the new kingdom.
\vs p135 5:6 Some of the Jews held to the opinion that God might possibly establish this new kingdom by direct and divine intervention, but the vast majority believed that he would interpose some representative intermediary, the Messiah. And that was the only possible meaning the term Messiah could have had in the minds of the Jews of the generation of John and Jesus. \bibemph{Messiah} could not possibly refer to one who merely taught God’s will or proclaimed the necessity for righteous living. To all such holy persons the Jews gave the title of \bibemph{prophet.} The Messiah was to be more than a prophet; the Messiah was to bring in the establishment of the new kingdom, the kingdom of God. No one who failed to do this could be the Messiah in the traditional Jewish sense.
\vs p135 5:7 Who would this Messiah be? Again the Jewish teachers differed. The older ones clung to the doctrine of the son of David. The newer taught that, since the new kingdom was a heavenly kingdom, the new ruler might also be a divine personality, one who had long sat at God’s right hand in heaven. And strange as it may appear, those who thus conceived of the ruler of the new kingdom looked upon him not as a human Messiah, not as a mere \bibemph{man,} but as “the Son of Man” --- a Son of God --- a heavenly Prince, long held in waiting thus to assume the rulership of the earth made new. Such was the religious background of the Jewish world when John went forth proclaiming: “Repent, for the kingdom of heaven is at hand!”
\vs p135 5:8 It becomes apparent, therefore, that John’s announcement of the coming kingdom had not less than half a dozen different meanings in the minds of those who listened to his impassioned preaching. But no matter what significance they attached to the phrases which John employed, each of these various groups of Jewish\hyp{}kingdom expectants was intrigued by the proclamations of this sincere, enthusiastic, rough\hyp{}and\hyp{}ready preacher of righteousness and repentance, who so solemnly exhorted his hearers to “flee from the wrath to come.”
\usection{John Begins to Preach}
\vs p135 6:1 Early in the month of March, A.D.\,25, John journeyed around the western coast of the Dead Sea and up the river Jordan to opposite Jericho, the ancient ford over which Joshua and the children of Israel passed when they first entered the promised land; and crossing over to the other side of the river, he established himself near the entrance to the ford and began to preach to the people who passed by on their way back and forth across the river. This was the most frequented of all the Jordan crossings.
\vs p135 6:2 It was apparent to all who heard John that he was more than a preacher. The great majority of those who listened to this strange man who had come up from the Judean wilderness went away believing that they had heard the voice of a prophet. No wonder the souls of these weary and expectant Jews were deeply stirred by such a phenomenon. Never in all Jewish history had the devout children of Abraham so longed for the “consolation of Israel” or more ardently anticipated “the restoration of the kingdom.” Never in all Jewish history could John’s message, “the kingdom of heaven is at hand,” have made such a deep and universal appeal as at the very time he so mysteriously appeared on the bank of this southern crossing of the Jordan.
\vs p135 6:3 He came from the herdsmen, like Amos. He was dressed like Elijah of old, and he thundered his admonitions and poured forth his warnings in the “spirit and power of Elijah.” It is not surprising that this strange preacher created a mighty stir throughout all Palestine as the travellers carried abroad the news of his preaching along the Jordan.
\vs p135 6:4 There was still another and a \bibemph{new} feature about the work of this Nazarite preacher: He baptized every one of his believers in the Jordan “for the remission of sins.” Although baptism was not a new ceremony among the Jews, they had never seen it employed as John now made use of it. It had long been the practice thus to baptize the gentile proselytes into the fellowship of the outer court of the temple, but never had the Jews themselves been asked to submit to the baptism of repentance. Only 15 months intervened between the time John began to preach and baptize and his arrest and imprisonment at the instigation of Herod Antipas, but in this short time he baptized considerably over 100,000 penitents.
\vs p135 6:5 John preached four months at Bethany ford before starting north up the Jordan. Tens of thousands of listeners, some curious but many earnest and serious, came to hear him from all parts of Judea, Perea, and Samaria. Even a few came from Galilee.
\vs p135 6:6 In May of this year, while he still lingered at Bethany ford, the priests and Levites sent a delegation out to inquire of John whether he claimed to be the Messiah, and by whose authority he preached. John answered these questioners by saying: “Go tell your masters that you have heard ‘the voice of one crying in the wilderness,’ as spoken by the prophet, saying, ‘make ready the way of the Lord, make straight a highway for our God. Every valley shall be filled, and every mountain and hill shall be brought low; the uneven ground shall become a plain, while the rough places shall become a smooth valley; and all flesh shall see the salvation of God.’”
\vs p135 6:7 John was a heroic but tactless preacher. One day when he was preaching and baptizing on the west bank of the Jordan, a group of Pharisees and a number of Sadducees came forward and presented themselves for baptism. Before leading them down into the water, John, addressing them as a group said: “Who warned you to flee, as vipers before the fire, from the wrath to come? I will baptize you, but I warn you to bring forth fruit worthy of sincere repentance if you would receive the remission of your sins. Tell me not that Abraham is your father. I declare that God is able of these twelve stones here before you to raise up worthy children for Abraham. And even now is the ax laid to the very roots of the trees. Every tree that brings not forth good fruit is destined to be cut down and cast into the fire.” (The twelve stones to which he referred were the reputed memorial stones set up by Joshua to commemorate the crossing of the “twelve tribes” at this very point when they first entered the promised land.)
\vs p135 6:8 John conducted classes for his disciples, in the course of which he instructed them in the details of their new life and endeavoured to answer their many questions. He counselled the teachers to instruct in the spirit as well as the letter of the law. He instructed the rich to feed the poor; to the tax gatherers he said: “Extort no more than that which is assigned you.” To the soldiers he said: “Do no violence and exact nothing wrongfully --- be content with your wages.” While he counselled all: “Make ready for the end of the age --- the kingdom of heaven is at hand.”
\usection{John Journeys North}
\vs p135 7:1 John still had confused ideas about the coming kingdom and its king. The longer he preached the more confused he became, but never did this intellectual uncertainty concerning the nature of the coming kingdom in the least lessen his conviction of the certainty of the kingdom’s immediate appearance. In mind John might be confused, but in spirit never. He was in no doubt about the coming kingdom, but he was far from certain as to whether or not Jesus was to be the ruler of that kingdom. As long as John held to the idea of the restoration of the throne of David, the teachings of his parents that Jesus, born in the City of David, was to be the long\hyp{}expected deliverer, seemed consistent; but at those times when he leaned more toward the doctrine of a spiritual kingdom and the end of the temporal age on earth, he was sorely in doubt as to the part Jesus would play in such events. Sometimes he questioned everything, but not for long. He really wished he might talk it all over with his cousin, but that was contrary to their expressed agreement.
\vs p135 7:2 \pc As John journeyed north, he thought much about Jesus. He paused at more than a dozen places as he travelled up the Jordan. It was at Adam that he first made reference to “another one who is to come after me” in answer to the direct question which his disciples asked him, “Are you the Messiah?” And he went on to say: “There will come after me one who is greater than I, whose sandal straps I am not worthy to stoop down and unloose. I baptize you with water, but he will baptize you with the Holy Spirit. And his shovel is in his hand thoroughly to cleanse his threshing floor; he will gather the wheat into his garner, but the chaff will he burn up with the judgment fire.”
\vs p135 7:3 In response to the questions of his disciples John continued to expand his teachings, from day to day adding more that was helpful and comforting compared with his early and cryptic message: “Repent and be baptized.” By this time throngs were arriving from Galilee and the Decapolis. Scores of earnest believers lingered with their adored teacher day after day.
\usection{Meeting of Jesus and John}
\vs p135 8:1 By December of A.D.\,25, when John reached the neighbourhood of Pella in his journey up the Jordan, his fame had extended throughout all Palestine, and his work had become the chief topic of conversation in all the towns about the lake of Galilee. Jesus had spoken favourably of John’s message, and this had caused many from Capernaum to join John’s cult of repentance and baptism. James and John the fishermen sons of Zebedee had gone down in December, soon after John took up his preaching position near Pella, and had offered themselves for baptism. They went to see John once a week and brought back to Jesus fresh, firsthand reports of the evangelist’s work.
\vs p135 8:2 Jesus’ brothers James and Jude had talked about going down to John for baptism; and now that Jude had come over to Capernaum for the Sabbath services, both he and James, after listening to Jesus’ discourse in the synagogue, decided to take counsel with him concerning their plans. This was on Saturday night, January 12, A.D.\,26. Jesus requested that they postpone the discussion until the following day, when he would give them his answer. He slept very little that night, being in close communion with the Father in heaven. He had arranged to have noontime lunch with his brothers and to advise them concerning baptism by John. That Sunday morning Jesus was working as usual in the boatshop. James and Jude had arrived with the lunch and were waiting in the lumber room for him, as it was not yet time for the midday recess, and they knew that Jesus was very regular about such matters.\tunemarkup{private}{\begin{figure}[H]\centering\includegraphics[width=\columnwidth]{images/My-Hour-Has-Come.jpg}\caption{My hour has come by Russ Docken}\end{figure}}
\vs p135 8:3 Just before the noon rest, Jesus laid down his tools, removed his work apron, and merely announced to the three workmen in the room with him, \textcolour{ubdarkred}{“My hour has come.”} He went out to his brothers James and Jude, repeating, \textcolour{ubdarkred}{“My hour has come --- let us go to John.”} And they started immediately for Pella, eating their lunch as they journeyed. This was on Sunday, January 13. They tarried for the night in the Jordan valley and arrived on the scene of John’s baptizing about noon of the next day.
\vs p135 8:4 \pc John had just begun baptizing the candidates for the day. Scores of repentants were standing in line awaiting their turn when Jesus and his two brothers took up their positions in this line of earnest men and women who had become believers in John’s preaching of the coming kingdom. John had been inquiring about Jesus of Zebedee’s sons. He had heard of Jesus’ remarks concerning his preaching, and he was day by day expecting to see him arrive on the scene, but he had not expected to greet him in the line of baptismal candidates.
\vs p135 8:5 Being engrossed with the details of rapidly baptizing such a large number of converts, John did not look up to see Jesus until the Son of Man stood in his immediate presence. When John recognized Jesus, the ceremonies were halted for a moment while he greeted his cousin in the flesh and asked, “But why do you come down into the water to greet me?” And Jesus answered, \textcolour{ubdarkred}{“To be subject to your baptism.”} John replied: “But I have need to be baptized by you. Why do you come to me?” And Jesus whispered to John: \textcolour{ubdarkred}{“Bear with me now, for it becomes us to set this example for my brothers standing here with me, and that the people may know that my hour has come.”}
\vs p135 8:6 \pc There was a tone of finality and authority in Jesus’ voice. John was atremble with emotion as he made ready to baptize Jesus of Nazareth in the Jordan at noon on Monday, January 14, A.D.\,26. Thus did John baptize Jesus and his two brothers James and Jude. And when John had baptized these three, he dismissed the others for the day, announcing that he would resume baptisms at noon the next day. As the people were departing, the four men still standing in the water heard a strange sound, and presently there appeared for a moment an apparition immediately over the head of Jesus, and they heard a voice saying, “This is my beloved Son in whom I am well pleased.” A great change came over the countenance of Jesus, and coming up out of the water in silence he took leave of them, going toward the hills to the east. And no man saw Jesus again for 40 days.\tunemarkup{private}{\begin{figure}[H]\centering\includegraphics[width=\tunemarkup{pgkoboaurahd}{0.8}\tunemarkup{pgnexus10}{0.9}\columnwidth]{images/Baptism.jpg}\caption{The Baptism of Jesus by William Hole}\end{figure}}
\vs p135 8:7 John followed Jesus a sufficient distance to tell him the story of Gabriel’s visit to his mother ere either had been born, as he had heard it so many times from his mother’s lips. He allowed Jesus to continue on his way after he had said, “Now I know of a certainty that you are the Deliverer.” But Jesus made no reply.
\usection{Forty Days of Preaching}
\vs p135 9:1 When John returned to his disciples (he now had some 25 or 30 who abode with him constantly), he found them in earnest conference, discussing what had just happened in connection with Jesus’ baptism. They were all the more astonished when John now made known to them the story of the Gabriel visitation to Mary before Jesus was born, and also that Jesus spoke no word to him even after he had told him about this. There was no rain that evening, and this group of 30 or more talked long into the starlit night. They wondered where Jesus had gone, and when they would see him again.
\vs p135 9:2 \pc After the experience of this day the preaching of John took on new and certain notes of proclamation concerning the coming kingdom and the expected Messiah. It was a tense time, these 40 days of tarrying, waiting for the return of Jesus. But John continued to preach with great power, and his disciples began at about this time to preach to the overflowing throngs which gathered around John at the Jordan.
\vs p135 9:3 In the course of these 40 days of waiting, many rumours spread about the countryside and even to Tiberias and Jerusalem. Thousands came over to see the new attraction in John’s camp, the reputed Messiah, but Jesus was not to be seen. When the disciples of John asserted that the strange man of God had gone to the hills, many doubted the entire story.
\vs p135 9:4 About three weeks after Jesus had left them, there arrived on the scene at Pella a new deputation from the priests and Pharisees at Jerusalem. They asked John directly if he was Elijah or the prophet that Moses promised; and when John said, “I am not,” they made bold to ask, “Are you the Messiah?” and John answered, “I am not.” Then said these men from Jerusalem: “If you are not Elijah, nor the prophet, nor the Messiah, then why do you baptize the people and create all this stir?” And John replied: “It should be for those who have heard me and received my baptism to say who I am, but I declare to you that, while I baptize with water, there has been among us one who will return to baptize you with the Holy Spirit.”
\vs p135 9:5 These 40 days were a difficult period for John and his disciples. What was to be the relation of John to Jesus? A hundred questions came up for discussion. Politics and selfish preferment began to make their appearance. Intense discussions grew up around the various ideas and concepts of the Messiah. Would he become a military leader and a Davidic king? Would he smite the Roman armies as Joshua had the Canaanites? Or would he come to establish a spiritual kingdom? John rather decided, with the minority, that Jesus had come to establish the kingdom of heaven, although he was not altogether clear in his own mind as to just what was to be embraced within this mission of the establishment of the kingdom of heaven.
\vs p135 9:6 These were strenuous days in John’s experience, and he prayed for the return of Jesus. Some of John’s disciples organized scouting parties to go in search of Jesus, but John forbade, saying: “Our times are in the hands of the God of heaven; he will direct his chosen Son.”
\vs p135 9:7 \pc It was early on the morning of Sabbath, February 23, that the company of John, engaged in eating their morning meal, looked up toward the north and beheld Jesus coming to them. As he approached them, John stood upon a large rock and, lifting up his sonorous voice, said: “Behold the Son of God, the deliverer of the world! This is he of whom I have said, ‘After me there will come one who is preferred before me because he was before me.’ For this cause came I out of the wilderness to preach repentance and to baptize with water, proclaiming that the kingdom of heaven is at hand. And now comes one who shall baptize you with the Holy Spirit. And I beheld the divine spirit descending upon this man, and I heard the voice of God declare, ‘This is my beloved Son in whom I am well pleased.’”
\vs p135 9:8 Jesus bade them return to their food while he sat down to eat with John, his brothers James and Jude having returned to Capernaum.
\vs p135 9:9 \pc Early in the morning of the next day he took leave of John and his disciples, going back to Galilee. He gave them no word as to when they would again see him. To John’s inquiries about his own preaching and mission Jesus only said, \textcolour{ubdarkred}{“My Father will guide you now and in the future as he has in the past.”} And these two great men separated that morning on the banks of the Jordan, never again to greet each other in the flesh.
\usection{John Journeys South}
\vs p135 10:1 Since Jesus had gone north into Galilee, John felt led to retrace his steps southward. Accordingly, on Sunday morning, March 3, John and the remainder of his disciples began their journey south. About one quarter of John’s immediate followers had meantime departed for Galilee in quest of Jesus. There was a sadness of confusion about John. He never again preached as he had before baptizing Jesus. He somehow felt that the responsibility of the coming kingdom was no longer on his shoulders. He felt that his work was almost finished; he was disconsolate and lonely. But he preached, baptized, and journeyed on southward.
\vs p135 10:2 Near the village of Adam, John tarried for several weeks, and it was here that he made the memorable attack upon Herod Antipas for unlawfully taking the wife of another man. By June of this year (A.D.\,26) John was back at the Bethany ford of the Jordan, where he had begun his preaching of the coming kingdom more than a year previously. In the weeks following the baptism of Jesus the character of John’s preaching gradually changed into a proclamation of mercy for the common people, while he denounced with renewed vehemence the corrupt political and religious rulers.
\vs p135 10:3 Herod Antipas, in whose territory John had been preaching, became alarmed lest he and his disciples should start a rebellion. Herod also resented John’s public criticisms of his domestic affairs. In view of all this, Herod decided to put John in prison. Accordingly, very early in the morning of June 12, before the multitude arrived to hear the preaching and witness the baptizing, the agents of Herod placed John under arrest. As weeks passed and he was not released, his disciples scattered over all Palestine, many of them going into Galilee to join the followers of Jesus.
\usection{John in Prison}
\vs p135 11:1 John had a lonely and somewhat bitter experience in prison. Few of his followers were permitted to see him. He longed to see Jesus but had to be content with hearing of his work through those of his followers who had become believers in the Son of Man. He was often tempted to doubt Jesus and his divine mission. If Jesus were the Messiah, why did he do nothing to deliver him from this unbearable imprisonment? For more than a year and a half this rugged man of God’s outdoors languished in that despicable prison. And this experience was a great test of his faith in, and loyalty to, Jesus. Indeed, this whole experience was a great test of John’s faith even in God. Many times was he tempted to doubt even the genuineness of his own mission and experience.
\vs p135 11:2 \pc After he had been in prison several months, a group of his disciples came to him and, after reporting concerning the public activities of Jesus, said: “So you see, Teacher, that he who was with you at the upper Jordan prospers and receives all who come to him. He even feasts with publicans and sinners. You bore courageous witness to him, and yet he does nothing to effect your deliverance.” But John answered his friends: “This man can do nothing unless it has been given him by his Father in heaven. You well remember that I said, ‘I am not the Messiah, but I am one sent on before to prepare the way for him.’ And that I did. He who has the bride is the bridegroom, but the friend of the bridegroom who stands near by\fnst{In 1955 text ``nar\hyp{}by''.} and hears him rejoices greatly because of the bridegroom’s voice. This, my joy, therefore is fulfilled. He must increase but I must decrease. I am of this earth and have declared my message. Jesus of Nazareth comes down to the earth from heaven and is above us all. The Son of Man has descended from God, and the words of God he will declare to you. For the Father in heaven gives not the spirit by measure to his own Son. The Father loves his Son and will presently put all things in the hands of this Son. He who believes in the Son has eternal life. And these words which I speak are true and abiding.”
\vs p135 11:3 \pc These disciples were amazed at John’s pronouncement, so much so that they departed in silence. John was also much agitated, for he perceived that he had uttered a prophecy. Never again did he wholly doubt the mission and divinity of Jesus. But it was a sore disappointment to John that Jesus sent him no word, that he came not to see him, and that he exercised none of his great power to deliver him from prison. But Jesus knew all about this. He had great love for John, but being now cognizant of his divine nature and knowing fully the great things in preparation for John when he departed from this world and also knowing that John’s work on earth was finished, he constrained himself not to interfere in the natural outworking of the great preacher\hyp{}prophet’s career.
\vs p135 11:4 \pc This long suspense in prison was humanly unbearable. Just a few days before his death John again sent trusted messengers to Jesus, inquiring: “Is my work done? Why do I languish in prison? Are you truly the Messiah, or shall we look for another?” And when these two disciples gave this message to Jesus, the Son of Man replied: \textcolour{ubdarkred}{“Go back to John and tell him that I have not forgotten but to suffer me also this, for it becomes us to fulfil all righteousness. Tell John what you have seen and heard --- that the poor have good tidings preached to them --- and, finally, tell the beloved herald of my earth mission that he shall be abundantly blessed in the age to come if he finds no occasion to doubt and stumble over me.”} And this was the last word John received from Jesus. This message greatly comforted him and did much to stabilize his faith and prepare him for the tragic end of his life in the flesh which followed so soon upon the heels of this memorable occasion.
\usection{Death of John the Baptist}
\vs p135 12:1 As John was working in southern Perea when arrested, he was taken immediately to the prison of the fortress of Machaerus, where he was incarcerated until his execution. Herod ruled over Perea as well as Galilee, and he maintained residence at this time at both Julias and Machaerus in Perea. In Galilee the official residence had been moved from Sepphoris to the new capital at Tiberias.
\vs p135 12:2 Herod feared to release John lest he instigate rebellion. He feared to put him to death lest the multitude riot in the capital, for thousands of Pereans believed that John was a holy man, a prophet. Therefore Herod kept the Nazarite preacher in prison, not knowing what else to do with him. Several times John had been before Herod, but never would he agree either to leave the domains of Herod or to refrain from all public activities if he were released. And this new agitation concerning Jesus of Nazareth, which was steadily increasing, admonished Herod that it was no time to turn John loose. Besides, John was also a victim of the intense and bitter hatred of Herodias, Herod’s unlawful wife.
\vs p135 12:3 On numerous occasions Herod talked with John about the kingdom of heaven, and while sometimes seriously impressed with his message, he was afraid to release him from prison.
\vs p135 12:4 Since much building was still going on at Tiberias, Herod spent considerable time at his Perean residences, and he was partial to the fortress of Machaerus. It was a matter of several years before all the public buildings and the official residence at Tiberias were fully completed.
\vs p135 12:5 \pc In celebration of his birthday Herod made a great feast in the Machaerian palace for his chief officers and other men high in the councils of the government of Galilee and Perea. Since Herodias had failed to bring about John’s death by direct appeal to Herod, she now set herself to the task of having John put to death by cunning planning.
\vs p135 12:6 In the course of the evening’s festivities and entertainment, Herodias presented her daughter to dance before the banqueters. Herod was very much pleased with the damsel’s performance and, calling her before him, said: “You are charming. I am much pleased with you. Ask me on this my birthday for whatever you desire, and I will give it to you, even to the half of my kingdom.” And Herod did all this while well under the influence of his many wines. The young lady drew aside and inquired of her mother what she should ask of Herod. Herodias said, “Go to Herod and ask for the head of John the Baptist.” And the young woman, returning to the banquet table, said to Herod, “I request that you forthwith give me the head of John the Baptist on a platter.”
\vs p135 12:7 Herod was filled with fear and sorrow, but because of his oath and because of all those who sat at meat with him, he would not deny the request. And Herod Antipas sent a soldier, commanding him to bring the head of John. So was John that night beheaded in the prison, the soldier bringing the head of the prophet on a platter and presenting it to the young woman at the rear of the banquet hall. And the damsel gave the platter to her mother. When John’s disciples heard of this, they came to the prison for the body of John, and after laying it in a tomb, they went and told Jesus.
\quizlink
