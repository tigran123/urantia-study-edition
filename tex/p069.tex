\upaper{69}{Primitive Human Institutions}
\uminitoc{Basic Human Institutions}
\uminitoc{The Dawn of Industry}
\uminitoc{The Specialization of Labour}
\uminitoc{The Beginnings of Trade}
\uminitoc{The Beginnings of Capital}
\uminitoc{Fire in Relation to Civilization}
\uminitoc{The Utilization of Animals}
\uminitoc{Slavery as a Factor in Civilization}
\uminitoc{Private Property}
\author{Melchizedek}
\vs p069 0:1 Emotionally, man transcends his animal ancestors in his ability to appreciate humour, art, and religion. Socially, man exhibits his superiority in that he is a toolmaker, a communicator, and an institution builder.
\vs p069 0:2 When human beings long maintain social groups, such aggregations always result in the creation of certain activity trends which culminate in institutionalization. Most of man’s institutions have proved to be laboursaving while at the same time contributing something to the enhancement of group security.
\vs p069 0:3 Civilized man takes great pride in the character, stability, and continuity of his established institutions, but all human institutions are merely the accumulated mores of the past as they have been conserved by taboos and dignified by religion. Such legacies become traditions, and traditions ultimately metamorphose into conventions.
\usection{Basic Human Institutions}
\vs p069 1:1 All human institutions minister to some social need, past or present, notwithstanding that their overdevelopment unfailingly detracts from the worth\hyp{}whileness of the individual in that personality is overshadowed and initiative is diminished. Man should control his institutions rather than permit himself to be dominated by these creations of advancing civilization.
\vs p069 1:2 \pc Human institutions are of three general classes:
\vs p069 1:3 \ublistelem{1.}\bibnobreakspace \bibemph{The institutions of self\hyp{}maintenance.}\tunemarkup{pgkoboaurahd}{\linebreak} These institutions embrace those practices growing out of food hunger and its associated instincts of self\hyp{}preservation. They include industry, property, war for gain, and all the regulative machinery of society. Sooner or later the fear instinct fosters the establishment of these institutions of survival by means of taboo, convention, and religious sanction. But fear, ignorance, and superstition have played a prominent part in the early origin and subsequent development of all human institutions.
\vs p069 1:4 \ublistelem{2.}\bibnobreakspace \bibemph{The institutions of self\hyp{}perpetuation.}\tunemarkup{pgkoboaurahd}{\linebreak} These are the establishments of society growing out of sex hunger, maternal instinct, and the higher tender emotions of the races. They embrace the social safeguards of the home and the school, of family life, education, ethics, and religion. They include marriage customs, war for defence, and home building.
\vs p069 1:5 \ublistelem{3.}\bibnobreakspace \bibemph{The institutions of self\hyp{}gratification.}\tunemarkup{pgkoboaurahd}{\linebreak} These are the practices growing out of vanity proclivities and pride emotions; and they embrace customs in dress and personal adornment, social usages, war for glory, dancing, amusement, games, and other phases of sensual gratification. But civilization has never evolved distinctive institutions of self\hyp{}gratification.
\vs p069 1:6 \pc These three groups of social practices are intimately interrelated and minutely interdependent the one upon the other. On Urantia they represent a complex organization which functions as a single social mechanism.
\usection{The Dawn of Industry}
\vs p069 2:1 Primitive industry slowly grew up as an insurance against the terrors of famine. Early in his existence man began to draw lessons from some of the animals that, during a harvest of plenty, store up food against the days of scarcity.
\vs p069 2:2 Before the dawn of early frugality and primitive industry the lot of the average tribe was one of destitution and real suffering. Early man had to compete with the whole animal world for his food. Competition\hyp{}gravity ever pulls man down toward the beast level; poverty is his natural and tyrannical estate. Wealth is not a natural gift; it results from labour, knowledge, and organization.
\vs p069 2:3 Primitive man was not slow to recognize the advantages of association. Association led to organization, and the first result of organization was division of labour, with its immediate saving of time and materials. These specializations of labour arose by adaptation to pressure --- pursuing the paths of lessened resistance. Primitive savages never did any real work cheerfully or willingly. With them conformity was due to the coercion of necessity.
\vs p069 2:4 Primitive man disliked hard work, and he would not hurry unless confronted by grave danger. The time element in labour, the idea of doing a given task within a certain time limit, is entirely a modern notion. The ancients were never rushed. It was the double demands of the intense struggle for existence and of the ever\hyp{}advancing standards of living that drove the naturally inactive races of early man into avenues of industry.
\vs p069 2:5 Labour, the efforts of design, distinguishes man from the beast, whose exertions are largely instinctive. The necessity for labour is man’s paramount blessing. The Prince’s staff all worked; they did much to ennoble physical labour on Urantia. Adam was a gardener; the God of the Hebrews laboured --- he was the creator and upholder of all things. The Hebrews were the first tribe to put a supreme premium on industry; they were the first people to decree that “he who does not work shall not eat.” But many of the religions of the world reverted to the early ideal of idleness. Jupiter was a reveller, and Buddha became a reflective devotee of leisure.
\vs p069 2:6 The Sangik tribes were fairly industrious when residing away from the tropics. But there was a long, long struggle between the lazy devotees of magic and the apostles of work --- those who exercised foresight.
\vs p069 2:7 The first human foresight was directed toward the preservation of fire, water, and food. But primitive man was a natural\hyp{}born gambler; he always wanted to get something for nothing, and all too often during these early times the success which accrued from patient practice was attributed to charms. Magic was slow to give way before foresight, self\hyp{}denial, and industry.
\usection{The Specialization of Labour}
\vs p069 3:1 The divisions of labour in primitive society were determined first by natural, and then by social, circumstances. The early order of specialization in labour was:
\vs p069 3:2 \ublistelem{1.}\bibnobreakspace \bibemph{Specialization based on sex.} Woman’s work was derived from the selective presence of the child; women naturally love babies more than men do. Thus woman became the routine worker, while man became the hunter and fighter, engaging in accentuated periods of work and rest.
\vs p069 3:3 \pc All down through the ages the taboos have operated to keep woman strictly in her own field. Man has most selfishly chosen the more agreeable work, leaving the routine drudgery to woman. Man has always been ashamed to do woman’s work, but woman has never shown any reluctance to doing man’s work. But strange to record, both men and women have always worked together in building and furnishing the home.
\vs p069 3:4 \ublistelem{2.}\bibnobreakspace \bibemph{Modification consequent upon age and disease.} These differences determined the next division of labour. The old men and cripples were early set to work making tools and weapons. They were later assigned to building irrigation works.
\vs p069 3:5 \ublistelem{3.}\bibnobreakspace \bibemph{Differentiation based on religion.} The medicine men were the first human beings to be exempted from physical toil; they were the pioneer professional class. The smiths were a small group who competed with the medicine men as magicians. Their skill in working with metals made the people afraid of them. The “white smiths” and the “black smiths” gave origin to the early beliefs in white and black magic. And this belief later became involved in the superstition of good and bad ghosts, good and bad spirits.
\vs p069 3:6 Smiths were the first nonreligious group to enjoy special privileges. They were regarded as neutrals during war, and this extra leisure led to their becoming, as a class, the politicians of primitive society. But through gross abuse of these privileges the smiths became universally hated, and the medicine men lost no time in fostering hatred for their competitors. In this first contest between science and religion, religion (superstition) won. After being driven out of the villages, the smiths maintained the first inns, public lodginghouses, on the outskirts of the settlements.
\vs p069 3:7 \ublistelem{4.}\bibnobreakspace \bibemph{Master and slave.} The next differentiation of labour grew out of the relations of the conqueror to the conquered, and that meant the beginning of human slavery.
\vs p069 3:8 \ublistelem{5.}\bibnobreakspace \bibemph{Differentiation based on diverse physical and mental endowments.} Further divisions of labour were favoured by the inherent differences in men; all human beings are not born equal.
\vs p069 3:9 The early specialists in industry were the flint flakers and stone masons\fnst{In 1955 text ``stonemasons'' one word.}; next came the smiths. Subsequently group specialization developed; whole families and clans dedicated themselves to certain sorts of labour. The origin of one of the earliest castes of priests, apart from the tribal medicine men, was due to the superstitious exaltation of a family of expert swordmakers.
\vs p069 3:10 \pc The first group specialists in industry were rock salt exporters and potters. Women made the plain pottery and men the fancy. Among some tribes sewing and weaving were done by women, in others by the men.
\vs p069 3:11 The early traders were women; they were employed as spies, carrying on commerce as a side line. Presently trade expanded, the women acting as intermediaries --- jobbers. Then came the merchant class, charging a commission, profit, for their services. Growth of group barter developed into commerce; and following the exchange of commodities came the exchange of skilled labour.
\usection{The Beginnings of Trade}
\vs p069 4:1 Just as marriage by contract followed marriage by capture, so trade by barter followed seizure by raids. But a long period of piracy intervened between the early practices of silent barter and the later trade by modern exchange methods.
\vs p069 4:2 The first barter was conducted by armed traders who would leave their goods on a neutral spot. Women held the first markets; they were the earliest traders, and this was because they were the burden bearers; the men were warriors. Very early the trading counter was developed, a wall wide enough to prevent the traders reaching each other with weapons.
\vs p069 4:3 A fetish was used to stand guard over the deposits of goods for silent barter. Such market places were secure against theft; nothing would be removed except by barter or purchase; with a fetish on guard the goods were always safe. The early traders were scrupulously honest within their own tribes but regarded it as all right to cheat distant strangers. Even the early Hebrews recognized a separate code of ethics in their dealings with the gentiles.
\vs p069 4:4 For ages silent barter continued before men would meet, unarmed, on the sacred market place. These same market squares became the first places of sanctuary and in some countries were later known as “cities of refuge.” Any fugitive reaching the market place was safe and secure against attack.
\vs p069 4:5 \pc The first weights were grains of wheat and other cereals. The first medium of exchange was a fish or a goat. Later the cow became a unit of barter.
\vs p069 4:6 Modern writing originated in the early trade records; the first literature of man was a trade\hyp{}promotion document, a salt advertisement. Many of the earlier wars were fought over natural deposits, such as flint, salt, and metals. The first formal tribal treaty concerned the intertribalizing of a salt deposit. These treaty spots afforded opportunity for friendly and peaceful interchange of ideas and the intermingling of various tribes.
\vs p069 4:7 Writing progressed up through the stages of the “message stick,” knotted cords, picture writing, hieroglyphics, and wampum belts, to the early symbolic alphabets. Message sending evolved from the primitive smoke signal up through runners, animal riders, railroads, and airplanes, as well as telegraph, telephone, and wireless communication.
\vs p069 4:8 New ideas and better methods were carried around the inhabited world by the ancient traders. Commerce, linked with adventure, led to exploration and discovery. And all of these gave birth to transportation. Commerce has been the great civilizer through promoting the cross\hyp{}fertilization of culture.
\usection{The Beginnings of Capital}
\vs p069 5:1 Capital is labour applied as a renunciation of the present in favour of the future. Savings represent a form of maintenance and survival insurance. Food hoarding developed self\hyp{}control and created the first problems of capital and labour. The man who had food, provided he could protect it from robbers, had a distinct advantage over the man who had no food.
\vs p069 5:2 The early banker was the valorous man of the tribe. He held the group treasures on deposit, while the entire clan would defend his hut in event of attack. Thus the accumulation of individual capital and group wealth immediately led to military organization. At first such precautions were designed to defend property against foreign raiders, but later on it became the custom to keep the military organization in practice by inaugurating raids on the property and wealth of neighbouring tribes.
\vs p069 5:3 The basic urges which led to the accumulation of capital were:
\vs p069 5:4 \ublistelem{1.}\bibnobreakspace \bibemph{Hunger --- associated with foresight.} Food saving and preservation meant power and comfort for those who possessed sufficient \bibemph{foresight} thus to provide for future needs. Food storage was adequate insurance against famine and disaster. And the entire body of primitive mores was really designed to help man subordinate the present to the future.
\vs p069 5:5 \ublistelem{2.}\bibnobreakspace \bibemph{Love of family ---} desire to provide for their wants. Capital represents the saving of property in spite of the pressure of the wants of today in order to insure against the demands of the future. A part of this future need may have to do with one’s posterity.
\vs p069 5:6 \ublistelem{3.}\bibnobreakspace \bibemph{Vanity ---} longing to display one’s property accumulations. Extra clothing was one of the first badges of distinction. Collection vanity early appealed to the pride of man.
\vs p069 5:7 \ublistelem{4.}\bibnobreakspace \bibemph{Position ---} eagerness to buy social and political prestige. There early sprang up a commercialized nobility, admission to which depended on the performance of some special service to royalty or was granted frankly for the payment of money.
\vs p069 5:8 \ublistelem{5.}\bibnobreakspace \bibemph{Power ---} the craving to be master. Treasure lending was carried on as a means of enslavement, 100\% a year being the loan rate of these ancient times. The moneylenders made themselves kings by creating a standing army of debtors. Bond servants were among the earliest form of property to be accumulated, and in olden days debt slavery extended even to the control of the body after death.
\vs p069 5:9 \ublistelem{6.}\bibnobreakspace \bibemph{Fear of the ghosts of the dead ---} priest fees for protection. Men early began to give death presents to the priests with a view to having their property used to facilitate their progress through the next life. The priesthoods thus became very rich; they were chief among ancient capitalists.
\vs p069 5:10 \ublistelem{7.}\bibnobreakspace \bibemph{Sex urge ---} the desire to buy one or more wives. Man’s first form of trading was woman exchange; it long preceded horse trading. But never did the barter in sex slaves advance society; such traffic was and is a racial disgrace, for at one and the same time it hindered the development of family life and polluted the biologic fitness of superior peoples.
\vs p069 5:11 \ublistelem{8.}\bibnobreakspace \bibemph{Numerous forms of self\hyp{}gratification.}\tunemarkup{pgkoboaurahd}{\linebreak} Some sought wealth because it conferred power; others toiled for property because it meant ease. Early man (and some later\hyp{}day ones) tended to squander his resources on luxury. Intoxicants and drugs intrigued the primitive races.
\vs p069 5:12 \pc As civilization developed, men acquired new incentives for saving; new wants were rapidly added to the original food hunger. Poverty became so abhorred that only the rich were supposed to go direct to heaven when they died. Property became so highly valued that to give a pretentious feast would wipe a dishonour from one’s name.
\vs p069 5:13 Accumulations of wealth early became the badge of social distinction. Individuals in certain tribes would accumulate property for years just to create an impression by burning it up on some holiday or by freely distributing it to fellow tribesmen. This made them great men. Even modern peoples revel in the lavish distribution of Christmas gifts, while rich men endow great institutions of philanthropy and learning. Man’s technique varies, but his disposition remains quite unchanged.
\vs p069 5:14 But it is only fair to record that many an ancient rich man distributed much of his fortune because of the fear of being killed by those who coveted his treasures. Wealthy men commonly sacrificed scores of slaves to show disdain for wealth.
\vs p069 5:15 Though capital has tended to liberate man, it has greatly complicated his social and industrial organization. The abuse of capital by unfair capitalists does not destroy the fact that it is the basis of modern industrial society. Through capital and invention the present generation enjoys a higher degree of freedom than any that ever preceded it on earth. This is placed on record as a fact and not in justification of the many misuses of capital by thoughtless and selfish custodians.
\usection{Fire in Relation to Civilization}
\vs p069 6:1 Primitive society with its four divisions --- industrial, regulative, religious, and military --- rose through the instrumentality of fire, animals, slaves, and property.
\vs p069 6:2 Fire building, by a single bound, forever separated man from animal; it is the basic human invention, or discovery. Fire enabled man to stay on the ground at night as all animals are afraid of it. Fire encouraged eventide social intercourse; it not only protected against cold and wild beasts but was also employed as security against ghosts. It was at first used more for light than heat; many backward tribes refuse to sleep unless a flame burns all night.
\vs p069 6:3 Fire was a great civilizer, providing man with his first means of being altruistic without loss by enabling him to give live coals to a neighbour without depriving himself. The household fire, which was attended by the mother or eldest daughter, was the first educator, requiring watchfulness and dependability. The early home was not a building but the family gathered about the fire, the family hearth. When a son founded a new home, he carried a firebrand from the family hearth.
\vs p069 6:4 \pc Though Andon, the discoverer of fire, avoided treating it as an object of worship, many of his descendants regarded the flame as a fetish or as a spirit. They failed to reap the sanitary benefits of fire because they would not burn refuse. Primitive man feared fire and always sought to keep it in good humour, hence the sprinkling of incense. Under no circumstances would the ancients spit in a fire, nor would they ever pass between anyone and a burning fire. Even the iron pyrites and flints used in striking fire were held sacred by early mankind.
\vs p069 6:5 It was a sin to extinguish a flame; if a hut caught fire, it was allowed to burn. The fires of the temples and shrines were sacred and were never permitted to go out except that it was the custom to kindle new flames annually or after some calamity. Women were selected as priests because they were custodians of the home fires.
\vs p069 6:6 The early myths about how fire came down from the gods grew out of the observations of fire caused by lightning. These ideas of supernatural origin led directly to fire worship, and fire worship led to the custom of “passing through fire,” a practice carried on up to the times of Moses. And there still persists the idea of passing through fire after death. The fire myth was a great bond in early times and still persists in the symbolism of the Parsees.
\vs p069 6:7 \pc Fire led to cooking, and “raw eaters” became a term of derision. And cooking lessened the expenditure of vital energy necessary for the digestion of food and so left early man some strength for social culture, while animal husbandry, by reducing the effort necessary to secure food, provided time for social activities.
\vs p069 6:8 It should be remembered that fire opened the doors to metalwork and led to the subsequent discovery of steam power and the present\hyp{}day uses of electricity.
\usection{The Utilization of Animals}
\vs p069 7:1 To start with, the entire animal world was man’s enemy; human beings had to learn to protect themselves from the beasts. First, man ate the animals but later learned to domesticate and make them serve him.
\vs p069 7:2 The domestication of animals came about accidentally. The savage would hunt herds much as the American Indians hunted the bison. By surrounding the herd they could keep control of the animals, thus being able to kill them as they were required for food. Later, corrals were constructed, and entire herds would be captured.
\vs p069 7:3 It was easy to tame some animals, but like the elephant, many of them would not reproduce in captivity. Still further on it was discovered that certain species of animals would submit to man’s presence, and that they would reproduce in captivity. The domestication of animals was thus promoted by selective breeding, an art which has made great progress since the days of Dalamatia.
\vs p069 7:4 The dog was the first animal to be domesticated, and the difficult experience of taming it began when a certain dog, after following a hunter around all day, actually went home with him. For ages dogs were used for food, hunting, transportation, and companionship. At first dogs only howled, but later on they learned to bark. The dog’s keen sense of smell led to the notion it could see spirits, and thus arose the dog\hyp{}fetish cults. The employment of watchdogs made it first possible for the whole clan to sleep at night. It then became the custom to employ watchdogs to protect the home against spirits as well as material enemies. When the dog barked, man or beast approached, but when the dog howled, spirits were near. Even now many still believe that a dog’s howling at night betokens death.
\vs p069 7:5 When man was a hunter, he was fairly kind to woman, but after the domestication of animals, coupled with the Caligastia confusion, many tribes shamefully treated their women. They treated them altogether too much as they treated their animals. Man’s brutal treatment of woman constitutes one of the darkest chapters of human history.
\usection{Slavery as a Factor in Civilization}
\vs p069 8:1 Primitive man never hesitated to enslave his fellows. Woman was the first slave, a family slave. Pastoral man enslaved woman as his inferior sex partner. This sort of sex slavery grew directly out of man’s decreased dependence upon woman.
\vs p069 8:2 Not long ago enslavement was the lot of those military captives who refused to accept the conqueror’s religion. In earlier times captives were either eaten, tortured to death, set to fighting each other, sacrificed to spirits, or enslaved. Slavery was a great advancement over massacre and cannibalism.
\vs p069 8:3 Enslavement was a forward step in the merciful treatment of war captives. The ambush of Ai, with the wholesale slaughter of men, women, and children, only the king being saved to gratify the conqueror’s vanity, is a faithful picture of the barbaric slaughter practised by even supposedly civilized peoples. The raid upon Og, the king of Bashan, was equally brutal and effective. The Hebrews “utterly destroyed” their enemies, taking all their property as spoils. They put all cities under tribute on pain of the “destruction of all males.” But many of the contemporary tribes, those having less tribal egotism, had long since begun to practise the adoption of superior captives.
\vs p069 8:4 The hunter, like the American red man, did not enslave. He either adopted or killed his captives. Slavery was not prevalent among the pastoral peoples, for they needed few labourers. In war the herders made a practice of killing all men captives and taking as slaves only the women and children. The Mosaic code contained specific directions for making wives of these women captives. If not satisfactory, they could be sent away, but the Hebrews were not allowed to sell such rejected consorts as slaves --- that was at least one advance in civilization. Though the social standards of the Hebrews were crude, they were far above those of the surrounding tribes.
\vs p069 8:5 The herders were the first capitalists; their herds represented capital, and they lived on the interest --- the natural increase. And they were disinclined to trust this wealth to the keeping of either slaves or women. But later on they took male prisoners and forced them to cultivate the soil. This is the early origin of serfdom --- man attached to the land. The Africans could easily be taught to till the soil; hence they became the great slave race.
\vs p069 8:6 \pc Slavery was an indispensable link in the chain of human civilization. It was the bridge over which society passed from chaos and indolence to order and civilized activities; it compelled backward and lazy peoples to work and thus provide wealth and leisure for the social advancement of their superiors.
\vs p069 8:7 The institution of slavery compelled man to invent the regulative mechanism of primitive society; it gave origin to the beginnings of government. Slavery demands strong regulation and during the European Middle Ages virtually disappeared because the feudal lords could not control the slaves. The backward tribes of ancient times, like the native Australians of today, never had slaves.
\vs p069 8:8 True, slavery was oppressive, but it was in the schools of oppression that man learned industry. Eventually the slaves shared the blessings of a higher society which they had so unwillingly helped create. Slavery creates an organization of culture and social achievement but soon insidiously attacks society internally as the gravest of all destructive social maladies.
\vs p069 8:9 \pc Modern mechanical invention rendered the slave obsolete. Slavery, like polygamy, is passing because it does not pay. But it has always proved disastrous suddenly to liberate great numbers of slaves; less trouble ensues when they are gradually emancipated.
\vs p069 8:10 \pc Today, men are not social slaves, but thousands allow ambition to enslave them to debt. Involuntary slavery has given way to a new and improved form of modified industrial servitude.
\vs p069 8:11 While the ideal of society is universal freedom, idleness should never be tolerated. All able\hyp{}bodied persons should be compelled to do at least a self\hyp{}sustaining amount of work.
\vs p069 8:12 Modern society is in reverse. Slavery has nearly disappeared; domesticated animals are passing. Civilization is reaching back to fire --- the inorganic world --- for power. Man came up from savagery by way of fire, animals, and slavery; today he reaches back, discarding the help of slaves and the assistance of animals, while he seeks to wrest new secrets and sources of wealth and power from the elemental storehouse of nature.
\usection{Private Property}
\vs p069 9:1 While primitive society was virtually communal, primitive man did not adhere to the modern doctrines of communism. The communism of these early times was not a mere theory or social doctrine; it was a simple and practical automatic adjustment. Communism prevented pauperism and want; begging and prostitution were almost unknown among these ancient tribes.
\vs p069 9:2 \pc Primitive communism did not especially level men down, nor did it exalt mediocrity, but it did put a premium on inactivity and idleness, and it did stifle industry and destroy ambition. Communism was indispensable scaffolding in the growth of primitive society, but it gave way to the evolution of a higher social order because it ran counter to four strong human proclivities:
\vs p069 9:3 \ublistelem{1.}\bibnobreakspace \bibemph{The family.} Man not only craves to accumulate property; he desires to bequeath his capital goods to his progeny. But in early communal society a man’s capital was either immediately consumed or distributed among the group at his death. There was no inheritance of property --- the inheritance tax was 100\%. The later capital\hyp{}accumulation and property\hyp{}inheritance mores were a distinct social advance. And this is true notwithstanding the subsequent gross abuses attendant upon the misuse of capital.
\vs p069 9:4 \ublistelem{2.}\bibnobreakspace \bibemph{Religious tendencies.} Primitive man also wanted to save up property as a nucleus for starting life in the next existence. This motive explains why it was so long the custom to bury a man’s personal belongings with him. The ancients believed that only the rich survived death with any immediate pleasure and dignity. The teachers of revealed religion, more especially the Christian teachers, were the first to proclaim that the poor could have salvation on equal terms with the rich.
\vs p069 9:5 \ublistelem{3.}\bibnobreakspace \bibemph{The desire for liberty and leisure.} In the earlier days of social evolution the apportionment of individual earnings among the group was virtually a form of slavery; the worker was made slave to the idler. This was the suicidal weakness of communism: The improvident habitually lived off the thrifty. Even in modern times the improvident depend on the state (thrifty taxpayers) to take care of them. Those who have no capital still expect those who have to feed them.
\vs p069 9:6 \ublistelem{4.}\bibnobreakspace \bibemph{The urge for security and power.} Communism was finally destroyed by the deceptive practices of progressive and successful individuals who resorted to diverse subterfuges in an effort to escape enslavement to the shiftless idlers of their tribes. But at first all hoarding was secret; primitive insecurity prevented the outward accumulation of capital. And even at a later time it was most dangerous to amass too much wealth; the king would be sure to trump up some charge for confiscating a rich man’s property, and when a wealthy man died, the funeral was held up until the family donated a large sum to public welfare or to the king, an inheritance tax.
\vs p069 9:7 In earliest times women were the property of the community, and the mother dominated the family. The early chiefs owned all the land and were proprietors of all the women; marriage required the consent of the tribal ruler. With the passing of communism, women were held individually, and the father gradually assumed domestic control. Thus the home had its beginning, and the prevailing polygamous customs were gradually displaced by monogamy. (Polygamy is the survival of the female\hyp{}slavery element in marriage. Monogamy is the slave\hyp{}free ideal of the matchless association of one man and one woman in the exquisite enterprise of home building, offspring rearing, mutual culture, and self\hyp{}improvement.)
\vs p069 9:8 At first, all property, including tools and weapons, was the common possession of the tribe. Private property first consisted of all things personally touched. If a stranger drank from a cup, the cup was henceforth his. Next, any place where blood was shed became the property of the injured person or group.
\vs p069 9:9 Private property was thus originally respected because it was supposed to be charged with some part of the owner’s personality. Property honesty rested safely on this type of superstition; no police were needed to guard personal belongings. There was no stealing within the group, though men did not hesitate to appropriate the goods of other tribes. Property relations did not end with death; early, personal effects were burned, then buried with the dead, and later, inherited by the surviving family or by the tribe.
\vs p069 9:10 The ornamental type of personal effects originated in the wearing of charms. Vanity plus ghost fear led early man to resist all attempts to relieve him of his favourite charms, such property being valued above necessities.
\vs p069 9:11 \pc Sleeping space was one of man’s earliest properties. Later, homesites were assigned by the tribal chiefs, who held all real estate in trust for the group. Presently a fire site conferred ownership; and still later, a well constituted title to the adjacent land.
\vs p069 9:12 Water holes and wells were among the first private possessions. The whole fetish practice was utilized to guard water holes, wells, trees, crops, and honey. Following the loss of faith in the fetish, laws were evolved to protect private belongings. But game laws, the right to hunt, long preceded land laws. The American red man never understood private ownership of land; he could not comprehend the white man’s view.
\vs p069 9:13 Private property was early marked by family insignia, and this is the early origin of family crests. Real estate could also be put under the watchcare of spirits. The priests would “consecrate” a piece of land, and it would then rest under the protection of the magic taboos erected thereon. Owners thereof were said to have a “priest’s title.” The Hebrews had great respect for these family landmarks: “Cursed be he who removes his neighbour’s landmark.” These stone markers bore the priest’s initials. Even trees, when initialled, became private property.
\vs p069 9:14 In early days only the crops were private, but successive crops conferred title; agriculture was thus the genesis of the private ownership of land. Individuals were first given only a life tenureship; at death land reverted to the tribe. The very first land titles granted by tribes to individuals were graves --- family burying grounds. In later times land belonged to those who fenced it. But the cities always reserved certain lands for public pasturage and for use in case of siege; these “commons” represent the survival of the earlier form of collective ownership.
\vs p069 9:15 Eventually the state assigned property to the individual, reserving the right of taxation. Having made secure their titles, landlords could collect rents, and land became a source of income --- capital. Finally land became truly negotiable, with sales, transfers, mortgages, and foreclosures.
\vs p069 9:16 Private ownership brought increased liberty and enhanced stability; but private ownership of land was given social sanction only after communal control and direction had failed, and it was soon followed by a succession of slaves, serfs, and landless classes. But improved machinery is gradually setting men free from slavish toil.
\vs p069 9:17 The right to property is not absolute; it is purely social. But all government, law, order, civil rights, social liberties, conventions, peace, and happiness, as they are enjoyed by modern peoples, have grown up around the private ownership of property.
\vs p069 9:18 The present social order is not necessarily right --- not divine or sacred --- but mankind will do well to move slowly in making changes. That which you have is vastly better than any system known to your ancestors. Make certain that when you change the social order you change for the better. Do not be persuaded to experiment with the discarded formulas of your forefathers. Go forward, not backward! Let evolution proceed! Do not take a backward step.
\vsetoff
\vs p069 9:19 [Presented by a Melchizedek of Nebadon.]
\quizlink
