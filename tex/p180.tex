\upaper{180}{The Farewell Discourse}
\uminitoc{The New Commandment}
\uminitoc{The Vine and the Branches}
\uminitoc{Enmity of the World}
\uminitoc{The Promised Helper}
\uminitoc{The Spirit of Truth}
\uminitoc{The Necessity for Leaving}
\author{Midwayer Commission}
\vs p180 0:1 After singing the Psalm at the conclusion of the Last Supper, the apostles thought that Jesus intended to return immediately to the camp, but he indicated that they should sit down. Said the Master:
\vs p180 0:2 \textcolour{ubdarkred}{“You well remember when I sent you forth without purse or wallet and even advised that you take with you no extra clothes. And you will all recall that you lacked nothing. But now have you come upon troublous times. No longer can you depend upon the good will of the multitudes. Henceforth, he who has a purse, let him take it with him. When you go out into the world to proclaim this gospel, make such provision for your support as seems best. I have come to bring peace, but it will not appear for a time.}
\vs p180 0:3 \textcolour{ubdarkred}{“The time has now come for the Son of Man to be glorified, and the Father shall be glorified in me. My friends, I am to be with you only a little longer. Soon you will seek for me, but you will not find me, for I am going to a place to which you cannot, at this time, come. But when you have finished your work on earth as I have now finished mine, you shall then come to me even as I now prepare to go to my Father. In just a short time I am going to leave you, you will see me no more on earth, but you shall all see me in the age to come when you ascend to the kingdom which my Father has given to me.”}
\usection{The New Commandment}
\vs p180 1:1 After a few moments of informal conversation, Jesus stood up and said: \textcolour{ubdarkred}{“When I enacted for you a parable indicating how you should be willing to serve one another, I said that I desired to give you a new commandment; and I would do this now as I am about to leave you. You well know the commandment which directs that you love one another; that you love your neighbour even as yourself. But I am not wholly satisfied with even that sincere devotion on the part of my children. I would have you perform still greater acts of love in the kingdom of the believing brotherhood. And so I give you this new commandment: That you love one another even as I have loved you. And by this will all men know that you are my disciples if you thus love one another.}
\vs p180 1:2 \textcolour{ubdarkred}{“When I give you this new commandment, I do not place any new burden upon your souls; rather do I bring you new joy and make it possible for you to experience new pleasure in knowing the delights of the bestowal of your heart’s affection upon your fellow men. I am about to experience the supreme joy, even though enduring outward sorrow, in the bestowal of my affection upon you and your fellow mortals.}
\vs p180 1:3 \textcolour{ubdarkred}{“When I invite you to love one another, even as I have loved you, I hold up before you the supreme measure of true affection, for greater love can no man have than this: that he will lay down his life for his friends. And you are my friends; you will continue to be my friends if you are but willing to do what I have taught you. You have called me Master, but I do not call you servants. If you will only love one another as I am loving you, you shall be my friends, and I will ever speak to you of that which the Father reveals to me.}
\vs p180 1:4 \textcolour{ubdarkred}{“You have not merely chosen me, but I have also chosen you, and I have ordained you to go forth into the world to yield the fruit of loving service to your fellows even as I have lived among you and revealed the Father to you. The Father and I will both work with you, and you shall experience the divine fullness of joy if you will only obey my command to love one another, even as I have loved you.”}
\vs p180 1:5 \pc If you would share the Master’s joy, you must share his love. And to share his love means that you have shared his service. Such an experience of love does not deliver you from the difficulties of this world; it does not create a new world, but it most certainly does make the old world new.
\vs p180 1:6 Keep in mind: It is loyalty, not sacrifice, that Jesus demands. The consciousness of sacrifice implies the absence of that wholehearted affection which would have made such a loving service a supreme joy. The idea of \bibemph{duty} signifies that you are servant\hyp{}minded and hence are missing the mighty thrill of doing your service as a friend and for a friend. The impulse of friendship transcends all convictions of duty, and the service of a friend for a friend can never be called a sacrifice. The Master has taught the apostles that they are the sons of God. He has called them brethren, and now, before he leaves, he calls them his friends.
\usection{The Vine and the Branches}
\vs p180 2:1 Then Jesus stood up again and continued teaching his apostles: \textcolour{ubdarkred}{“I am the true vine, and my Father is the husbandman. I am the vine, and you are the branches. And the Father requires of me only that you shall bear much fruit. The vine is pruned only to increase the fruitfulness of its branches. Every branch coming out of me which bears no fruit, the Father will take away. Every branch which bears fruit, the Father will cleanse that it may bear more fruit. Already are you clean through the word I have spoken, but you must continue to be clean. You must abide in me, and I in you; the branch will die if it is separated from the vine. As the branch cannot bear fruit except it abides in the vine, so neither can you yield the fruits of loving service except you abide in me. Remember: I am the real vine, and you are the living branches. He who lives in me, and I in him, will bear much fruit of the spirit and experience the supreme joy of yielding this spiritual harvest. If you will maintain this living spiritual connection with me, you will bear abundant fruit. If you abide in me and my words live in you, you will be able to commune freely with me, and then can my living spirit so infuse you that you may ask whatsoever my spirit wills and do all this with the assurance that the Father will grant us our petition. Herein is the Father glorified: that the vine has many living branches, and that every branch bears much fruit. And when the world sees these fruit\hyp{}bearing branches --- my friends who love one another, even as I have loved them --- all men will know that you are truly my disciples.}
\vs p180 2:2 \textcolour{ubdarkred}{“As the Father has loved me, so have I loved you. Live in my love even as I live in the Father’s love. If you do as I have taught you, you shall abide in my love even as I have kept the Father’s word and evermore abide in his love.”}
\vs p180 2:3 The Jews had long taught that the Messiah would be “a stem arising out of the vine” of David’s ancestors, and in commemoration of this olden teaching a large emblem of the grape and its attached vine decorated the entrance to Herod’s temple. The apostles all recalled these things while the Master talked to them this night in the upper chamber.
\vs p180 2:4 But great sorrow later attended the misinterpretation of the Master’s inferences regarding prayer. There would have been little difficulty about these teachings if his exact words had been remembered and subsequently truthfully recorded. But as the record was made, believers eventually regarded prayer in Jesus’ name as a sort of supreme magic, thinking that they would receive from the Father anything they asked for. For centuries honest souls have continued to wreck their faith against this stumbling block. How long will it take the world of believers to understand that prayer is not a process of getting your way but rather a program of taking God’s way, an experience of learning how to recognize and execute the Father’s will? It is entirely true that, when your will has been truly aligned with his, you can ask anything conceived by that will\hyp{}union, and it will be granted. And such a will\hyp{}union is effected by and through Jesus even as the life of the vine flows into and through the living branches.
\vs p180 2:5 When there exists this living connection between divinity and humanity, if humanity should thoughtlessly and ignorantly pray for selfish ease and vainglorious accomplishments, there could be only one divine answer: more and increased bearing of the fruits of the spirit on the stems of the living branches. When the branch of the vine is alive, there can be only one answer to all its petitions: increased grape bearing. In fact, the branch exists only for, and can do nothing except, fruit bearing, yielding grapes. So does the true believer exist only for the purpose of bearing the fruits of the spirit: to love man as he himself has been loved by God --- that we should love one another, even as Jesus has loved us.
\vs p180 2:6 And when the Father’s hand of discipline is laid upon the vine, it is done in love, in order that the branches may bear much fruit. And a wise husbandman cuts away only the dead and fruitless branches.
\vs p180 2:7 Jesus had great difficulty in leading even his apostles to recognize that prayer is a function of spirit\hyp{}born believers in the spirit\hyp{}dominated kingdom.
\usection{Enmity of the World}
\vs p180 3:1 The eleven had scarcely ceased their discussions of the discourse on the vine and the branches when the Master, indicating that he was desirous of speaking to them further and knowing that his time was short, said: “When I have left you, be not discouraged by the enmity of the world. Be not downcast even when fainthearted believers turn against you and join hands with the enemies of the kingdom. If the world shall hate you, you should recall that it hated me even before it hated you. If you were of this world, then would the world love its own, but because you are not, the world refuses to love you. You are in this world, but your lives are not to be worldlike. I have chosen you out of the world to represent the spirit of another world even to this world from which you have been chosen. But always remember the words I have spoken to you: The servant is not greater than his master. If they dare to persecute me, they will also persecute you. If my words offend the unbelievers, so also will your words offend the ungodly. And all of this will they do to you because they believe not in me nor in Him who sent me; so will you suffer many things for the sake of my gospel. But when you endure these tribulations, you should recall that I also suffered before you for the sake of this gospel of the heavenly kingdom.
\vs p180 3:2 \textcolour{ubdarkred}{“Many of those who will assail you are ignorant of the light of heaven, but this is not true of some who now persecute us. If we had not taught them the truth, they might do many strange things without falling under condemnation, but now, since they have known the light and presumed to reject it, they have no excuse for their attitude. He who hates me hates my Father. It cannot be otherwise; the light which would save you if accepted can only condemn you if it is knowingly rejected. And what have I done to these men that they should hate me with such a terrible hatred? Nothing, save to offer them fellowship on earth and salvation in heaven. But have you not read in the Scripture the saying: ‘And they hated me without a cause’?}
\vs p180 3:3 \textcolour{ubdarkred}{“But I will not leave you alone in the world. Very soon, after I have gone, I will send you a spirit helper. You shall have with you one who will take my place among you, one who will continue to teach you the way of truth, who will even comfort you.}
\vs p180 3:4 “Let not your hearts be troubled. You believe in God; continue to believe also in me. Even though I must leave you, I will not be far from you. I have already told you that in my Father’s universe there are many tarrying\hyp{}places. If this were not true, I would not have repeatedly told you about them. I am going to return to these worlds of light, stations in the Father’s heaven to which you shall sometime ascend. From these places I came into this world, and the hour is now at hand when I must return to my Father’s work in the spheres on high.
\vs p180 3:5 \textcolour{ubdarkred}{“If I thus go before you into the Father’s heavenly kingdom, so will I surely send for you that you may be with me in the places that were prepared for the mortal sons of God before this world was. Even though I must leave you, I will be present with you in spirit, and eventually you shall be with me in person when you have ascended to me in my universe even as I am about to ascend to my Father in his greater universe. And what I have told you is true and everlasting, even though you may not fully comprehend it. I go to the Father, and though you cannot now follow me, you shall certainly follow me in the ages to come.”}
\vs p180 3:6 When Jesus sat down, Thomas arose and said: “Master, we do not know where you are going; so of course we do not know the way. But we will follow you this very night if you will show us the way.”
\vs p180 3:7 When Jesus heard Thomas, he answered:\tunemarkup{pgauraone}{\linebreak} \textcolour{ubdarkred}{“Thomas, I am the way, the truth, and the life. No man goes to the Father except through me. All who find the Father, first find me. If you know me, you know the way to the Father. And you do know me, for you have lived with me and you now see me.”}
\vs p180 3:8 But this teaching was too deep for many of the apostles, especially for Philip, who, after speaking a few words with Nathaniel, arose and said: “Master, show us the Father, and everything you have said will be made plain.”
\vs p180 3:9 And when Philip had spoken, Jesus said: \textcolour{ubdarkred}{“Philip, have I been so long with you and yet you do not even now know me? Again do I declare: He who has seen me has seen the Father. How can you then say, Show us the Father? Do you not believe that I am in the Father and the Father in me? Have I not taught you that the words which I speak are not my words but the words of the Father? I speak for the Father and not of myself. I am in this world to do the Father’s will, and that I have done. My Father abides in me and works through me. Believe me when I say that the Father is in me, and that I am in the Father, or else believe me for the sake of the very life I have lived --- for the work’s sake.”}
\vs p180 3:10 As the Master went aside to refresh himself with water, the eleven engaged in a spirited discussion of these teachings, and Peter was beginning to deliver himself of an extended speech when Jesus returned and beckoned them to be seated.
\usection{The Promised Helper}
\vs p180 4:1 Jesus continued to teach, saying: \textcolour{ubdarkred}{“When I have gone to the Father, and after he has fully accepted the work I have done for you on earth, and after I have received the final sovereignty of my own domain, I shall say to my Father: Having left my children alone on earth, it is in accordance with my promise to send them another teacher. And when the Father shall approve, I will pour out the Spirit of Truth upon all flesh. Already is my Father’s spirit in your hearts, and when this day shall come, you will also have me with you even as you now have the Father. This new gift is the spirit of living truth. The unbelievers will not at first listen to the teachings of this spirit, but the sons of light will all receive him gladly and with a whole heart. And you shall know this spirit when he comes even as you have known me, and you will receive this gift in your hearts, and he will abide with you. You thus perceive that I am not going to leave you without help and guidance. I will not leave you desolate. Today I can be with you only in person. In the times to come I will be with you and all other men who desire my presence, wherever you may be, and with each of you at the same time. Do you not discern that it is better for me to go away; that I leave you in the flesh so that I may the better and the more fully be with you in the spirit?}
\vs p180 4:2 \textcolour{ubdarkred}{“In just a few hours the world will see me no more; but you will continue to know me in your hearts even until I send you this new teacher, the Spirit of Truth. As I have lived with you in person, then shall I live in you; I shall be one with your personal experience in the spirit kingdom. And when this has come to pass, you shall surely know that I am in the Father, and that, while your life is hid with the Father in me, I am also in you. I have loved the Father and have kept his word; you have loved me, and you will keep my word. As my Father has given me of his spirit, so will I give you of my spirit. And this Spirit of Truth which I will bestow upon you shall guide and comfort you and shall eventually lead you into all truth.}
\vs p180 4:3 \textcolour{ubdarkred}{“I am telling you these things while I am still with you that you may be the better prepared to endure those trials which are even now right upon us. And when this new day comes, you will be indwelt by the Son as well as by the Father. And these gifts of heaven will ever work the one with the other even as the Father and I have wrought on earth and before your very eyes as one person, the Son of Man. And this spirit friend will bring to your remembrance everything I have taught you.”}
\vs p180 4:4 As the Master paused for a moment, Judas Alpheus made bold to ask one of the few questions which either he or his brother ever addressed to Jesus in public. Said Judas: “Master, you have always lived among us as a friend; how shall we know you when you no longer manifest yourself to us save by this spirit? If the world sees you not, how shall we be certain about you? How will you show yourself to us?”
\vs p180 4:5 Jesus looked down upon them all, smiled, and said: \textcolour{ubdarkred}{“My little children, I am going away, going back to my Father. In a little while you will not see me as you do here, as flesh and blood. In a very short time I am going to send you my spirit, just like me except for this material body. This new teacher is the Spirit of Truth who will live with each one of you, in your hearts, and so will all the children of light be made one and be drawn toward one another. And in this very manner will my Father and I be able to live in the souls of each one of you and also in the hearts of all other men who love us and make that love real in their experiences by loving one another, even as I am now loving you.”}
\vs p180 4:6 Judas Alpheus did not fully understand what the Master said, but he grasped the promise of the new teacher, and from the expression on Andrew’s face, he perceived that his question had been satisfactorily answered.
\usection{The Spirit of Truth}
\vs p180 5:1 The new helper which Jesus promised to send into the hearts of believers, to pour out upon all flesh, is the \bibemph{Spirit of Truth.} This divine endowment is not the letter or law of truth, neither is it to function as the form or expression of truth. The new teacher is the \bibemph{conviction of truth,} the consciousness and assurance of true meanings on real spirit levels. And this new teacher is the spirit of living and growing truth, expanding, unfolding, and adaptative truth.
\vs p180 5:2 Divine truth is a spirit\hyp{}discerned and living reality. Truth exists only on high spiritual levels of the realization of divinity and the consciousness of communion with God. You can know the truth, and you can live the truth; you can experience the growth of truth in the soul and enjoy the liberty of its enlightenment in the mind, but you cannot imprison truth in formulas, codes, creeds, or intellectual patterns of human conduct. When you undertake the human formulation of divine truth, it speedily dies. The post\hyp{}mortem salvage of imprisoned truth, even at best, can eventuate only in the realization of a peculiar form of intellectualized glorified wisdom. Static truth is dead truth, and only dead truth can be held as a theory. Living truth is dynamic and can enjoy only an experiential existence in the human mind.
\vs p180 5:3 Intelligence grows out of a material existence which is illuminated by the presence of the cosmic mind. Wisdom comprises the consciousness of knowledge elevated to new levels of meaning and activated by the presence of the universe endowment of the adjutant of wisdom. Truth is a spiritual reality value experienced only by spirit\hyp{}endowed beings who function upon supermaterial levels of universe consciousness, and who, after the realization of truth, permit its spirit of activation to live and reign within their souls.
\vs p180 5:4 The true child of universe insight looks for the living Spirit of Truth in every wise saying. The God\hyp{}knowing individual is constantly elevating wisdom to the living\hyp{}truth levels of divine attainment; the spiritually unprogressive soul is all the while dragging the living truth down to the dead levels of wisdom and to the domain of mere exalted knowledge.
\vs p180 5:5 The golden rule, when divested of the superhuman insight of the Spirit of Truth, becomes nothing more than a rule of high ethical conduct. The golden rule, when literally interpreted, may become the instrument of great offence to one’s fellows. Without a spiritual discernment of the golden rule of wisdom you might reason that, since you are desirous that all men speak the full and frank truth of their minds to you, you should therefore fully and frankly speak the full thought of your mind to your fellow beings. Such an unspiritual interpretation of the golden rule might result in untold unhappiness and no end of sorrow.
\vs p180 5:6 Some persons discern and interpret the golden rule as a purely intellectual affirmation of human fraternity. Others experience this expression of human relationship as an emotional gratification of the tender feelings of the human personality. Another mortal recognizes this same golden rule as the yardstick for measuring all social relations, the standard of social conduct. Still others look upon it as being the positive injunction of a great moral teacher who embodied in this statement the highest concept of moral obligation as regards all fraternal relationships. In the lives of such moral beings the golden rule becomes the wise centre and circumference of all their philosophy.
\vs p180 5:7 In the kingdom of the believing brotherhood of God\hyp{}knowing truth lovers, this golden rule takes on living qualities of spiritual realization on those higher levels of interpretation which cause the mortal sons of God to view this injunction of the Master as requiring them so to relate themselves to their fellows that they will receive the highest possible good as a result of the believer’s contact with them. This is the essence of true religion: that you love your neighbour as yourself.
\vs p180 5:8 But the highest realization and the truest interpretation of the golden rule consists in the consciousness of the spirit of the truth of the enduring and living reality of such a divine declaration. The true cosmic meaning of this rule of universal relationship is revealed only in its spiritual realization, in the interpretation of the law of conduct by the spirit of the Son to the spirit of the Father that indwells the soul of mortal man. And when such spirit\hyp{}led mortals realize the true meaning of this golden rule, they are filled to overflowing with the assurance of citizenship in a friendly universe, and their ideals of spirit reality are satisfied only when they love their fellows as Jesus loved us all, and that is the reality of the realization of the love of God.
\vs p180 5:9 This same philosophy of the living flexibility and cosmic adaptability of divine truth to the individual requirements and capacity of every son of God, must be perceived before you can hope adequately to understand the Master’s teaching and practice of nonresistance to evil. The Master’s teaching is basically a spiritual pronouncement. Even the material implications of his philosophy cannot be helpfully considered apart from their spiritual correlations. The spirit of the Master’s injunction consists in the nonresistance of all selfish reaction to the universe, coupled with the aggressive and progressive attainment of righteous levels of true spirit values: divine beauty, infinite goodness, and eternal truth --- to know God and to become increasingly like him.
\vs p180 5:10 Love, unselfishness, must undergo a constant and living readaptative interpretation of relationships in accordance with the leading of the Spirit of Truth. Love must thereby grasp the ever\hyp{}changing and enlarging concepts of the highest cosmic good of the individual who is loved. And then love goes on to strike this same attitude concerning all other individuals who could possibly be influenced by the growing and living relationship of one spirit\hyp{}led mortal’s love for other citizens of the universe. And this entire living adaptation of love must be effected in the light of both the environment of present evil and the eternal goal of the perfection of divine destiny.
\vs p180 5:11 And so must we clearly recognize that neither the golden rule nor the teaching of nonresistance can ever be properly understood as dogmas or precepts. They can only be comprehended by living them, by realizing their meanings in the living interpretation of the Spirit of Truth, who directs the loving contact of one human being with another.
\vs p180 5:12 And all this clearly indicates the difference between the old religion and the new. The old religion taught self\hyp{}sacrifice; the new religion teaches only self\hyp{}forgetfulness, enhanced self\hyp{}realization in conjoined social service and universe comprehension. The old religion was motivated by fear\hyp{}consciousness; the new gospel of the kingdom is dominated by truth\hyp{}conviction, the spirit of eternal and universal truth. And no amount of piety or creedal loyalty can compensate for the absence in the life experience of kingdom believers of that spontaneous, generous, and sincere friendliness which characterizes the spirit\hyp{}born sons of the living God. Neither tradition nor a ceremonial system of formal worship can atone for the lack of genuine compassion for one’s fellows.
\usection{The Necessity for Leaving}
\vs p180 6:1 After Peter, James, John, and Matthew had asked the Master numerous questions, he continued his farewell discourse by saying: \textcolour{ubdarkred}{“And I am telling you about all this before I leave you in order that you may be so prepared for what is coming upon you that you will not stumble into serious error. The authorities will not be content with merely putting you out of the synagogues; I warn you the hour draws near when they who kill you will think they are doing a service to God. And all of these things they will do to you and to those whom you lead into the kingdom of heaven because they do not know the Father. They have refused to know the Father by refusing to receive me; and they refuse to receive me when they reject you, provided you have kept my new commandment that you love one another even as I have loved you. I am telling you in advance about these things so that, when your hour comes, as mine now has, you may be strengthened in the knowledge that all was known to me, and that my spirit shall be with you in all your sufferings for my sake and the gospel’s. It was for this purpose that I have been talking so plainly to you from the very beginning. I have even warned you that a man’s foes may be those of his own household. Although this gospel of the kingdom never fails to bring great peace to the soul of the individual believer, it will not bring peace on earth until man is willing to believe my teaching wholeheartedly and to establish the practice of doing the Father’s will as the chief purpose in living the mortal life.}
\vs p180 6:2 \textcolour{ubdarkred}{“Now that I am leaving you, seeing that the hour has come when I am about to go to the Father, I am surprised that none of you have asked me, Why do you leave us? Nevertheless, I know that you ask such questions in your hearts. I will speak to you plainly, as one friend to another. It is really profitable for you that I go away. If I go not away, the new teacher cannot come into your hearts. I must be divested of this mortal body and be restored to my place on high before I can send this spirit teacher to live in your souls and lead your spirits into the truth. And when my spirit comes to indwell you, he will illuminate the difference between sin and righteousness and will enable you to judge wisely in your hearts concerning them.}
\vs p180 6:3 \textcolour{ubdarkred}{“I have yet much to say to you, but you cannot stand any more just now. Albeit, when he, the Spirit of Truth, comes, he shall eventually guide you into all truth as you pass through the many abodes in my Father’s universe.}
\vs p180 6:4 \textcolour{ubdarkred}{“This spirit will not speak of himself, but he will declare to you that which the Father has revealed to the Son, and he will even show you things to come; he will glorify me even as I have glorified my Father. This spirit comes forth from me, and he will reveal my truth to you. Everything which the Father has in this domain is now mine; wherefore did I say that this new teacher would take of that which is mine and reveal it to you.}
\vs p180 6:5 \textcolour{ubdarkred}{“In just a little while I will leave you for a short time. Afterwards, when you again see me, I shall already be on my way to the Father so that even then you will not see me for long.”}
\vs p180 6:6 While he paused for a moment, the apostles began to talk with each other: “What is this that he tells us? ‘In just a little while I will leave you,’ and ‘When you see me again it will not be for long, for I will be on my way to the Father.’ What can he mean by this ‘little while’ and ‘not for long’? We cannot understand what he is telling us.”
\vs p180 6:7 And since Jesus knew they asked these questions, he said: \textcolour{ubdarkred}{“Do you inquire among yourselves about what I meant when I said that in a little while I would not be with you, and that, when you would see me again, I would be on my way to the Father? I have plainly told you that the Son of Man must die, but that he will rise again. Can you not then discern the meaning of my words? You will first be made sorrowful, but later on will you rejoice with many who will understand these things after they have come to pass. A woman is indeed sorrowful in the hour of her travail, but when she is once delivered of her child, she immediately forgets her anguish in the joy of the knowledge that a man has been born into the world. And so are you about to sorrow over my departure, but I will soon see you again, and then will your sorrow be turned into rejoicing, and there shall come to you a new revelation of the salvation of God which no man can ever take away from you. And all the worlds will be blessed in this same revelation of life in effecting the overthrow of death. Hitherto have you made all your requests in my Father’s name. After you see me again, you may also ask in my name, and I will hear you.}
\vs p180 6:8 \textcolour{ubdarkred}{“Down here I have taught you in proverbs and spoken to you in parables. I did so because you were only children in the spirit; but the time is coming when I will talk to you plainly concerning the Father and his kingdom. And I shall do this because the Father himself loves you and desires to be more fully revealed to you. Mortal man cannot see the spirit Father; therefore have I come into the world to show the Father to your creature eyes. But when you have become perfected in spirit growth, you shall then see the Father himself.”}
\vs p180 6:9 When the eleven had heard him speak, they said to each other: “Behold, he does speak plainly to us. Surely the Master did come forth from God. But why does he say he must return to the Father?” And Jesus saw that they did not even yet comprehend him. These eleven men could not get away from their long\hyp{}nourished ideas of the Jewish concept of the Messiah. The more fully they believed in Jesus as the Messiah, the more troublesome became these deep\hyp{}rooted notions regarding the glorious material triumph of the kingdom on earth.
\quizlink
