\upaper{2}{The Nature of God}
\uminitoc{The Infinity of God}
\uminitoc{The Father’s Eternal Perfection}
\uminitoc{Justice and Righteousness}
\uminitoc{The Divine Mercy}
\uminitoc{The Love of God}
\uminitoc{The Goodness of God}
\uminitoc{Divine Truth and Beauty}
\author{Divine Counsellor}
\vs p002 0:1 Inasmuch as man’s highest possible concept of God is embraced within the human idea and ideal of a primal and infinite personality, it is permissible, and may prove helpful, to study certain characteristics of the divine nature which constitute the character of Deity. The nature of God can best be understood by the revelation of the Father which Michael of Nebadon unfolded in his manifold teachings and in his superb mortal life in the flesh. The divine nature can also be better understood by man if he regards himself as a child of God and looks up to the Paradise Creator as a true spiritual Father.
\vs p002 0:2 The nature of God can be studied in a revelation of supreme ideas, the divine character can be envisaged as a portrayal of supernal ideals, but the most enlightening and spiritually edifying of all revelations of the divine nature is to be found in the comprehension of the religious life of Jesus of Nazareth, both before and after his attainment of full consciousness of divinity. If the incarnated life of Michael is taken as the background of the revelation of God to man, we may attempt to put in human word symbols certain ideas and ideals concerning the divine nature which may possibly contribute to a further illumination and unification of the human concept of the nature and the character of the personality of the Universal Father.
\vs p002 0:3 In all our efforts to enlarge and spiritualize the human concept of God, we are tremendously handicapped by the limited capacity of the mortal mind. We are also seriously handicapped in the execution of our assignment by the limitations of language and by the poverty of material which can be utilized for purposes of illustration or comparison in our efforts to portray divine values and to present spiritual meanings to the finite, mortal mind of man. All our efforts to enlarge the human concept of God would be well\hyp{}nigh futile except for the fact that the mortal mind is indwelt by the bestowed Adjuster of the Universal Father and is pervaded by the Truth Spirit of the Creator Son. Depending, therefore, on the presence of these divine spirits within the heart of man for assistance in the enlargement of the concept of God, I cheerfully undertake the execution of my mandate to attempt the further portrayal of the nature of God to the mind of man.
\usection{The Infinity of God}
\vs p002 1:1 “Touching the Infinite\fnst{Here, as well as in all 30 occurrences of ``touching'' in the KJV, where this quote comes from, viz. Job~37:23, is of course used as a preposition meaning ``regarding'', ``concerning''. The same applies to the usage in \bibref[96:6.4]{p096 6:4}.}, we cannot find him out. The divine footsteps are not known.” “His understanding is infinite and his greatness is unsearchable.” The blinding light of the Father’s presence is such that to his lowly creatures he apparently “dwells in the thick darkness.” Not only are his thoughts and plans unsearchable, but “he does great and marvellous things without number.” “God is great; we comprehend him not, neither can the number of his years be searched out.” “Will God indeed dwell on the earth? Behold, the heaven (universe) and the heaven of heavens (universe of universes) cannot contain him.” “How unsearchable are his judgments and his ways past finding out!”
\vs p002 1:2 “There is but one God, the infinite Father, who is also a faithful Creator.” “The divine Creator is also the Universal Disposer, the source and destiny of souls. He is the Supreme Soul, the Primal Mind, and the Unlimited Spirit of all creation.” “The great Controller makes no mistakes. He is resplendent in majesty and glory.” “The Creator God is wholly devoid of fear and enmity. He is immortal, eternal, self\hyp{}existent, divine, and bountiful.” “How pure and beautiful, how deep and unfathomable is the supernal Ancestor of all things!” “The Infinite is most excellent in that he imparts himself to men. He is the beginning and the end, the Father of every good and perfect purpose.” “With God all things are possible; the eternal Creator is the cause of causes.”
\vs p002 1:3 \pc Notwithstanding the infinity of the stupendous manifestations of the Father’s eternal and universal personality, he is unqualifiedly self\hyp{}conscious of both his infinity and eternity; likewise he knows fully his perfection and power. He is the only being in the universe, aside from his divine co\hyp{}ordinates, who experiences a perfect, proper, and complete appraisal of himself.
\vs p002 1:4 The Father constantly and unfailingly meets the need of the differential of demand for himself as it changes from time to time in various sections of his master universe. The great God knows and understands himself; he is infinitely self\hyp{}conscious of all his primal attributes of perfection. God is not a cosmic accident; neither is he a universe experimenter. The Universe Sovereigns may engage in adventure; the Constellation Fathers may experiment; the system heads may practise; but the Universal Father sees the end from the beginning, and his divine plan and eternal purpose actually embrace and comprehend all the experiments and all the adventures of all his subordinates in every world, system, and constellation in every universe of his vast domains.
\vs p002 1:5 No thing is new to God, and no cosmic event ever comes as a surprise; he inhabits the circle of eternity. He is without beginning or end of days. To God there is no past, present, or future; all time is present at any given moment. He is the great and only I AM.
\vs p002 1:6 \pc The Universal Father is absolutely and without qualification infinite in all his attributes; and this fact, in and of itself, automatically shuts him off from all direct personal communication with finite material beings and other lowly created intelligences.
\vs p002 1:7 And all this necessitates such arrangements for contact and communication with his manifold creatures as have been ordained, first, in the personalities of the Paradise Sons of God, who, although perfect in divinity, also often partake of the nature of the very flesh and blood of the planetary races, becoming one of you and one with you; thus, as it were, God becomes man, as occurred in the bestowal of Michael, who was called interchangeably the Son of God and the Son of Man. And second, there are the personalities of the Infinite Spirit, the various orders of the seraphic hosts and other celestial intelligences who draw near to the material beings of lowly origin and in so many ways minister to them and serve them. And third, there are the impersonal Mystery Monitors, Thought Adjusters, the actual gift of the great God himself sent to indwell such as the humans of Urantia, sent without announcement and without explanation. In endless profusion they descend from the heights of glory to grace and indwell the humble minds of those mortals who possess the capacity for God\hyp{}consciousness or the potential therefor.
\vs p002 1:8 In these ways and in many others, in ways unknown to you and utterly beyond finite comprehension, does the Paradise Father lovingly and willingly downstep and otherwise modify, dilute, and attenuate his infinity in order that he may be able to draw nearer the finite minds of his creature children. And so, through a series of personality distributions which are diminishingly absolute, the infinite Father is enabled to enjoy close contact with the diverse intelligences of the many realms of his far\hyp{}flung universe.
\vs p002 1:9 All this he has done and now does, and evermore will continue to do, without in the least detracting from the fact and reality of his infinity, eternity, and primacy. And these things are absolutely true, notwithstanding the difficulty of their comprehension, the mystery in which they are enshrouded, or the impossibility of their being fully understood by creatures such as dwell on Urantia.
\vs p002 1:10 \pc Because the First Father is infinite in his plans and eternal in his purposes, it is inherently impossible for any finite being ever to grasp or comprehend these divine plans and purposes in their fullness. Mortal man can glimpse the Father’s purposes only now and then, here and there, as they are revealed in relation to the outworking of the plan of creature ascension on its successive levels of universe progression. Though man cannot encompass the significance of infinity, the infinite Father does most certainly fully comprehend and lovingly embrace all the finity of all his children in all universes.
\vs p002 1:11 Divinity and eternity the Father shares with large numbers of the higher Paradise beings, but we question whether infinity and consequent universal primacy is fully shared with any save his co\hyp{}ordinate associates of the Paradise Trinity. Infinity of personality must, perforce, embrace all finitude of personality; hence the truth --- literal truth --- of the teaching which declares that “In Him we live and move and have our being.” That fragment of the pure Deity of the Universal Father which indwells mortal man \bibemph{is} a part of the infinity of the First Great Source and Centre, the Father of Fathers.
\usection{The Father’s Eternal Perfection}
\vs p002 2:1 Even your olden prophets understood the eternal, never\hyp{}beginning, never\hyp{}ending, circular nature of the Universal Father. God is literally and eternally present in his universe of universes. He inhabits the present moment with all his absolute majesty and eternal greatness. “The Father has life in himself, and this life is eternal life.” Throughout the eternal ages it has been the Father who “gives to all life.” There is infinite perfection in the divine integrity. “I am the Lord; I change not.” Our knowledge of the universe of universes discloses not only that he is the Father of lights, but also that in his conduct of interplanetary affairs there “is no variableness neither shadow of changing.” He “declares the end from the beginning.” He says: “My counsel shall stand; I will do all my pleasures” “according to the eternal purpose which I purposed in my Son.” Thus are the plans and purposes of the First Source and Centre like himself: eternal, perfect, and forever changeless.
\vs p002 2:2 There is finality of completeness and perfection of repleteness in the mandates of the Father. “Whatsoever God does, it shall be forever; nothing can be added to it nor anything taken from it.” The Universal Father does not repent of his original purposes of wisdom and perfection. His plans are steadfast, his counsel immutable, while his acts are divine and infallible. “A thousand years in his sight are but as yesterday when it is past and as a watch in the night.” The perfection of divinity and the magnitude of eternity are forever beyond the full grasp of the circumscribed mind of mortal man.
\vs p002 2:3 \pc The reactions of a changeless God, in the execution of his eternal purpose, may seem to vary in accordance with the changing attitude and the shifting minds of his created intelligences; that is, they may apparently and superficially vary; but underneath the surface and beneath all outward manifestations, there is still present the changeless purpose, the everlasting plan, of the eternal God.
\vs p002 2:4 Out in the universes, perfection must necessarily be a relative term, but in the central universe and especially on Paradise, perfection is undiluted; in certain phases it is even absolute. Trinity manifestations vary the exhibition of the divine perfection but do not attenuate it.
\vs p002 2:5 \pc God’s primal perfection consists not in an assumed righteousness but rather in the inherent perfection of the goodness of his divine nature. He is final, complete, and perfect. There is no thing lacking in the beauty and perfection of his righteous character. And the whole scheme of living existences on the worlds of space is centred in the divine purpose of elevating all will creatures to the high destiny of the experience of sharing the Father’s Paradise perfection. God is neither self\hyp{}centred nor self\hyp{}contained; he never ceases to bestow himself upon all self\hyp{}conscious creatures of the vast universe of universes.
\vs p002 2:6 God is eternally and infinitely perfect, he cannot personally know imperfection as his own experience, but he does share the consciousness of all the experience of imperfectness of all the struggling creatures of the evolutionary universes of all the Paradise Creator Sons. The personal and liberating touch of the God of perfection overshadows the hearts and encircuits the natures of all those mortal creatures who have ascended to the universe level of moral discernment. In this manner, as well as through the contacts of the divine presence, the Universal Father actually participates in the experience \bibemph{with} immaturity and imperfection in the evolving career of every moral being of the entire universe.
\vs p002 2:7 Human limitations, potential evil, are not a part of the divine nature, but mortal experience \bibemph{with} evil and all man’s relations thereto are most certainly a part of God’s ever\hyp{}expanding self\hyp{}realization in the children of time --- creatures of moral responsibility who have been created or evolved by every Creator Son going out from Paradise.
\usection{Justice and Righteousness}
\vs p002 3:1 God is righteous; therefore is he just. “The Lord is righteous in all his ways.” “‘I have not done without cause all that I have done,’ says the Lord.” “The judgments of the Lord are true and righteous altogether.” The justice of the Universal Father cannot be influenced by the acts and performances of his creatures, “for there is no iniquity with the Lord our God, no respect of persons, no taking of gifts.”
\vs p002 3:2 \pc How futile to make puerile appeals to such a God to modify his changeless decrees so that we can avoid the just consequences of the operation of his wise natural laws and righteous spiritual mandates! “Be not deceived; God is not mocked, for whatsoever a man sows that shall he also reap.” True, even in the justice of reaping the harvest of wrongdoing, this divine justice is always tempered with mercy. Infinite wisdom is the eternal arbiter which determines the proportions of justice and mercy which shall be meted out in any given circumstance. The greatest punishment (in reality an inevitable consequence) for wrongdoing and deliberate rebellion against the government of God is loss of existence as an individual subject of that government. The final result of wholehearted sin is annihilation. In the last analysis, such sin\hyp{}identified individuals have destroyed themselves by becoming wholly unreal through their embrace of iniquity. The factual disappearance of such a creature is, however, always delayed until the ordained order of justice current in that universe has been fully complied with.
\vs p002 3:3 Cessation of existence is usually decreed at the dispensational or epochal adjudication of the realm or realms. On a world such as Urantia it comes at the end of a planetary dispensation. Cessation of existence can be decreed at such times by co\hyp{}ordinate action of all tribunals of jurisdiction, extending from the planetary council up through the courts of the Creator Son to the judgment tribunals of the Ancients of Days. The mandate of dissolution originates in the higher courts of the superuniverse following an unbroken confirmation of the indictment originating on the sphere of the wrongdoer’s residence; and then, when sentence of extinction has been confirmed on high, the execution is by the direct act of those judges residential on, and operating from, the headquarters of the superuniverse.
\vs p002 3:4 When this sentence is finally confirmed, the sin\hyp{}identified being instantly becomes as though he had not been. There is no resurrection from such a fate; it is everlasting and eternal. The living energy factors of identity are resolved by the transformations of time and the metamorphoses of space into the cosmic potentials whence they once emerged. As for the personality of the iniquitous one, it is deprived of a continuing life vehicle by the creature’s failure to make those choices and final decisions which would have assured eternal life. When the continued embrace of sin by the associated mind culminates in complete self\hyp{}identification with iniquity, then upon the cessation of life, upon cosmic dissolution, such an isolated personality is absorbed into the oversoul of creation, becoming a part of the evolving experience of the Supreme Being. Never again does it appear as a personality; its identity becomes as though it had never been. In the case of an Adjuster\hyp{}indwelt personality, the experiential spirit values survive in the reality of the continuing Adjuster.
\vs p002 3:5 \pc In any universe contest between actual levels of reality, the personality of the higher level will ultimately triumph over the personality of the lower level. This inevitable outcome of universe controversy is inherent in the fact that divinity of quality equals the degree of reality or actuality of any will creature. Undiluted evil, complete error, wilful sin, and unmitigated iniquity are inherently and automatically suicidal. Such attitudes of cosmic unreality can survive in the universe only because of transient mercy\hyp{}tolerance pending the action of the justice\hyp{}determining and fairness\hyp{}finding mechanisms of the universe tribunals of righteous adjudication.
\vs p002 3:6 The rule of the Creator Sons in the local universes is one of creation and spiritualization. These Sons devote themselves to the effective execution of the Paradise plan of progressive mortal ascension, to the rehabilitation of rebels and wrong thinkers, but when all such loving efforts are finally and forever rejected, the final decree of dissolution is executed by forces acting under the jurisdiction of the Ancients of Days.
\usection{The Divine Mercy}
\vs p002 4:1 Mercy is simply justice tempered by that wisdom which grows out of perfection of knowledge and the full recognition of the natural weaknesses and environmental handicaps of finite creatures. “Our God is full of compassion, gracious, long\hyp{}suffering, and plenteous in mercy.” Therefore “whosoever calls upon the Lord shall be saved,” “for he will abundantly pardon.” “The mercy of the Lord is from everlasting to everlasting”; yes, “his mercy endures forever.” “I am the Lord who executes loving\hyp{}kindness, judgment, and righteousness in the earth, for in these things I delight.” “I do not afflict willingly nor grieve the children of men,” for I am “the Father of mercies and the God of all comfort.”
\vs p002 4:2 God is inherently kind, naturally compassionate, and everlastingly merciful. And never is it necessary that any influence be brought to bear upon the Father to call forth his loving\hyp{}kindness. The creature’s need is wholly sufficient to ensure the full flow of the Father’s tender mercies and his saving grace. Since God knows all about his children, it is easy for him to forgive. The better man understands his neighbour, the easier it will be to forgive him, even to love him.
\vs p002 4:3 \pc Only the discernment of infinite wisdom enables a righteous God to minister justice and mercy at the same time and in any given universe situation. The heavenly Father is never torn by conflicting attitudes towards his universe children; God is never a victim of attitudinal antagonisms. God’s all\hyp{}knowingness unfailingly directs his free will in the choosing of that universe conduct which perfectly, simultaneously, and equally satisfies the demands of all his divine attributes and the infinite qualities of his eternal nature.
\vs p002 4:4 Mercy is the natural and inevitable offspring of goodness and love. The good nature of a loving Father could not possibly withhold the wise ministry of mercy to each member of every group of his universe children. Eternal justice and divine mercy together constitute what in human experience would be called \bibemph{fairness.}
\vs p002 4:5 Divine mercy represents a fairness technique of adjustment between the universe levels of perfection and imperfection. Mercy is the justice of Supremacy adapted to the situations of the evolving finite, the righteousness of eternity modified to meet the highest interests and universe welfare of the children of time. Mercy is not a contravention of justice but rather an understanding interpretation of the demands of supreme justice as it is fairly applied to the subordinate spiritual beings and to the material creatures of the evolving universes. Mercy is the justice of the Paradise Trinity wisely and lovingly visited upon the manifold intelligences of the creations of time and space as it is formulated by divine wisdom and determined by the all\hyp{}knowing mind and the sovereign free will of the Universal Father and all his associated Creators.
\usection{The Love of God}
\vs p002 5:1 “God is love”; therefore his only personal attitude towards the affairs of the universe is always a reaction of divine affection. The Father loves us sufficiently to bestow his life upon us. “He makes his sun to rise on the evil and on the good and sends rain on the just and on the unjust.”
\vs p002 5:2 \pc It is wrong to think of God as being coaxed into loving his children because of the sacrifices of his Sons or the intercession of his subordinate creatures, “for the Father himself loves you.” It is in response to this paternal affection that God sends the marvellous Adjusters to indwell the minds of men. God’s love is universal; “whosoever will may come.” He would “have all men be saved by coming into the knowledge of the truth.” He is “not willing that any should perish.”
\vs p002 5:3 The Creators are the very first to attempt to save man from the disastrous results of his foolish transgression of the divine laws. God’s love is by nature a fatherly affection; therefore does he sometimes “chasten us for our own profit, that we may be partakers of his holiness.” Even during your fiery trials remember that “in all our afflictions he is afflicted with us.”
\vs p002 5:4 God is divinely kind to sinners. When rebels return to righteousness, they are mercifully received, “for our God will abundantly pardon.” “I am he who blots out your transgressions for my own sake, and I will not remember your sins.” “Behold what manner of love the Father has bestowed upon us that we should be called the sons of God.”
\vs p002 5:5 After all, the greatest evidence of the goodness of God and the supreme reason for loving him is the indwelling gift of the Father --- the Adjuster who so patiently awaits the hour when you both shall be eternally made one. Though you cannot find God by searching, if you will submit to the leading of the indwelling spirit, you will be unerringly guided, step by step, life by life, through universe upon universe, and age by age, until you finally stand in the presence of the Paradise personality of the Universal Father.
\vs p002 5:6 \pc How unreasonable that you should not worship God because the limitations of human nature and the handicaps of your material creation make it impossible for you to see him. Between you and God there is a tremendous distance (physical space) to be traversed. There likewise exists a great gulf of spiritual differential which must be bridged; but notwithstanding all that physically and spiritually separates you from the Paradise personal presence of God, stop and ponder the solemn fact that God lives within you; he has in his own way already bridged the gulf. He has sent of himself, his spirit, to live in you and to toil with you as you pursue your eternal universe career.
\vs p002 5:7 I find it easy and pleasant to worship one who is so great and at the same time so affectionately devoted to the uplifting ministry of his lowly creatures. I naturally love one who is so powerful in creation and in the control thereof, and yet who is so perfect in goodness and so faithful in the loving\hyp{}kindness which constantly overshadows us. I think I would love God just as much if he were not so great and powerful, as long as he is so good and merciful. We all love the Father more because of his nature than in recognition of his amazing attributes.
\vs p002 5:8 When I observe the Creator Sons and their subordinate administrators struggling so valiantly with the manifold difficulties of time inherent in the evolution of the universes of space, I discover that I bear these lesser rulers of the universes a great and profound affection. After all, I think we all, including the mortals of the realms, love the Universal Father and all other beings, divine or human, because we discern that these personalities truly love us. The experience of loving is very much a direct response to the experience of being loved. Knowing that God loves me, I should continue to love him supremely, even though he were divested of all his attributes of supremacy, ultimacy, and absoluteness.
\vs p002 5:9 The Father’s love follows us now and throughout the endless circle of the eternal ages. As you ponder the loving nature of God, there is only one reasonable and natural personality reaction thereto: You will increasingly love your Maker; you will yield to God an affection analogous to that given by a child to an earthly parent; for, as a father, a real father, a true father, loves his children, so the Universal Father loves and forever seeks the welfare of his created sons and daughters.
\vs p002 5:10 But the love of God is an intelligent and farseeing parental affection. The divine love functions in unified association with divine wisdom and all other infinite characteristics of the perfect nature of the Universal Father. God is love, but love is not God. The greatest manifestation of the divine love for mortal beings is observed in the bestowal of the Thought Adjusters, but your greatest revelation of the Father’s love is seen in the bestowal life of his Son Michael as he lived on earth the ideal spiritual life. It is the indwelling Adjuster who individualizes the love of God to each human soul.
\vs p002 5:11 \pc At times I am almost pained to be compelled to portray the divine affection of the heavenly Father for his universe children by the employment of the human word symbol \bibemph{love.} This term, even though it does connote man’s highest concept of the mortal relations of respect and devotion, is so frequently designative of so much of human relationship that is wholly ignoble and utterly unfit to be known by any word which is also used to indicate the matchless affection of the living God for his universe creatures! How unfortunate that I cannot make use of some supernal and exclusive term which would convey to the mind of man the true nature and exquisitely beautiful significance of the divine affection of the Paradise Father.
\vs p002 5:12 \pc When man loses sight of the love of a personal God, the kingdom of God becomes merely the kingdom of good. Notwithstanding the infinite unity of the divine nature, love is the dominant characteristic of all God’s personal dealings with his creatures.
\usection{The Goodness of God}
\vs p002 6:1 In the physical universe we may see the divine beauty, in the intellectual world we may discern eternal truth, but the goodness of God is found only in the spiritual world of personal religious experience. In its true essence, religion is a faith\hyp{}trust in the goodness of God. God could be great and absolute, somehow even intelligent and personal, in philosophy, but in religion God must also be moral; he must be good. Man might fear a great God, but he trusts and loves only a good God. This goodness of God is a part of the personality of God, and its full revelation appears only in the personal religious experience of the believing sons of God.
\vs p002 6:2 Religion implies that the superworld of spirit nature is cognizant of, and responsive to, the fundamental needs of the human world. Evolutionary religion may become ethical, but only revealed religion becomes truly and spiritually moral. The olden concept that God is a Deity dominated by kingly morality was upstepped by Jesus to that affectionately touching level of intimate family morality of the parent\hyp{}child relationship, than which there is none more tender and beautiful in mortal experience.
\vs p002 6:3 \pc The “richness of the goodness of God leads erring man to repentance.” “Every good gift and every perfect gift comes down from the Father of lights.” “God is good; he is the eternal refuge of the souls of men.” “The Lord God is merciful and gracious. He is long\hyp{}suffering and abundant in goodness and truth.” “Taste and see that the Lord is good! Blessed is the man who trusts him.” “The Lord is gracious and full of compassion. He is the God of salvation.” “He heals the brokenhearted and binds up the wounds of the soul. He is man’s all\hyp{}powerful benefactor.”
\vs p002 6:4 \pc The concept of God as a king\hyp{}judge, although it fostered a high moral standard and created a law\hyp{}respecting people as a group, left the individual believer in a sad position of insecurity respecting his status in time and in eternity. The later Hebrew prophets proclaimed God to be a Father to Israel; Jesus revealed God as the Father of each human being. The entire mortal concept of God is transcendently illuminated by the life of Jesus. Selflessness is inherent in parental love. God loves not \bibemph{like} a father, but \bibemph{as} a father. He is the Paradise Father of every universe personality.
\vs p002 6:5 \pc Righteousness implies that God is the source of the moral law of the universe. Truth exhibits God as a revealer, as a teacher. But love gives and craves affection, seeks understanding fellowship such as exists between parent and child. Righteousness may be the divine thought, but love is a father’s attitude. The erroneous supposition that the righteousness of God was irreconcilable with the selfless love of the heavenly Father, presupposed absence of unity in the nature of Deity and led directly to the elaboration of the atonement doctrine, which is a philosophic assault upon both the unity and the free\hyp{}willness of God.
\vs p002 6:6 The affectionate heavenly Father, whose spirit indwells his children on earth, is not a divided personality --- one of justice and one of mercy --- neither does it require a mediator to secure the Father’s favour or forgiveness. Divine righteousness is not dominated by strict retributive justice; God as a father transcends God as a judge.
\vs p002 6:7 \pc God is never wrathful, vengeful, or angry. It is true that wisdom does often restrain his love, while justice conditions his rejected mercy. His love of righteousness cannot help being exhibited as equal hatred for sin. The Father is not an inconsistent personality; the divine unity is perfect. In the Paradise Trinity there is absolute unity despite the eternal identities of the co\hyp{}ordinates of God.
\vs p002 6:8 \pc God loves the sinner and \bibemph{hates} the sin: such a statement is true philosophically, but God is a transcendent personality, and persons can only love and hate other persons. Sin is not a person. God loves the sinner because he is a personality reality (potentially eternal), while towards sin God strikes no personal attitude, for sin is not a spiritual reality; it is not personal; therefore does only the justice of God take cognizance of its existence. The love of God saves the sinner; the law of God destroys the sin. This attitude of the divine nature would apparently change if the sinner finally identified himself wholly with sin just as the same mortal mind may also fully identify itself with the indwelling spirit Adjuster. Such a sin\hyp{}identified mortal would then become wholly unspiritual in nature (and therefore personally unreal) and would experience eventual extinction of being. Unreality, even incompleteness of creature nature, cannot exist forever in a progressingly real and increasingly spiritual universe.
\vs p002 6:9 \pc Facing the world of personality, God is discovered to be a loving person; facing the spiritual world, he is a personal love; in religious experience he is both. Love identifies the volitional will of God. The goodness of God rests at the bottom of the divine free\hyp{}willness --- the universal tendency to love, show mercy, manifest patience, and minister forgiveness.
\usection{Divine Truth and Beauty}
\vs p002 7:1 All finite knowledge and creature understanding are \bibemph{relative.} Information and intelligence, gleaned from even high sources, is only relatively complete, locally accurate, and personally true.
\vs p002 7:2 Physical facts are fairly uniform, but truth is a living and flexible factor in the philosophy of the universe. Evolving personalities are only partially wise and relatively true in their communications. They can be certain only as far as their personal experience extends. That which apparently may be wholly true in one place may be only relatively true in another segment of creation.
\vs p002 7:3 Divine truth, final truth, is uniform and universal, but the story of things spiritual, as it is told by numerous individuals hailing from various spheres, may sometimes vary in details owing to this relativity in the completeness of knowledge and in the repleteness of personal experience as well as in the length and extent of that experience. While the laws and decrees, the thoughts and attitudes, of the First Great Source and Centre are eternally, infinitely, and universally true; at the same time, their application to, and adjustment for, every universe, system, world, and created intelligence, are in accordance with the plans and technique of the Creator Sons as they function in their respective universes, as well as in harmony with the local plans and procedures of the Infinite Spirit and of all other associated celestial personalities.
\vs p002 7:4 \pc The false science of materialism would sentence mortal man to become an outcast in the universe. Such partial knowledge is potentially evil; it is knowledge composed of both good and evil. Truth is beautiful because it is both replete and symmetrical. When man searches for truth, he pursues the divinely real.
\vs p002 7:5 Philosophers commit their gravest error when they are misled into the fallacy of abstraction, the practice of focusing the attention upon one aspect of reality and then of pronouncing such an isolated aspect to be the whole truth. The wise philosopher will always look for the creative design which is behind, and pre\hyp{}existent to, all universe phenomena. The creator thought invariably precedes creative action.
\vs p002 7:6 Intellectual self\hyp{}consciousness can discover the beauty of truth, its spiritual quality, not only by the philosophic consistency of its concepts, but more certainly and surely by the unerring response of the ever\hyp{}present Spirit of Truth. Happiness ensues from the recognition of truth because it can be \bibemph{acted out;} it can be lived. Disappointment and sorrow attend upon error because, not being a reality, it cannot be realized in experience. Divine truth is best known by its \bibemph{spiritual flavour.}
\vs p002 7:7 \pc The eternal quest is for unification, for divine coherence. The far\hyp{}flung physical universe coheres in the Isle of Paradise; the intellectual universe coheres in the God of mind, the Conjoint Actor; the spiritual universe is coherent in the personality of the Eternal Son. But the isolated mortal of time and space coheres in God the Father through the direct relationship between the indwelling Thought Adjuster and the Universal Father. Man’s Adjuster is a fragment of God and everlastingly seeks for divine unification; it coheres with, and in, the Paradise Deity of the First Source and Centre.
\vs p002 7:8 \pc The discernment of supreme beauty is the discovery and integration of reality: The discernment of the divine goodness in the eternal truth, that is ultimate beauty. Even the charm of human art consists in the harmony of its unity.
\vs p002 7:9 The great mistake of the Hebrew religion was its failure to associate the goodness of God with the factual truths of science and the appealing beauty of art. As civilization progressed, and since religion continued to pursue the same unwise course of overemphasizing the goodness of God to the relative exclusion of truth and neglect of beauty, there developed an increasing tendency for certain types of men to turn away from the abstract and dissociated concept of isolated goodness. The overstressed and isolated morality of modern religion, which fails to hold the devotion and loyalty of many XX century men, would rehabilitate itself if, in addition to its moral mandates, it would give equal consideration to the truths of science, philosophy, and spiritual experience, and to the beauties of the physical creation, the charm of intellectual art, and the grandeur of genuine character achievement.
\vs p002 7:10 The religious challenge of this age is to those farseeing and forward\hyp{}looking men and women of spiritual insight who will dare to construct a new and appealing philosophy of living out of the enlarged and exquisitely integrated modern concepts of cosmic truth, universe beauty, and divine goodness. Such a new and righteous vision of morality will attract all that is good in the mind of man and challenge that which is best in the human soul. Truth, beauty, and goodness are divine realities, and as man ascends the scale of spiritual living, these supreme qualities of the Eternal become increasingly co\hyp{}ordinated and unified in God, who is love.
\vs p002 7:11 \pc All truth --- material, philosophic, or spiritual --- is both beautiful and good. All real beauty --- material art or spiritual symmetry --- is both true and good. All genuine goodness --- whether personal morality, social equity, or divine ministry --- is equally true and beautiful. Health, sanity, and happiness are integrations of truth, beauty, and goodness as they are blended in human experience. Such levels of efficient living come about through the unification of energy systems, idea systems, and spirit systems.
\vs p002 7:12 Truth is coherent, beauty attractive, goodness stabilizing. And when these values of that which is real are co\hyp{}ordinated in personality experience, the result is a high order of love conditioned by wisdom and qualified by loyalty. The real purpose of all universe education is to effect the better co\hyp{}ordination of the isolated child of the worlds with the larger realities of his expanding experience. Reality is finite on the human level, infinite and eternal on the higher and divine levels.
\vsetoff
\vs p002 7:13 [Presented by a Divine Counsellor acting by authority of the Ancients of Days on Uversa.]
\quizlink
\tunemarkuptwo{nobiblio}{}{%
\begin{thebibliography}{100}
\bibitem{Knudson1}
Albert C. Knudson.
{``The Doctrine of God.''}
{\em New York: Abingdon-Cokesbury Press}, 1930.
\bibitem{Hume1}
Robert Ernest Hume, M.A., Ph.D.,
{``Treasure\hyp{}House of the Living Religions: Selections from Their Sacred Scriptures.''}
{\em New York: Charles Scribner's Sons}, 1932.
\bibitem{Overstreet1}
H.A. Overstreet
{``The Enduring Question: A Search for a Philosophy of Life.''}
{\em New York: W. Norton \&\ Company, Inc.}, 1931.
\end{thebibliography}
}
