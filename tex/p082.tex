\upaper{82}{The Evolution of Marriage}
\author{Chief of Seraphim}
\vs p082 0:1 Marriage --- mating --- grows out of bisexuality. Marriage is man’s reactional adjustment to such bisexuality, while the family life is the sum total resulting from all such evolutionary and adaptative adjustments. Marriage is enduring; it is not inherent in biologic evolution, but it is the basis of all social evolution and is therefore certain of continued existence in some form. Marriage has given mankind the home, and the home is the crowning glory of the whole long and arduous evolutionary struggle.
\vs p082 0:2 While religious, social, and educational institutions are all essential to the survival of cultural civilization, \bibemph{the family is the master civilizer.} A child learns most of the essentials of life from his family and the neighbours.
\vs p082 0:3 The humans of olden times did not possess a very rich social civilization, but such as they had they faithfully and effectively passed on to the next generation. And you should recognize that most of these civilizations of the past continued to evolve with a bare minimum of other institutional influences because the home was effectively functioning. Today the human races possess a rich social and cultural heritage, and it should be wisely and effectively passed on to succeeding generations. The family as an educational institution must be maintained.
\usection{1.\bibnobreakspace The Mating Instinct}
\vs p082 1:1 Notwithstanding the personality gulf between men and women, the sex urge is sufficient to ensure their coming together for the reproduction of the species. This instinct operated effectively long before humans experienced much of what was later called love, devotion, and marital loyalty. Mating is an innate propensity, and marriage is its evolutionary social repercussion.
\vs p082 1:2 Sex interest and desire were not dominating passions in primitive peoples; they simply took them for granted. The entire reproductive experience was free from imaginative embellishment. The all\hyp{}absorbing sex passion of the more highly civilized peoples is chiefly due to race mixtures, especially where the evolutionary nature has been stimulated by the associative imagination and beauty appreciation of the Nodites and Adamites. But this Andite inheritance was absorbed by the evolutionary races in such limited amounts as to fail to provide sufficient self\hyp{}control for the animal passions thus quickened and aroused by the endowment of keener sex consciousness and stronger mating urges. Of the evolutionary races, the red man had the highest sex code.
\vs p082 1:3 \pc The regulation of sex in relation to marriage indicates:
\vs p082 1:4 \ublistelem{1.}\bibnobreakspace The relative progress of civilization. Civilization has increasingly demanded that sex be gratified in useful channels and in accordance with the mores.
\vs p082 1:5 \ublistelem{2.}\bibnobreakspace The amount of Andite stock in any people. Among such groups sex has become expressive of both the highest and the lowest in both the physical and emotional natures.
\vs p082 1:6 \pc The Sangik races had normal animal passion, but they displayed little imagination or appreciation of the beauty and physical attractiveness of the opposite sex. What is called sex appeal is virtually absent even in present\hyp{}day primitive races; these unmixed peoples have a definite mating instinct but insufficient sex attraction to create serious problems requiring social control.
\vs p082 1:7 The mating instinct is one of the dominant physical driving forces of human beings; it is the one emotion which, in the guise of individual gratification, effectively tricks selfish man into putting race welfare and perpetuation high above individual ease and personal freedom from responsibility.
\vs p082 1:8 As an institution, marriage, from its early beginnings down to modern times, pictures the social evolution of the biologic propensity for self\hyp{}perpetuation. The perpetuation of the evolving human species is made certain by the presence of this racial mating impulse, an urge which is loosely called sex attraction. This great biologic urge becomes the impulse hub for all sorts of associated instincts, emotions, and usages --- physical, intellectual, moral, and social.
\vs p082 1:9 With the savage, the food supply was the impelling motivation, but when civilization ensures plentiful food, the sex urge many times becomes a dominant impulse and therefore ever stands in need of social regulation. In animals, instinctive periodicity checks the mating propensity, but since man is so largely a self\hyp{}controlled being, sex desire is not altogether periodic; therefore does it become necessary for society to impose self\hyp{}control upon the individual.
\vs p082 1:10 No human emotion or impulse, when unbridled and overindulged, can produce so much harm and sorrow as this powerful sex urge. Intelligent submission of this impulse to the regulations of society is the supreme test of the actuality of any civilization. Self\hyp{}control, more and more self\hyp{}control, is the ever\hyp{}increasing demand of advancing mankind. Secrecy, insincerity, and hypocrisy may obscure sex problems, but they do not provide solutions, nor do they advance ethics.
\usection{2.\bibnobreakspace The Restrictive Taboos}
\vs p082 2:1 The story of the evolution of marriage is simply the history of sex control through the pressure of social, religious, and civil restrictions. Nature hardly recognizes individuals; it takes no cognizance of so\hyp{}called morals; it is only and exclusively interested in the reproduction of the species. Nature compellingly insists on reproduction but indifferently leaves the consequential problems to be solved by society, thus creating an ever\hyp{}present and major problem for evolutionary mankind. This social conflict consists in the unending war between basic instincts and evolving ethics.
\vs p082 2:2 \pc Among the early races there was little or no regulation of the relations of the sexes. Because of this sex license, no prostitution existed. Today, the Pygmies and other backward groups have no marriage institution; a study of these peoples reveals the simple mating customs followed by primitive races. But all ancient peoples should always be studied and judged in the light of the moral standards of the mores of their own times.
\vs p082 2:3 Free love, however, has never been in good standing above the scale of rank savagery. The moment societal groups began to form, marriage codes and marital restrictions began to develop. Mating has thus progressed through a multitude of transitions from a state of almost complete sex license to the XX century standards of relatively complete sex restriction.
\vs p082 2:4 In the earliest stages of tribal development the mores and restrictive taboos were very crude, but they did keep the sexes apart --- this favoured quiet, order, and industry --- and the long evolution of marriage and the home had begun. The sex customs of dress, adornment, and religious practices had their origin in these early taboos which defined the range of sex liberties and thus eventually created concepts of vice, crime, and sin. But it was long the practice to suspend all sex regulations on high festival days, especially May Day.
\vs p082 2:5 \pc Women have always been subject to more restrictive taboos than men. The early mores granted the same degree of sex liberty to unmarried women as to men, but it has always been required of wives that they be faithful to their husbands. Primitive marriage did not much curtail man’s sex liberties, but it did render further sex license taboo to the wife. Married women have always borne some mark which set them apart as a class by themselves, such as hairdress, clothing, veil, seclusion, ornamentation, and rings.
\usection{3.\bibnobreakspace Early Marriage Mores}
\vs p082 3:1 Marriage is the institutional response of the social organism to the ever\hyp{}present biologic tension of man’s unremitting urge to reproduction --- self\hyp{}propagation. Mating is universally natural, and as society evolved from the simple to the complex, there was a corresponding evolution of the mating mores, the genesis of the marital institution. Wherever social evolution has progressed to the stage at which mores are generated, marriage will be found as an evolving institution.
\vs p082 3:2 There always have been and always will be two distinct realms of marriage: the mores, the laws regulating the external aspects of mating, and the otherwise secret and personal relations of men and women. Always has the individual been rebellious against the sex regulations imposed by society; and this is the reason for this agelong sex problem: Self\hyp{}maintenance is individual but is carried on by the group; self\hyp{}perpetuation is social but is secured by individual impulse.
\vs p082 3:3 The mores, when respected, have ample power to restrain and control the sex urge, as has been shown among all races. Marriage standards have always been a true indicator of the current power of the mores and the functional integrity of the civil government. But the early sex and mating mores were a mass of inconsistent and crude regulations. Parents, children, relatives, and society all had conflicting interests in the marriage regulations. But in spite of all this, those races which exalted and practised marriage naturally evolved to higher levels and survived in increased numbers.
\vs p082 3:4 \pc In primitive times marriage was the price of social standing; the possession of a wife was a badge of distinction. The savage looked upon his wedding day as marking his entrance upon responsibility and manhood. In one age, marriage has been looked upon as a social duty; in another, as a religious obligation; and in still another, as a political requirement to provide citizens for the state.
\vs p082 3:5 Many early tribes required feats of stealing as a qualification for marriage; later peoples substituted for such raiding forays, athletic contests and competitive games. The winners in these contests were awarded the first prize --- choice of the season’s brides. Among the head\hyp{}hunters a youth might not marry until he possessed at least one head, although such skulls were sometimes purchasable. As the buying of wives declined, they were won by riddle contests, a practice that still survives among many groups of the black man.
\vs p082 3:6 With advancing civilization, certain tribes put the severe marriage tests of male endurance in the hands of the women; they thus were able to favour the men of their choice. These marriage tests embraced skill in hunting, fighting, and ability to provide for a family. The groom was long required to enter the bride’s family for at least one year, there to live and labour and prove that he was worthy of the wife he sought.
\vs p082 3:7 The qualifications of a wife were the ability to perform hard work and to bear children. She was required to execute a certain piece of agricultural work within a given time. And if she had borne a child before marriage, she was all the more valuable; her fertility was thus assured.
\vs p082 3:8 \pc The fact that ancient peoples regarded it as a disgrace, or even a sin, not to be married, explains the origin of child marriages; since one must be married, the earlier the better. It was also a general belief that unmarried persons could not enter spiritland, and this was a further incentive to child marriages even at birth and sometimes before birth, contingent upon sex. The ancients believed that even the dead must be married. The original matchmakers were employed to negotiate marriages for deceased individuals. One parent would arrange for these intermediaries to effect the marriage of a dead son with a dead daughter of another family.
\vs p082 3:9 Among later peoples, puberty was the common age of marriage, but this has advanced in direct proportion to the progress of civilization. Early in social evolution peculiar and celibate orders of both men and women arose; they were started and maintained by individuals more or less lacking normal sex urge.
\vs p082 3:10 Many tribes allowed members of the ruling group to have sex relations with the bride just before she was to be given to her husband. Each of these men would give the girl a present, and this was the origin of the custom of giving wedding presents. Among some groups it was expected that a young woman would earn her dowry, which consisted of the presents received in reward for her sex service in the bride’s exhibition hall.
\vs p082 3:11 Some tribes married the young men to the widows and older women and then, when they were subsequently left widowers, would allow them to marry the young girls, thus ensuring, as they expressed it, that both parents would not be fools, as they conceived would be the case if two youths were allowed to mate. Other tribes limited mating to similar age groups. It was the limitation of marriage to certain age groups that first gave origin to ideas of incest. (In India there are even now no age restrictions on marriage.)
\vs p082 3:12 \pc Under certain mores widowhood was greatly to be feared, widows being either killed or allowed to commit suicide on their husbands’ graves, for they were supposed to go over into spiritland with their spouses. The surviving widow was almost invariably blamed for her husband’s death. Some tribes burned them alive. If a widow continued to live, her life was one of continuous mourning and unbearable social restriction since remarriage was generally disapproved.
\vs p082 3:13 In olden days many practices now regarded as immoral were encouraged. Primitive wives not infrequently took great pride in their husbands’ affairs with other women. Chastity in girls was a great hindrance to marriage; the bearing of a child before marriage greatly increased a girl’s desirability as a wife since the man was sure of having a fertile companion.
\vs p082 3:14 Many primitive tribes sanctioned trial marriage until the woman became pregnant, when the regular marriage ceremony would be performed; among other groups the wedding was not celebrated until the first child was born. If a wife was barren, she had to be redeemed by her parents, and the marriage was annulled. The mores demanded that every pair have children.
\vs p082 3:15 These primitive trial marriages were entirely free from all semblance of license; they were simply sincere tests of fecundity. The contracting individuals married permanently just as soon as fertility was established. When modern couples marry with the thought of convenient divorce in the background of their minds if they are not wholly pleased with their married life, they are in reality entering upon a form of trial marriage and one that is far beneath the status of the honest adventures of their less civilized ancestors.
\usection{4.\bibnobreakspace Marriage under the Property Mores}
\vs p082 4:1 Marriage has always been closely linked with both property and religion. Property has been the stabilizer of marriage; religion, the moralizer.
\vs p082 4:2 Primitive marriage was an investment, an economic speculation; it was more a matter of business than an affair of flirtation. The ancients married for the advantage and welfare of the group; wherefore their marriages were planned and arranged by the group, their parents and elders. And that the property mores were effective in stabilizing the marriage institution is borne out by the fact that marriage was more permanent among the early tribes than it is among many modern peoples.
\vs p082 4:3 As civilization advanced and private property gained further recognition in the mores, stealing became the great crime. Adultery was recognized as a form of stealing, an infringement of the husband’s property rights; it is not therefore specifically mentioned in the earlier codes and mores. Woman started out as the property of her father, who transferred his title to her husband, and all legalized sex relations grew out of these pre\hyp{}existent property rights. The Old Testament deals with women as a form of property; the Koran teaches their inferiority. Man had the right to lend his wife to a friend or guest, and this custom still obtains among certain peoples.
\vs p082 4:4 Modern sex jealousy is not innate; it is a product of the evolving mores. Primitive man was not jealous of his wife; he was just guarding his property. The reason for holding the wife to stricter sex account than the husband was because her marital infidelity involved descent and inheritance. Very early in the march of civilization the illegitimate child fell into disrepute. At first only the woman was punished for adultery; later on, the mores also decreed the chastisement of her partner, and for long ages the offended husband or the protector father had the full right to kill the male trespasser. Modern peoples retain these mores, which allow so\hyp{}called crimes of honour under the unwritten law.
\vs p082 4:5 Since the chastity taboo had its origin as a phase of the property mores, it applied at first to married women but not to unmarried girls. In later years, chastity was more demanded by the father than by the suitor; a virgin was a commercial asset to the father --- she brought a higher price. As chastity came more into demand, it was the practice to pay the father a bride fee in recognition of the service of properly rearing a chaste bride for the husband\hyp{}to\hyp{}be. When once started, this idea of female chastity took such hold on the races that it became the practice literally to cage up girls, actually to imprison them for years, in order to assure their virginity. And so the more recent standards and virginity tests automatically gave origin to the professional prostitute classes; they were the rejected brides, those women who were found by the grooms’ mothers not to be virgins.
\usection{5.\bibnobreakspace Endogamy and Exogamy}
\vs p082 5:1 Very early the savage observed that race mixture improved the quality of the offspring. It was not that inbreeding was always bad, but that outbreeding was always comparatively better; therefore the mores tended to crystallize in restriction of sex relations among near relatives. It was recognized that outbreeding greatly increased the selective opportunity for evolutionary variation and advancement. The outbred individuals were more versatile and had greater ability to survive in a hostile world; the inbreeders, together with their mores, gradually disappeared. This was all a slow development; the savage did not consciously reason about such problems. But the later and advancing peoples did, and they also made the observation that general weakness sometimes resulted from excessive inbreeding.
\vs p082 5:2 While the inbreeding of good stock sometimes resulted in the upbuilding of strong tribes, the spectacular cases of the bad results of the inbreeding of hereditary defectives more forcibly impressed the mind of man, with the result that the advancing mores increasingly formulated taboos against all marriages among near relatives.
\vs p082 5:3 \pc Religion has long been an effective barrier against outmarriage; many religious teachings have proscribed marriage outside the faith. Woman has usually favoured the practice of in\hyp{}marriage; man, outmarriage. Property has always influenced marriage, and sometimes, in an effort to conserve property within a clan, mores have arisen compelling women to choose husbands within their fathers’ tribes. Rulings of this sort led to a great multiplication of cousin marriages. In\hyp{}mating was also practised in an effort to preserve craft secrets; skilled workmen sought to keep the knowledge of their craft within the family.
\vs p082 5:4 \pc Superior groups, when isolated, always reverted to consanguineous mating. The Nodites for over 150,000 years were one of the great in\hyp{}marriage groups. The later\hyp{}day in\hyp{}marriage mores were tremendously influenced by the traditions of the violet race, in which, at first, matings were, perforce, between brother and sister. And brother and sister marriages were common in early Egypt, Syria, Mesopotamia, and throughout the lands once occupied by the Andites. The Egyptians long practised brother and sister marriages in an effort to keep the royal blood pure, a custom which persisted even longer in Persia. Among the Mesopotamians, before the days of Abraham, cousin marriages were obligatory; cousins had prior marriage rights to cousins. Abraham himself married his half sister, but such unions were not allowed under the later mores of the Jews.
\vs p082 5:5 The first move away from brother and sister marriages came about under the plural\hyp{}wife mores because the sister\hyp{}wife would arrogantly dominate the other wife or wives. Some tribal mores forbade marriage to a dead brother’s widow but required the living brother to beget children for his departed brother. There is no biologic instinct against any degree of in\hyp{}marriage; such restrictions are wholly a matter of taboo.
\vs p082 5:6 \pc Outmarriage finally dominated because it was favoured by the man; to get a wife from the outside ensured greater freedom from in\hyp{}laws. Familiarity breeds contempt; so, as the element of individual choice began to dominate mating, it became the custom to choose partners from outside the tribe.
\vs p082 5:7 Many tribes finally forbade marriages within the clan; others limited mating to certain castes. The taboo against marriage with a woman of one’s own totem gave impetus to the custom of stealing women from neighbouring tribes. Later on, marriages were regulated more in accordance with territorial residence than with kinship. There were many steps in the evolution of in\hyp{}marriage into the modern practice of outmarriage. Even after the taboo rested upon in\hyp{}marriages for the common people, chiefs and kings were permitted to marry those of close kin in order to keep the royal blood concentrated and pure. The mores have usually permitted sovereign rulers certain licenses in sex matters.
\vs p082 5:8 The presence of the later Andite peoples had much to do with increasing the desire of the Sangik races to mate outside their own tribes. But it was not possible for out\hyp{}mating to become prevalent until neighbouring groups had learned to live together in relative peace.
\vs p082 5:9 Outmarriage itself was a peace promoter; marriages between the tribes lessened hostilities. Outmarriage led to tribal co\hyp{}ordination and to military alliances; it became dominant because it provided increased strength; it was a nation builder. Outmarriage was also greatly favoured by increasing trade contacts; adventure and exploration contributed to the extension of the mating bounds and greatly facilitated the cross\hyp{}fertilization of racial cultures.
\vs p082 5:10 The otherwise inexplicable inconsistencies of the racial marriage mores are largely due to this outmarriage custom with its accompanying wife stealing and buying from foreign tribes, all of which resulted in a compounding of the separate tribal mores. That these taboos respecting in\hyp{}marriage were sociologic, not biologic, is well illustrated by the taboos on kinship marriages, which embraced many degrees of in\hyp{}law relationships, cases representing no blood relation whatsoever.
\usection{6.\bibnobreakspace Racial Mixtures}
\vs p082 6:1 There are no pure races in the world today. The early and original evolutionary peoples of colour have only two representative races persisting in the world, the yellow man and the black man; and even these two races are much admixed with the extinct coloured peoples. While the so\hyp{}called white race is predominantly descended from the ancient blue man, it is admixed more or less with all other races much as is the red man of the Americas.
\vs p082 6:2 Of the six coloured Sangik races, three were primary and three were secondary. Though the primary races --- blue, red, and yellow --- were in many respects superior to the three secondary peoples, it should be remembered that these secondary races had many desirable traits which would have considerably enhanced the primary peoples if their better strains could have been absorbed.
\vs p082 6:3 Present\hyp{}day prejudice against “half\hyp{}castes,” “hybrids,” and “mongrels” arises because modern racial crossbreeding is, for the greater part, between the grossly inferior strains of the races concerned. You also get unsatisfactory offspring when the degenerate strains of the same race intermarry.
\vs p082 6:4 If the present\hyp{}day races of Urantia could be freed from the curse of their lowest strata of deteriorated, antisocial, feeble\hyp{}minded, and outcast specimens, there would be little objection to a limited race amalgamation. And if such racial mixtures could take place between the highest types of the several races, still less objection could be offered.
\vs p082 6:5 Hybridization of superior and dissimilar stocks is the secret of the creation of new and more vigorous strains. And this is true of plants, animals, and the human species. Hybridization augments vigour and increases fertility. Race mixtures of the average or superior strata of various peoples greatly increase \bibemph{creative} potential, as is shown in the present population of the United States of North America\fnst{\textbf{United States of North America}, This may explain why \bibemph{English} language was chosen to convey this great epochal revelation to all Urantia mortals.}. When such matings take place between the lower or inferior strata, creativity is diminished, as is shown by the present\hyp{}day peoples of southern India.
\vs p082 6:6 Race blending greatly contributes to the sudden appearance of \bibemph{new} characteristics, and if such hybridization is the union of superior strains, then these new characteristics will also be \bibemph{superior} traits.
\vs p082 6:7 As long as present\hyp{}day races are so overloaded with inferior and degenerate strains, race intermingling on a large scale would be most detrimental, but most of the objections to such experiments rest on social and cultural prejudices rather than on biological considerations. Even among inferior stocks, hybrids often are an improvement on their ancestors. Hybridization makes for species improvement because of the role of the \bibemph{dominant genes.} Racial intermixture increases the likelihood of a larger number of the desirable \bibemph{dominants} being present in the hybrid.
\vs p082 6:8 \pc For the past 100 years more racial hybridization has been taking place on Urantia than has occurred in thousands of years. The danger of gross disharmonies as a result of crossbreeding of human stocks has been greatly exaggerated. The chief troubles of “half\hyp{}breeds” are due to social prejudices.
\vs p082 6:9 The Pitcairn experiment of blending the white and Polynesian races turned out fairly well because the white men and the Polynesian women were of fairly good racial strains. Interbreeding between the highest types of the white, red, and yellow races would immediately bring into existence many new and biologically effective characteristics. These three peoples belong to the primary Sangik races. Mixtures of the white and black races are not so desirable in their immediate results, neither are such mulatto offspring so objectionable as social and racial prejudice would seek to make them appear. Physically, such white\hyp{}black hybrids are excellent specimens of humanity, notwithstanding their slight inferiority in some other respects.
\vs p082 6:10 \pc When a primary Sangik race amalgamates with a secondary Sangik race, the latter is considerably improved at the expense of the former. And on a small scale --- extending over long periods of time --- there can be little serious objection to such a sacrificial contribution by the primary races to the betterment of the secondary groups. Biologically considered, the secondary Sangiks were in some respects superior to the primary races.
\vs p082 6:11 After all, the real jeopardy of the human species is to be found in the unrestrained multiplication of the inferior and degenerate strains of the various civilized peoples rather than in any supposed danger of their racial interbreeding.
\vsetoff
\vs p082 6:12 [Presented by the Chief of Seraphim stationed on Urantia.]
\quizlink
